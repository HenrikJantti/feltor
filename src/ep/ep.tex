%\documentclass[12pt]{article}
%\documentclass[12pt]{scrartcl}
\documentclass{hitec}
\settextfraction{0.9} % indent text
\usepackage{csquotes}
\usepackage[hidelinks]{hyperref} % doi links are short and usefull?
\hypersetup{%
    colorlinks=true,
    linkcolor=blue,
    urlcolor=magenta
}
\urlstyle{rm}
\usepackage[english]{babel}
\usepackage{mathtools} % loads and extends amsmath
\usepackage{amssymb}
% packages not used
%\usepackage{graphicx}
%\usepackage{amsthm}
%\usepackage{subfig}
\usepackage{bm}
\usepackage{longtable}
\usepackage{booktabs}
\usepackage{ragged2e} % maybe use \RaggedRight for tables and literature?
\usepackage[table]{xcolor} % for alternating colors
\rowcolors{2}{gray!25}{white}
\renewcommand\arraystretch{1.3}

%%% reset bibliography distances %%%
\let\oldthebibliography\thebibliography
\let\endoldthebibliography\endthebibliography
\renewenvironment{thebibliography}[1]{
  \begin{oldthebibliography}{#1}
    \RaggedRight % remove if justification is desired
    \setlength{\itemsep}{0em}
    \setlength{\parskip}{0em}
}
{
  \end{oldthebibliography}
}
%%% --- %%%

%%%%%%%%%%%%%%%%%%%%%definitions%%%%%%%%%%%%%%%%%%%%%%%%%%%%%%%%%%%%%%%
\newcommand{\eps}{\varepsilon}
\renewcommand{\d}{\mathrm{d}}
\newcommand{\T}{\mathrm{T}}
\renewcommand{\vec}[1]{{\mathbf{#1}}}
\newcommand{\dx}{\,\mathrm{d}x}
%\newcommand{\dA}{\,\mathrm{d}(x,y)}
%\newcommand{\dV}{\mathrm{d}^3{x}\,}
\newcommand{\dA}{\,\mathrm{dA}}
\newcommand{\dV}{\mathrm{dV}\,}
\newcommand{\Eins}{\mathbf{1}}
\newcommand{\ExB}{$\bm{E}\times\bm{B} \,$}
\newcommand{\GKI}{\int d^6 \bm{Z} \BSP}
\newcommand{\GKIV}{\int dv_{\|} d \mu d \theta \BSP}
\newcommand{\BSP}{B_{\|}^*}
\newcommand{\GA}[1]{\langle #1   \rangle}
\newcommand{\Abar}{\langle A_\parallel \rangle}
%Vectors
\newcommand{\bhat}{\bm{\hat{b}}}
\newcommand{\bbar}{\overline{\bm{b}}}
\newcommand{\chat}{\bm{\hat{c}}}
\newcommand{\ahat}{\bm{\hat{a}}}
\newcommand{\xhat}{\bm{\hat{x}}}
\newcommand{\yhat}{\bm{\hat{y}}}
\newcommand{\zhat}{\bm{\hat{z}}}
\newcommand{\Xbar}{\bar{\vec{X}}}
\newcommand{\phat}{\bm{\hat{\perp}}}
\newcommand{\that}{\bm{\hat{\theta}}}
\newcommand{\eI}{\bm{\hat{e}}_1}
\newcommand{\eII}{\bm{\hat{e}}_2}
\newcommand{\ud}{\mathrm{d}}
%Derivatives etc.
\newcommand{\pfrac}[2]{\frac{\partial#1}{\partial#2}}
\newcommand{\ffrac}[2]{\frac{\delta#1}{\delta#2}}
\newcommand{\fixd}[1]{\Big{\arrowvert}_{#1}}
\newcommand{\curl}[1]{\nabla \times #1}
\newcommand{\np}{\nabla_{\perp}}
\newcommand{\npc}{\nabla_{\perp} \cdot }
\newcommand{\nc}{\nabla\cdot }
\newcommand{\GAI}{\Gamma_{1}^{\dagger}}
\newcommand{\GAII}{\Gamma_{1}^{\dagger -1}}

%%%%%%%%%%%%%%%%%%%%%%%%%%%%%DOCUMENT%%%%%%%%%%%%%%%%%%%%%%%%%%%%%%%%%%%%%%%
\begin{document}

\title{The electron-positron plasma project}
\maketitle

\begin{abstract}
This is a program for 2d isothermal blob simulations.
\end{abstract}

\section{Equations}
$N_e$ is the electron density, $N_p$ is the positron density. 
 $\phi$ is the electric potential. We use Cartesian coordinates $x$, $y$. 

\subsection{Model}

"global"
\begin{subequations}
\begin{align}
B(x)^{-1} = \kappa x +1-\kappa X\quad \Gamma_1 = ( 1- 0.5\mu\tau\nabla^2)^{-1}\\
 -\nabla\cdot \left( \varepsilon + \frac{z_e\mu_e N_e + z_p\mu_pN_p}{B^2} \nabla \phi\right) = z_p \Gamma_{1p} N_p +z_e\Gamma_{1e} N_e, \quad
\psi = \Gamma_1 \phi - \frac{1}{2} \mu \frac{(\nabla\phi)^2}{B^2}\\
  \frac{\partial N}{\partial t} =
  \frac{1}{B}\{ N, \psi\} 
  + \kappa N\frac{\partial \psi}{\partial y} 
  + \tau \kappa\frac{\partial N}{\partial y} +\nu\nabla^2N
\end{align}
\end{subequations}

\subsection{Initialization}
We define a Gaussian 
\begin{align}
    n(x,y) = 1 + A\exp\left( -\frac{(x-X)^2 + (y-Y)^2}{2\sigma^2}\right)
    \label{}
\end{align}
where $X = p_x l_x$ and $Y=p_yl_y$ are the initial centre of mass position coordinates, $A$ is the amplitude and $\sigma$ the
radius of the blob.
We initialize 
\begin{align}
    N_s = \Gamma_{1s}^{-1} n \quad \phi = 0 
    \label{}
\end{align}
\subsection{Diagnostics}
\begin{align}
    M(t) = \int n-1 \\
    \Lambda_n = \nu \int \Delta n  \\
    ...
    \label{}
\end{align}
\section{Numerical methods}
discontinuous Galerkin on structured grid
\begin{longtable}{ll>{\RaggedRight}p{7cm}}
\toprule
\rowcolor{gray!50}\textbf{Term} &  \textbf{Method} & \textbf{Description}  \\ \midrule
coordinate system & Cartesian 2D & equidistant discretization of $[0,l_x] \times [0,l_y]$, equal number of Gaussian nodes in x and y \\
matrix inversions & conjugate gradient & Use previous two solutions to extrapolate initial guess and $1/\chi$ as preconditioner \\
\ExB advection & Arakawa & - \\
curvature terms & direct & flux conserving \\
time &  Karniadakis multistep & $3rd$ order explicit, diffusion $2nd$ order implicit \\
\bottomrule
\end{longtable}

\section{Compilation and useage}
There are two programs toeflR.cu, and toefl\_mpi.cu . Compilation with 
\begin{verbatim}
make device = <omp or gpu>
\end{verbatim}
Run with
\begin{verbatim}
path/to/feltor/src/ep/toeflR input.json
echo np_x np_y | mpirun -n np_x*np_y path/to/feltor/src/ep/toefl_mpi\
    input.json output.nc
\end{verbatim}
All programs write performance informations to std::cout.
The first opens a terminal window with life simulation results the
other two write the results to disc. 
The second for distributed
memory systems, which expects the distribution of processes in the
x and y directions.

\subsection{Input file structure}
Input file format: json

%%This is a booktabs table
\begin{longtable}{llll>{\RaggedRight}p{7cm}}
\toprule
\rowcolor{gray!50}\textbf{Name} &  \textbf{Type} & \textbf{Example} & \textbf{Default} & \textbf{Description}  \\ \midrule
n      & integer & 3 & - &\# Gaussian nodes in x and y \\
Nx     & integer &100& - &\# grid points in x \\
Ny     & integer &100& - &\# grid points in y \\
dt     & integer &3.0& - &time step in units of $c_s/\rho_s$ \\
n\_out  & integer &3  & - &\# Gaussian nodes in x and y in output \\
Nx\_out & integer &100& - &\# grid points in x in output fields \\
Ny\_out & integer &100& - &\# grid points in y in output fields \\
itstp  & integer &2  & - &   steps between outputs \\
maxout & integer &100& - &      \# outputs excluding first \\
eps\_pol   & float &1e-6    & - &  accuracy of polarisation solver \\
eps\_gamma & float &1e-7    & - & accuracy of $\Gamma_1$ (only in gyrofluid model) \\
eps\_time  & float &1e-10   & - & accuracy of implicit time-stepper \\
curvature  & float &0.00015& - & magnetic curvature $\kappa$ \\
tau        & float array &[-1., 1.]  & - & $\tau_s = T_s/Z_s T_e$\\
mu         & float array &[-1., 1.]  & - & $\mu_s = m_s/Z_s m_e$\\
z          & integer array  & [-1,1]  & - & charge number\\
nu\_perp    & float &5e-3   & - & pependicular viscosity $\nu$ \\
amplitude  & float &1.0    & - & amplitude $A$ of the blob \\
sigma      & float &10     & - & blob radius $\sigma$ \\
posX       & float &0.3    & - & blob x-position in units of $l_x$, i.e. $X = p_x l_x$\\
posY       & float &0.5    & - & blob y-position in units of $l_y$, i.e. $Y = p_y l_y$ \\
lx         & float &200    & - & $l_x$  \\
ly         & float &200    & - & $l_y$  \\
bc\_x   & char & "DIR"      & - & boundary condition in x (one of PER, DIR, NEU, DIR\_NEU or NEU\_DIR) \\
bc\_y   & char & "PER"      & - & boundary condition in y (one of PER, DIR, NEU, DIR\_NEU or NEU\_DIR) \\
debye  & float & 0 & 0 & debye length parameter\\
\bottomrule
\end{longtable}

The default value is taken if the value name is not found in the input file. If there is no default and
the value is not found,
the program exits with an error message.

\subsection{Structure of output file}
Output file format: netcdf-4/hdf5
%
%Name | Type | Dimensionality | Description
%---|---|---|---|
\begin{longtable}{lll>{\RaggedRight}p{7cm}}
\toprule
\rowcolor{gray!50}\textbf{Name} &  \textbf{Type} & \textbf{Dimension} & \textbf{Description}  \\ \midrule
inputfile  &             text attribute & 1 & verbose input file as a string \\
energy\_time             & Dataset & 1 & timesteps at which 1d variables are written \\
time                     & Dataset & 1 & time at which fields are written \\
x                        & Dataset & 1 & x-coordinate  \\
y                        & Dataset & 1 & y-coordinate \\
electrons                & Dataset & 3 (time, y, x) & electon density $n$ \\
positrons                & Dataset & 3 (time, y, x) & positron density \\
potential                & Dataset & 3 (time, y, x) & electric potential $\phi$  \\
vorticity                & Dataset & 3 (time, y, x) & Laplacian of potential $\nabla^2\phi$  \\
dEdt                     & Dataset & 1 (energy\_time) & change of energy per time  \\
dissipation              & Dataset & 1 (energy\_time) & diffusion integrals  \\
energy                   & Dataset & 1 (energy\_time) & total energy integral  \\
mass                     & Dataset & 1 (energy\_time) & mass integral   \\
\bottomrule
\end{longtable}
\section{Diagnostics toeflEPdiag.cu}
There only is a shared memory version available
\begin{verbatim}
cd path/to/feltor/diag
make toeflEPdiag 
path/to/feltor/diag/toeflEPdiag input.nc output.nc
\end{verbatim}

Input file format: netcdf-4/hdf5
%
%Name | Type | Dimensionality | Description
%---|---|---|---|
\begin{longtable}{lll>{\RaggedRight}p{7cm}}
\toprule
\rowcolor{gray!50}\textbf{Name} &  \textbf{Type} & \textbf{Dimension} & \textbf{Description}  \\ \midrule
inputfile  &             text attribute & 1 & verbose input file as a string \\
electrons                & Dataset & 3 & electon density (time, y, x) \\
positrons                & Dataset & 3 & positron density (time, y, x) \\
potential                & Dataset & 3 & electric potential (time, y, x) \\
\bottomrule
\end{longtable}

Output file format: netcdf-4/hdf5
%
%Name | Type | Dimensionality | Description
%---|---|---|---|
\begin{longtable}{lll>{\RaggedRight}p{7cm}}
\toprule
\rowcolor{gray!50}\textbf{Name} &  \textbf{Type} & \textbf{Dimension} & \textbf{Description}  \\ \midrule
 inputfile & text attribute & 1 & copy of inputfile attribute of the input file (the json string of the simulation input file) \\
 time & Dataset & 1 & the time steps at which variables are written \\
 posX & Dataset & 1 (time) & centre of mass (COM) position x-coordinate \\
 posY & Dataset & 1 (time) &COM y-position \\
 velX & Dataset & 1 (time)& COM x-velocity \\
 velY & Dataset & 1 (time)& COM y-velocity \\
 accX & Dataset & 1 (time)& COM x-acceleration \\
 accY & Dataset & 1 (time)& COM y-acceleration \\
 velCOM & Dataset & 1 (time)&absolute value of the COM velocity \\
 posXmax& Dataset & 1 (time)&maximum amplitude x-position \\
 posYmax& Dataset & 1 (time)&maximum amplitude y-position \\
 velXmax& Dataset & 1 (time)&maximum amplitude x-velocity \\
 velYmax& Dataset & 1 (time)&maximum amplitude y-velocity \\
 maxamp & Dataset & 1 (time)&value of the maximum amplitude  \\
  compactness\_ne& Dataset & 1 (time) &compactness of the density field \\
 Ue& Dataset&  1 (time) &entropy electrons \\
 Ui &Dataset& 1 (time) & entropy ions \\
 Uphi& Dataset& 1 (time) &  exb energy \\
 mass& Dataset & 1 (time) & mass of the blob without background \\
\bottomrule
\end{longtable}
%..................................................................
\begin{thebibliography}{1}
    \bibitem{Kendl2017}
    A. Kendl, G. Danler, M. Wiesenberger, and M. Held, "Interchange instability in matter-antimatter plasmas" Phys. Rev. Letters (2017) https://arxiv.org/abs/1611.07836
\end{thebibliography}
%..................................................................


\end{document}
