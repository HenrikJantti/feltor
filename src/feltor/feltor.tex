%%%%%%%%%%%%%%%%%%%%%definitions%%%%%%%%%%%%%%%%%%%%%%%%%%%%%%%%%%%%%%%

%\documentclass[12pt]{article}
%\documentclass[12pt]{scrartcl}
\documentclass{hitec} % contained in texlive-latex-extra
\settextfraction{0.9} % indent text
\usepackage{csquotes}
\usepackage[hidelinks]{hyperref} % doi links are short and usefull?
\hypersetup{%
    colorlinks=true,
    linkcolor=blue,
    urlcolor=magenta
}
\urlstyle{rm}
\usepackage[english]{babel}
\usepackage{mathtools} % loads and extends amsmath
\usepackage{amssymb}
% packages not used
%\usepackage{graphicx}
%\usepackage{amsthm}
%\usepackage{subfig}
\usepackage{bm}
\usepackage{longtable}
\usepackage{booktabs}
\usepackage{ragged2e} % maybe use \RaggedRight for tables and literature?
\usepackage[table]{xcolor} % for alternating colors
%\rowcolors{2}{gray!25}{white} %%% Use this line in front of longtable
\renewcommand\arraystretch{1.3}
\usepackage[most]{tcolorbox}
\usepackage{doi}
\usepackage[sort,square,numbers]{natbib}
\bibliographystyle{abbrvnat}
%%% reset bibliography distances %%%
\let\oldthebibliography\thebibliography
\let\endoldthebibliography\endthebibliography
\renewenvironment{thebibliography}[1]{
  \begin{oldthebibliography}{#1}
    \RaggedRight % remove if justification is desired
    \setlength{\itemsep}{0em}
    \setlength{\parskip}{0em}
}
{
  \end{oldthebibliography}
}
%%% --- %%%

\definecolor{light-gray}{gray}{0.95}
\newcommand{\code}[1]{\colorbox{light-gray}{\texttt{#1}}}
\newcommand{\eps}{\varepsilon}
\renewcommand{\d}{\mathrm{d}}
\renewcommand{\vec}[1]{{\boldsymbol{#1}}}
\newcommand{\dx}{\,\mathrm{d}x}
%\newcommand{\dA}{\,\mathrm{d}(x,y)}
%\newcommand{\dV}{\mathrm{d}^3{x}\,}
\newcommand{\dA}{\,\mathrm{dA}}
\newcommand{\dV}{\mathrm{dV}\,}

\newcommand{\Eins}{\mathbf{1}}

\newcommand{\ExB}{$\bm{E}\times\bm{B} \,$}
\newcommand{\GKI}{\int d^6 \bm{Z} \BSP}
\newcommand{\GKIV}{\int dv_{\|} d \mu d \theta \BSP}
\newcommand{\BSP}{B_{\|}^*}
\newcommand{\Abar}{\langle A_\parallel \rangle}
%Averages
\newcommand{\RA}[1]{\left \langle #1 \right \rangle} %Reynolds (flux-surface) average
\newcommand{\RF}[1]{\widetilde{#1}} %Reynolds fluctuation
\newcommand{\FA}[1]{\left[\left[ #1 \right]\right]} %Favre average
\newcommand{\FF}[1]{\widehat{#1}} %Favre fluctuation
\newcommand{\PA}[1]{\left \langle #1 \right\rangle_\varphi} %Phi average

%Vectors
\newcommand{\ahat}{\bm{\hat{a}}}
\newcommand{\bhat}{\bm{\hat{b}}}
\newcommand{\chat}{\bm{\hat{c}}}
\newcommand{\ehat}{\bm{\hat{e}}}
\newcommand{\bbar}{\overline{\bm{b}}}
\newcommand{\xhat}{\bm{\hat{x}}}
\newcommand{\yhat}{\bm{\hat{y}}}
\newcommand{\zhat}{\bm{\hat{z}}}

\newcommand{\Xbar}{\bar{\vec{X}}}
\newcommand{\phat}{\bm{\hat{\perp}}}
\newcommand{\that}{\bm{\hat{\theta}}}

\newcommand{\eI}{\bm{\hat{e}}_1}
\newcommand{\eII}{\bm{\hat{e}}_2}
\newcommand{\ud}{\mathrm{d}}

%Derivatives etc.
\newcommand{\pfrac}[2]{\frac{\partial#1}{\partial#2}}
\newcommand{\ffrac}[2]{\frac{\delta#1}{\delta#2}}
\newcommand{\fixd}[1]{\Big{\arrowvert}_{#1}}
\newcommand{\curl}[1]{\nabla \times #1}

\newcommand{\np}{\vec{\nabla}_{\perp}}
\newcommand{\npc}{\nabla_{\perp} \cdot }
\newcommand{\nc}{\vec\nabla\cdot}
\newcommand{\cn}{\cdot\vec\nabla}
\newcommand{\vn}{\vec{\nabla}}
\newcommand{\npar}{\nabla_\parallel}

\newcommand{\GAI}{\Gamma_{1}^{\dagger}}
\newcommand{\GAII}{\Gamma_{1}^{\dagger -1}}
\newcommand{\T}{\mathrm{T}}
\newcommand{\Tp}{\mathcal T^+_{\Delta\varphi}}
\newcommand{\Tm}{\mathcal T^-_{\Delta\varphi}}
\newcommand{\Tpm}{\mathcal T^\pm_{\Delta\varphi}}
%%%%%%%%Some useful abbreviations %%%%%%%%%%%%%%%%
\def\feltor{{\textsc{Feltor }}}

\def\fixme#1{\typeout{FIXME in page \thepage :{#1}}%
 \textsc{\color{red}[{#1}]}}



%%%%%%%%%%%%%%%%%%%%%%%%%%%%%DOCUMENT%%%%%%%%%%%%%%%%%%%%%%%%%%%%%%%%%%%%%%%
\begin{document}

\title{
The full-F electromagnetic model in toroidal geometry \textsc{Feltor}}
\author{ M.~Wiesenberger and M.~Held}
\maketitle

\begin{abstract}
The purpose of this document is to describe the programs
\texttt{feltor\_hpc.cu, feltor.cu, feltor\_diag.cu} and to an extend
\texttt{geometry\_diag.cu}. The goal is to provide
information such that a user can avoid to look
into the actual codes on the one side and connect
the presented formulas to relevant journal publications on the other.

The program \texttt{feltor/inc/geometries/geometry\_diag.cu}
analyses the magnetic field geometry.
\texttt{feltor\_hpc.cu} and \texttt{feltor.cu} are programs for global 3d isothermal electromagnetic full-F gyro-fluid simulations.
\texttt{feltor/diag/feltor\_diag.cu} is a program to analyse the output
file(s) of \texttt{feltor\_hpc.cu}.

\end{abstract}

%%%%%%%%%%%%%%%%%%%%%%%%%%%%%%%%%%%%%%%%%%%%%%%%%%%%%%%%%%%%%%%%%%%%%%%%%%%%%%%%%
\section{The magnetic field}\label{sec:magnetic}
Let us assume a three-dimensional flat space with arbitrary coordinate
system $\vec x :=\{x_0, x_1, x_2\}$, metric
tensor $g_{ij}$ and volume element $\sqrt{g} := \sqrt{\det g}$.
Given a vector field $\vec B(\vec x)$ with unit vector $\bhat(\vec x) := (\vec B/B)({\vec x})$
we can define various differential operations.
%Let us further assume that $\bhat$ is perturbed by the parallel
%vector potential $A_\parallel$ via
%$\tilde{ \vec b }_\perp := ({\nabla \times A_\parallel \bhat)}/{B}$
\rowcolors{2}{gray!25}{white}
%\begin{longtable}{>{\RaggedRight}p{7cm}>{\RaggedRight}p{7cm}}
\begin{longtable}{lll>{\RaggedRight}p{7cm}}
%\toprule
\rowcolor{gray!50}\textbf{Name} &  \textbf{Symbol} & \textbf{Definition} \\
\midrule
    Perpendicular Poisson bracket&
    $\left[.,.\right]_\perp$ &
    $\left[f,g\right]_\perp := \bhat \cdot \left(\vec{\nabla} f \times \vec\nabla g\right) =
    b_i \varepsilon^{ijk}\partial_j f\partial_k g/\sqrt{g}$  \\
    Projection Tensor&
    $h $ & $h^{ij} := g^{ij} - b^ib^j $\\
    %Alignment Tensor&
    %$t $ & $ t^{ij} := b^ib^j$\\
    Perpendicular Gradient&
    $\vec \nabla_\perp $&
    $ \vec \nabla_\perp f := \bhat\times(\vec \nabla f\times \bhat ) \equiv
    h \cdot \nabla f$ \\
    Perpendicular Laplacian&
    $\Delta_\perp $&
    $ \Delta_\perp f:= \vec \nabla\cdot (\vec \nabla_\perp f)
    = \nabla\cdot( h\cdot\nabla f)$  \\
    Curl-b Curvature &
    $\mathcal K_{\nabla\times\bhat}$ &
    $\mathcal K_{\nabla\times\bhat}(f) := \vec{ \mathcal K_{\nabla\times\bhat} }\cdot \vec \nabla f = \frac{1}{B}(\nabla \times \bhat)\cdot \vec \nabla f$ \\[4pt]
    Grad-B Curvature &
    $\mathcal K_{\nabla B} $ &
    $\mathcal K_{\nabla B}(f) := \vec{\mathcal K_{\nabla B}} \cdot \vec \nabla f = \frac{1}{B}(\bhat \times \vec \nabla \ln B)\cdot \vec \nabla f$ \\[4pt]
    Curvature &
    $\mathcal K$ &
    $\mathcal{K}(f):=\vec{\mathcal K} \cdot \vec \nabla f =
     \vec{\nabla}\cdot\left(\frac{\bhat\times\vec{\nabla} f}{B}\right)$,\\[4pt]
    Parallel derivative&
    $\nabla_\parallel $&
    $ \nabla_\parallel f := \bhat\cdot\vec{\nabla} f$ \\
    %Perturbed parallel Derivative&
    %$\bar\nabla_\parallel$ &
    %$\bar\nabla_\parallel f := (\bhat + \tilde{\vec b }_\perp)\cdot \nabla f = \nabla_\parallel f + A_\parallel \mathcal K_{\nabla\times\bhat}(f) + \frac{1}{B}[ f, A_\parallel]_\perp$ \\
    Parallel Laplacian&
    $\Delta_\parallel $&
    $\Delta_\parallel f:= \vec{\nabla} \cdot ( \bhat\bhat\cdot\vec{\nabla} f )$\\
\bottomrule
\end{longtable}
with $b^i$ the contra- and $b_i$ the co-variant components of $\bhat$, and
$\eps^{ijk}$ the Levi-Civita symbols.
Explicit expressions for the above expressions
depend on the choice of the magnetic field and the underlying coordinate system.
Note that we have
\begin{align}
    \vec \nabla \cdot \vec{\mathcal K_{\nabla\times\bhat}}
    &= -\vec\nabla \cdot \vec{\mathcal K_{\nabla B}} = -\vec{ \mathcal K_{\nabla\times\bhat}}\cdot\nabla\ln B, \\
    \vec\nabla\cdot\vec{ \mathcal K} &= 0, \\
    \vec{\mathcal K} &=
     \nabla\times\frac{\bhat}{B}\cdot\nabla f
    = \mathcal K_{\nabla\times\bhat}(f) + \mathcal K_{\nabla B}(f),\\
    \vec{ \mathcal K_{\nabla\times\bhat}} - \vec{ \mathcal K_{\nabla B}} &= \frac{1}{B^2} (\vec \nabla \times \vec B), \\
    \nabla_\parallel \ln B &= -\vec\nabla\cdot\bhat.
    \label{eq:curl_curvature}
\end{align}
The last equality holds if $\vec\nabla\cdot \vec B = 0$.
Note that in any arbitrary coordinate system we have
\begin{align}
(\vec \nabla f)^i = g^{ij}\partial_j f ~, \quad
\vec \nabla \cdot \vec v = \frac{1}{\sqrt{g}}\partial_i \left(\sqrt{g} v^i\right) ~, \quad
(\vec v \times \vec w)^i = \frac{1}{\sqrt{g}}\varepsilon^{ijk} v_jw_k ~.
%\label{}
\end{align}

%%%%%%%%%%%%%%%%%%%%%%%%%%%%%%%%%%%%%%%%%%%%%%%%%%
\subsection{Coordinate system}\label{sec:cylmetric}
We employ cylindrical coordinates \( (R,Z,\varphi) \), with \(\varphi\) anti directed to the geometric toroidal angle ({\bf clockwise} if viewed from above) to
obtain a right handed system. The parametric representation in Cartesian \((x,y,z)\) coordinates is therefore simply:
\begin{align}
 x &= R \hspace{1 mm} \sin{(\varphi)}, &
 y &= R \hspace{1 mm} \cos{(\varphi)}, &
 z &= Z .
\end{align}
Note here that the angle $\varphi = 0$ corresponds to the Cartesian $y$-axis.
The unit
basis vectors and (covariant) metric tensor are:
\begin{align}
 \ehat_R      &= (\sin{(\varphi)} ,   \cos{(\varphi)},0)^T, &
 \ehat_Z      &= ( 0 ,0 ,1 )^T, &
 \ehat_{\varphi} &= ( \cos{(\varphi)} , -\sin{(\varphi)} , 0 )^T,
\\
    (g_{ij}) &= \begin{pmatrix}
  1 & 0 & 0 \\
  0 & 1 & 0 \\
  0 & 0 & R^2
   \end{pmatrix}
% \vec{\nabla} R &= (\sin{(\varphi)} ,   \cos{(\varphi)},0 )^T , &
%  \vec{\nabla}Z &= ( 0 ,0 ,1 )^T,  &
%  \vec{\nabla}{\varphi} &= \frac{1}{R} ( \cos{(\varphi)} , -\sin{(\varphi)} , 0 )^T .
\end{align}
With the help of the metric elements we get a well behaved volume element \(\sqrt{g} = R\). However, we have a coordinate singularity at \(R=0\).
The cylindrical coordinate basis vectors are mutually orthogonal to each other.

\subsection{Solov'ev equilbrium}\label{sec:solovev}
In cylindrical coordinates the general axisymmetric  magnetic field can be written as (dimensionless)
\begin{align}
 \vec{B} &= \frac{R_0}{R}\left[I(\psi) \ehat_{\varphi} + \frac{\partial
 \psi_p}{\partial Z} \ehat_R -  \frac{\partial \psi_p}{\partial R} \ehat_Z\right] ,
\end{align}
which can obviously not be manipulated to be in Clebsch form.
Hence we are dealing with a non-flux aligned coordinate system.
We scaled $R$, $Z$ and $R_0$ with $\rho_s$, the magnetic field with $B_0$, the flux with $\psi_{p0} = B_0\rho_s \hat R_0$
and $I_0 = B_0 \hat R_0$ (with $\hat R_0$ being the dimensional major radius).
Note that this normalization is in line with the one later chosen for the gyrofluid
equations but is unnatural for the MHD type equilibrium equations through the introduction
of $\rho_s$.
Also note that with a typically convex function $\psi$, $I(\psi)>0$ and the previously defined coordinate system the field line winding is a {\bf left handed screw} in the positive $\vec B$-direction.
For the sake of clarity we define the poloidal magnetic field \( \vec{B}_p = \frac{R_0}{R}\left( \frac{\partial \psi}{\partial Z}\ehat_R - \frac{\partial \psi}{\partial R}\ehat_Z\right)
\) and the toroidal magnetic field \(\vec{B}_t =\frac{R_0I}{R} \ehat_{\varphi}\).
%The unit vectors are denoted by \(\ehat_{R}\), \(\ehat_{Z}\), \(\ehat_{\varphi}\).
Note that with $\psi_p = I(\psi_p) \equiv 1$ we recover a purely toroidal field.
With a subsequent identification of $x:=R-R_0$ and the transition to
Cartesian coordinates while straightening the field lines for large $R_0$
we recover the familiar slab geometry.

We have the equilibrium equations in toroidally symmetric, ideal MHD (
$\vec\nabla p = \vec j\times \vec B$ and $\vec \nabla\times\vec B = \vec j$ normalized with $p_0 = B_0^2/\mu_0$, and $j_0 = B_0/\rho_s/\mu_0$ )
\begin{align}
    \vec\nabla\times \vec B &= \frac{R_0}{R}\left[ -\Delta^*\psi_p\ehat_\varphi + I_Z \ehat_R - I_R\ehat_Z \right]\equiv \vec j\\
 j_\parallel &= \vec j\cdot \bhat = \frac{\d p}{\d\psi_p} \frac{I(\psi_p)}{B} +
 \frac{\d I}{\d\psi_p} B \quad \text{  Pfirsch-Schl\"uter \& Bootstrap current } \\
 \vec j_\perp &= \bhat\times\left(\vec j\times\bhat\right)=
 \frac{\bhat \times \nabla p}{B} \quad\quad\quad \text{ diamagnetic current} \\
 \vec j\times\vec B &= \frac{R_0^2}{R^2}\left[ -\Delta^* \psi_p - I
     \frac{\d I}{\d \psi_p} \right]\vec\nabla\psi_p \equiv \frac{\d p}{\d\psi_p}\vec\nabla\psi_p =\vec \nabla p
\end{align}
from where we recover the Grad-Shafranov equation
\begin{align}\label{eq:GSEdimless}
    -\Delta^*_\perp  \psi_p &= \frac{R^2}{R_0^2} \frac{d p}{d  \psi_p } + I \frac{d I}{d  \psi_p } \equiv \frac{R}{R_0} j_{\hat\varphi}
\end{align}
with $\Delta^*_\perp \psi_p = R\partial_R (R^{-1}\psi_R) + \psi_{ZZ}$.
The Solov'ev assumptions consist of \(A/R_0 = -I \frac{d I}{d  \psi_p }\) and \((1-A)/R_0 = -\frac{d p}{d  \psi_p }\), where \(A\) is a constant~\cite{Cerfon2010,Cerfon2014}.
By integration over \(\psi_p\) we find
\begin{align}\label{eq:solovevassumption}
 p(\psi_p) &= (A-1)\psi_p/R_0,  &
 I(\psi_p) &= \sqrt{-2 A \psi_p/R_0 + 1}, &
    j_{\hat\varphi} &= \left[(A-1)R^2/R_0^2 - A \right]/R.
\end{align}
Now, we introduce \(\bar{R} \equiv \frac{R}{R_0}\) and \(\bar{Z} \equiv\frac{Z}{R_0}\)
and solve Equations~\eqref{eq:GSEdimless} and~\eqref{eq:solovevassumption} to obtain
\begin{align}\label{eq:solovev}
 \psi_p (R,Z) &= R_0 \left[ A\left( \frac{1}{2} \bar{R}^2 \ln{\bar{R}}
   - \frac{1}{8}\bar{R}^4\right)+ B\frac{1}{8}\bar{R}^4 
   + \sum_{i=1}^{12} c_{i}  \bar{\psi}_{pi}\right],
\end{align}
with
\rowcolors{2}{gray!25}{white}
\begin{longtable}{>{\RaggedRight}p{7cm}>{\RaggedRight}p{7cm}}
\toprule
  $\bar{\psi}_{p1}=1$
  & $\bar{\psi}_{p7}=8\bar{Z}^6 -140 \bar{R}^2 \bar{Z}^4
                      + 75 \bar{R}^4 \bar{Z}^2 - 15\bar{R}^6\ln{\bar{R}}+ 180 \bar{R}^4 \bar{Z}^2 \ln{\bar{R}} \
                       -120 \bar{R}^2 \bar{Z}^4 \ln{\bar{R}}$\\
%
  $\bar{\psi}_{p2}=\bar{R}^2$ &
  $\bar{\psi}_{p8}=\bar{Z}$ \\
%
  $\bar{\psi}_{p3}=\bar{Z}^2 - \bar{R}^2 \ln{\bar{R}}$ &
  $\bar{\psi}_{p9}=\bar{Z}  \bar{R}^2$\\
%
  $\bar{\psi}_{p4}=\bar{R}^4 -4\bar{R}^2\bar{Z}^2$ &
  $\bar{\psi}_{p10}=\bar{Z}^3 - 3 \bar{Z} \bar{R}^2 \ln{\bar{R}}$\\
  %
  $\bar{\psi}_{p5}=2\bar{Z}^4 - 9 \bar{R}^2\bar{Z}^2 + \
                     3 \bar{R}^4 \ln{\bar{R}} \
                    -12  \bar{R}^2\bar{Z}^2 \ln{\bar{R}}$
  &
$\bar{\psi}_{p11}=3 \bar{Z}\bar{R}^4 - 4\bar{Z}^3\bar{R}^2$\\
%
  $\bar{\psi}_{p6}=\bar{R}^6 -12 \bar{R}^4 \bar{Z}^2
                     + 8  \bar{R}^2 \bar{Z}^4$ &
  $\bar{\psi}_{p12}= 8 \bar{Z}^5 -45 \bar{Z} \bar{R}^4 - \
                       80 \bar{Z}^3 \bar{R}^2\ln{\bar{R}} \
                       +60 \bar{Z} \bar{R}^4\ln{\bar{R}}$ \\
   & \\
\bottomrule
\end{longtable}
The choice of the coefficients \(c_{i}\) and \(A\) determines the actual form of the magnetic field, while $R_0$ appears as an artificial scaling factor (note here that a change in $\rho_s$ changes $R_0$ but not the form or size of the dimensional equilibrium magnetic field).
Note that $B$ is not part of the original work ($B=1$ per default) and has been added for additional flexibility in fitting the coefficients to an existing experimental equilibrium.
Eq.~\eqref{eq:solovev} allows axisymmetric equilibria with e.g. single and asymmetric double X-point configurations, force-free states,
field reversed configurations and low and high beta tokamak equilbria. This casts this simple analytical equilibrium to the ideal choice in order to study geometric effects (e.g. inverse aspect ratio, elongation and triangularity) in magnetised plasmas.

Note that
\begin{align}
    B^R&=B_R = R_0\psi_Z/R \\
    B^Z&=B_Z = - R_0\psi_R/R \\
    B^\varphi &= B_\varphi/R^2 = R_0I/R^2
\end{align}
(contra- and covariant components of $\vec B$).
By construction we have $\partial_\varphi B = 0$ with
\begin{align}
  B = \frac{R_0}{R}\sqrt{ {I^2 + |\nabla \psi_p|^2}}.
    \label{}
\end{align}
Furthermore, we have
\begin{align}
  \nabla_\parallel f(R,Z) = \frac{R_0}{RB}[f,\psi_p]_{RZ}\Rightarrow \nabla_\parallel \ln B = \frac{R_0}{RB^2}\left[B, \psi_p\right]_{RZ} = -\vec\nabla\cdot\bhat.
\end{align}
We allow various simplifications to the curvature operator
for the Solov'ev equilibrium.

%%%%%%%%%%%%%%%%%%%%%%%%%

\subsubsection{Toroidal field line approximation}\label{sec:torfieldlineapprox}
The toroidal field line approximation applies \(\bhat\approx \ehat_\varphi\) to all perpendicular operators
(e.g.: Poisson bracket, perpendicular elliptic operator and curvature operators)
but retains the full expression for the magnetic field unit vector \(\bhat\)
for parallel operators (\(\nabla_\parallel\) and \(\Delta_\parallel\)).
In cylindrical coordinates that is
\begin{align}
[f,g]_\perp \equiv [f,g]_{RZ} &= \frac{1}{R} \left(\partial_R f\partial_Z g - \partial_Z f\partial_R g\right) \\
\nabla_\perp f &= \partial_R f \ehat_R + \partial_Z f \ehat_Z \\
\Delta_\perp f &= \frac{1}{R}\partial_R \left( R \partial_R f\right) + \partial_Z(\partial_Z f)
\label{}
\end{align}
The curl of $\bhat$ reduces to
%\begin{align}
 $\nabla\times\bhat \approx -  \frac{1}{R} \ehat_Z$.
%end{align}
This simplifies the curvature operators to:
\begin{align}
\vec{\mathcal{K}}_{{\nabla\times\bhat}}  &\approx  -  \frac{1}{B R} \ehat_Z , &
\vec{ \mathcal{K} }_{\vec{\nabla}  B}  &\approx  -\frac{1}{B^2}\frac{\partial B}{\partial Z}\ehat_R +\frac{1}{B^2} \frac{\partial B}{\partial R}\ehat_Z &
%\ehat_\varphi \times \vec{\nabla} B, &
\vec{ \mathcal{K} } &\approx \vec{ \mathcal{K} }_{\vec{\nabla}  B}  +\vec{ \mathcal{K} }_{{\nabla\times\bhat}} ,
%\\
%\mathcal{K}_{{\nabla\times\bhat}}(f)   &\approx  -  \frac{1}{B R} \frac{\partial f}{\partial Z},&
%\mathcal{K}_{\vec{\nabla}  B} (f)  &= \frac{1}{B} \left[\ln B, f \right]_{RZ},&
%\mathcal{K} (f) &\approx\frac{1}{B} \left[\ln B, f \right]_{RZ}-  \frac{1}{B R} \frac{\partial f}{\partial Z} ,
\end{align}
and
\begin{align}
 \vec{\nabla} \cdot \vec{\mathcal{K}}_{{\nabla\times\bhat}} &\approx \frac{1}{R B^2} \frac{\partial B}{\partial Z}.
\end{align}
which, results in a vanishing divergence of the curvature operators \( \vec{\nabla} \cdot \vec{ \mathcal{K} } = 0\).

Note that in an actual toroidal field we have
\begin{align}
  \vec B(R) := \frac{R_0}{R} \ehat_\varphi
  \label{}
\end{align}
We then have $\bhat = \ehat_\varphi$ and the curvature operators further
simplify to
\begin{align}
  \vec{ \mathcal K_{\nabla\times\bhat}} = \vec{ \mathcal K_{\nabla B}} = -\frac{1}{R_0} \ehat_Z =
\vec{ \mathcal K}/2\\
  \nabla\cdot\vec{\mathcal K_{{\nabla\times\bhat}}}=
    \nabla_\parallel \ln B = 0
    \label{}
\end{align}

\subsubsection{Low beta approximation}\label{sec:lowbetaapprox}
In this approximation we apply the toroidal field line approximation
as in Section
\ref{sec:torfieldlineapprox}
but approximate the curvature operator $\mathcal K_{\nabla\times\bhat} \approx \bhat\times\vec \kappa$
  with
  $\vec \kappa := \bhat \cdot \vec \nabla\bhat = -\bhat \times(\vec \nabla\times \bhat)$.
For an isotropic pressure plasma \(\vec{P} = \vec{I} P_\perp + \vec{b} \vec{b} P_\Delta \approx \vec{I} P_\perp\) and with the definition of the plasma beta parameter
\(\beta = \frac{P}{B^2/(2 \mu_0) } \)
we can rewrite the curvature to
\begin{align}
    \vec{\kappa} &\approx \frac{\beta}{2} \vec{\nabla} \ln(P) +\vec{\nabla}_\perp \ln{B} .
\end{align}
In low beta plasmas \(\beta\ll1\) the curvature reduces to:
\begin{align}
    \vec{\kappa} & \approx \vec{\nabla}_\perp \ln{B} .
\end{align}
This simplifies the curvature operators to:
\begin{align}
\vec{\mathcal{K}_{{\nabla\times\bhat}}}(f) \approx
\vec{ \mathcal{K} }_{\vec{\nabla}  B}  &\approx  -\frac{1}{B^2}\frac{\partial B}{\partial Z}\ehat_R +\frac{1}{B^2} \frac{\partial B}{\partial R}\ehat_Z &
\mathcal{K} (f) &\approx 2\mathcal{K}_{\vec{\nabla}  B} (f) , &
    \vec{{\nabla\times\bhat}} \cdot \vec{\mathcal{K}}_{\vec{\nabla}  B} &= 0.
\end{align}
The divergence over the curvature vanishes \( \vec{\nabla} \cdot \vec{ \mathcal{K} } = 0\) only if \( \vec{\nabla} \cdot \vec{ \mathcal{K}}_{\vec{\nabla}  B}   = 0\).
In general, the divergence \( \vec{\nabla} \cdot \vec{ \mathcal{K} } \approx 0\) is only approximately vanishing.
\subsubsection{True perpendicular terms}

Without any approximations we have
\begin{align}
b^R = {\frac{\partial \psi}{\partial Z}}\left(I^2+|\nabla\psi|^2\right)^{-1/2} \quad
b^Z = -{\frac{\partial \psi}{\partial R}}\left(I^2+|\nabla\psi|^2\right)^{-1/2} \quad 
b^\varphi = \frac{I}{R}\left(I^2+|\nabla\psi|^2\right)^{-1/2} \\
\vec\nabla\cdot\bhat = -\nabla_\parallel \ln B = -\frac{R_0}{R B^2}[B,\psi_p]_{RZ} \\
\left({\nabla\times\bhat}\right) \cdot\bhat =
    (I'(\nabla\psi_p)^2 - I \Delta_\perp^* \psi_p)\frac{ R_0^2}{R^2B^2} \propto 1/R_0
\label{}
\end{align}
where for the last
estimate we inserted the Grad-Shafranov equation and the Solov'ev assumptions.
We can then insert $\bhat$ into the exact definitions for $[.,.]_\perp$, $\nabla_\perp$ and $\Delta_\perp$ from Section~\ref{sec:magnetic}.

For the curvature terms we can explicitly write
\begin{align}
K_{\nabla B}^R &= -\frac{R_0 I}{B^3R}\frac{\partial B}{\partial Z} \equiv -\frac{1}{B^2}\frac{\partial B}{\partial Z}b^\varphi \\
K_{\nabla B}^Z &= \frac{R_0 I}{B^3R}\frac{\partial B}{\partial R}\equiv \frac{1}{B^2}\frac{\partial B}{\partial R}b^\varphi \\
K_{\nabla B}^\varphi &= \frac{R_0}{B^3R^2}\left(
      \frac{\partial \psi}{\partial Z} \frac{\partial B}{\partial Z}
    + \frac{\partial \psi}{\partial R}\frac{\partial B}{\partial R}\right)
%\equiv \frac{1}{B^2R}\left(\bhat^R \frac{\partial B}{\partial Z} - \bhat^Z \frac{\partial B}{\partial R}\right)\quad %contravariant phi component
\label{}
\end{align}
and
\begin{align}
K_{\nabla\times\bhat}^R &= \frac{R_0 }{RB^3}\left( B\frac{\partial I}{\partial Z} -I\frac{\partial B}{\partial Z}\right) \\
K_{\nabla\times\bhat}^Z &= \frac{R_0 }{RB^3} \left( I\frac{\partial B}{\partial R} - B\frac{\partial I}{\partial R} \right)\\
K_{\nabla\times\bhat}^\varphi &= \frac{R_0}{R^2B^2}\left(
+ \frac{1}{B}\frac{\partial\psi}{\partial Z} \frac{\partial B}{\partial Z}
+ \frac{1}{B}\frac{\partial \psi}{\partial R}\frac{\partial B}{\partial R}
-R\frac{\partial}{\partial R}\left(\frac{1}{R}\frac{\partial\psi}{\partial R}\right) 
- \frac{\partial^2 \psi}{\partial Z^2}
\right) \\
\vec\nabla\cdot\vec{\mathcal K_{\nabla\times\bhat}} &= -\vec\nabla\cdot\vec{\mathcal K_{\nabla B}}=
    -\vec{\mathcal K_{\nabla\times\bhat}}\cdot \vec \nabla\ln B = \frac{R_0}{RB^3}[I,B]_{RZ}
%contravariant phi component
\label{}
\end{align}

\subsection{ Modified $\psi_p$}
Our computational domain is a box and in particular not aligned with the
magnetic flux surfaces. This means that particularly in the corners of
the domain the field lines inside the domain are very short (in the
sense that the distance between the entry point and leave point is short).
It turns out that this behaviour is numerically disadvantageous (may
blow up the simulation in the worst case) in the
computation of parallel derivatives. In order to remedy this situation
we propose to modify the flux surfaces $\psi_p$ to a constant value
if $\psi_p$ exceeds a certain critical value. In this way the poloidal
field component vanishes in the corners of the domain at the cost
of introducing a strong shear layer limiting the scrape-off layer width.

We define an approximation to the step function with width $\alpha$
\begin{align}
\Theta_\alpha(\psi) := \begin{cases}
    0 & \text{ for } \psi < - \alpha  \\
    \frac{1}{32 \alpha^7}  \left(16 \alpha^3-29 \alpha^2 \psi+20 \alpha \psi^2-5 \psi^3\right) (\alpha+\psi)^4
    &\text{ for } -\alpha<\psi<+\alpha \\
    1 & \text{ for } \psi > \alpha 
\end{cases}
    \approx H(\psi)
\label{eq:approx_heaviside}
\end{align}
if $H(\psi)$ is the Heaviside step function.
An integral of this function is
\begin{align}
\theta_\alpha(\psi) := \begin{cases}
    0 &\text{ for } \psi < -\alpha \\
    \frac{1}{256 \alpha^7} \left(35 \alpha^3-47 \alpha^2 \psi+25 \alpha \psi^2-5 \psi^3\right) (\alpha+\psi)^5
     &\text{ for } -\alpha<\psi<+\alpha \\
\psi &\text{ for } \psi > \alpha
\end{cases}
    \approx \psi H(\psi)
\end{align}

We now use 
\begin{align}
-\theta_\alpha(\psi_0 - \psi)+\psi_0 \approx (\psi- \psi_0)H(\psi_0-\psi) + \psi_0
\label{eq:modified_psip}
\end{align}
instead of $\psi$ for the computation of the
magnetic field, which introduces a shear layer around $\psi_0$ where the
fieldlines are straightened to match $\ehat_\varphi$.
Note that $\Theta_\alpha(0) = 0.5$ and $\theta_\alpha(0) = 35\alpha/256$.
\section{Flux surface averaging and safety factor}
\subsection{Preliminary}
Recall that the {\bf Dirac delta-function} has the property (in any dimension):
\begin{align} \label{eq:dirac_delta}
\int_V f(\vec x) \delta(h(\vec x) - h') \dV = \int_{h=h'} \frac{f(\vec x)}{|\nabla h|} \dA
\end{align}
which means that the delta-function can be used to express area integrals of the
submanifold given as a contour of the function $h(\vec x)$.
A numerically tractable approximation to the delta-function reads
\begin{align}\label{eq:delta}
\delta(h(\vec x)-h') = \frac{1}{2\pi \epsilon^2}
\exp\left( - \frac{\left(h(\vec x)-h'\right)^2}{2\epsilon^2}\right)
\end{align}
where $\epsilon$ is a small, free parameter.
In the DG framework the left-hand side
of Eq.~\eqref{eq:dirac_delta} can thus readily be computed
via Gauss-Legendre quadrature, which we propse as a first method to compute area
integrals even if our coordinate system is not aligned to the area.
Note: in order for this to work the Delta function needs to be numerically
resolved and cannot be made arbitrarily small.
This introduces a smoothing effect
over neighboring contour lines which is given by the grid distance.

Furthermore, recall the {\bf co-area formula}
\begin{align} \label{eq:coarea}
\int_{\Omega_0} f(\vec x) \dV =
\int_0^{h_0} \left( \int_{h=h'} \frac{f(\vec x)}{|\nabla h|}  \dA  \right) \d h'
\end{align}
where $\Omega_0$ is the volume enclosed by the contour $h=h_0$.
The co-area formula can be viewed as a change of variables in the
volume integral.

We define the {\bf toroidal average} of a function $f(R,Z,\varphi)$ as
\begin{align} \label{eq:phi_average}
\langle f\rangle_\varphi(R,Z) := \frac{1}{2\pi}\oint f(R,Z,\varphi)\d \varphi
\end{align}

In arbitrary coordinates the area integral is defined by the pull back
of the flux 2-form and the metric
\begin{align}
\label{}
\dA^2 = i_{\hat \psi_p} vol^3 \quad \hat \psi_p = \frac{\nabla \psi_p}{|\nabla \psi_p|}
\end{align}
to a parameterization of the flux-surface.
In a flux-aligned coordinate system $\{\zeta, \eta, \varphi\}$ the pull-back is trivial ($\zeta=const$) and we have
\begin{align}
\dA &= \sqrt{g^{\zeta\zeta}} \sqrt{g} \d\eta\d\varphi = f_0|\nabla\psi_p|\sqrt{g}\d\eta\d\varphi,
\\
\vec\dA &:= \hat\psi_p \dA = f_0 (\nabla\psi_p) \sqrt{g}\d\eta\d\varphi,\quad
\label{}
\end{align}
where we used that $g^{\zeta\zeta} = (\nabla\zeta)^2 = f_0^2(\nabla\psi_p)^2$.
Notice that numerically we can integrate in flux-aligned coordinates by generating a corresponding
grid and pulling back (interpolating) the relevant fields to this grid. This is the second method
to numerically compute area integrals.

\subsection{Flux surface average}

There is two possible ways to introduce a flux-surface average.
The first one is the straightforward average on the actual area of the
flux-surface.
The {\bf area average}
of a function $f(R,Z,\varphi)$ is given by the formula
\begin{align}\label{eq:fsa_area}
\langle f \rangle^{area}_{\psi_{p}} :=&
\frac{ \oint_{\psi_p  } f(R,Z,\varphi)\dA}{\oint_{\psi_p } \dA} \nonumber \\
=& \frac{\int_\Omega \langle f\rangle_\varphi(R,Z) |\vec\nabla\psi_p| \delta(\psi_p(R,Z)-\psi_{p})H(Z-Z_X)\ R \d R \d Z}
{\int_\Omega |\vec\nabla\psi_p|\delta(\psi_p(R,Z)-\psi_{p})H(Z-Z_X)\ R \d R \d Z} \nonumber\\
=& \frac{\oint \langle f\rangle_\varphi(\zeta,\eta) \sqrt{g g^{\zeta\zeta}}\d\eta}
         { \oint \sqrt{g g^{\zeta\zeta}}\d\eta}
\end{align}
%with $\dV := R\d R\d Z\d \varphi$ %(we define the average in computational space and omit one $R$)
and we use the Heaviside function $H(Z-Z_X)$ to cut away contributions from below the X-point
in our domain $\Omega$.

The second one (the {\bf volume average} after \cite{haeseleer}) defines an average on a
small volume - a shell centered around the flux-surface - defined by two neighboring flux-surfaces.
With the help of the volume
flux label (notice that both the volume $v$ as well as the poloidal flux $\psi_p$ have physical 
meaning while the coordinate $\zeta(\psi_p)$ is an arbitrary choice) we define
\begin{align} \label{eq:fsa_vol}
v(\psi_p) :=& \int_{\psi_{p,\min}}^\psi \dV = \int^{\zeta(\psi_p)} \sqrt{g}\d\zeta\d\eta\d\varphi,
\\
\frac{\d v}{\d\psi_p} =& \int\dA |\nabla\psi_p|^{-1} = 2\pi f_0\oint_{\zeta(\psi_p)} \sqrt{g}\d\eta \\
\langle f \rangle^{vol}_\psi :=& \frac{\partial}{\partial v} \int \dV f
 = \frac{1}{\int \dA |\nabla\psi_p|^{-1} } \int_{\psi_p} \frac{f(\vec x)}{|\nabla\psi_p|} \dA \nonumber\\
=& \frac{\int_\Omega \langle f\rangle_\varphi(R,Z) \delta(\psi_p(R,Z)-\psi_{p})H(Z-Z_X)\ R \d R \d Z}
{\int_\Omega \delta(\psi_p(R,Z)-\psi_{p})H(Z-Z_X)\ R \d R \d Z}\nonumber\\
 =& \left(\frac{\d v}{\d\psi_p }\right)^{-1} 2\pi f_0 \oint_0^{2\pi} \langle f\rangle_\varphi(\zeta,\eta) \sqrt{g}\d\eta
 = \frac{1}{\oint \sqrt{g}\d\eta } \oint_0^{2\pi} \langle f\rangle_\varphi(\zeta,\eta) \sqrt{g}\d\eta
\end{align}
where we used the co-area formula Eq.~\eqref{eq:coarea} for the second
identity. We immediately see that this definition differs from the first
Eq.~\eqref{eq:fsa_area} by the weight factor $|\nabla\psi_p|$ and that it is particularly easy to compute
in a flux-aligned coordinate system. Notice however that the volume element does appear (unlike e.g. Tokam3X papers)

Both averages fulfill the basic identities
\begin{align}
\label{eq:fsa_identities}
\langle \mu f + \lambda g\rangle &= \mu\langle f\rangle + \lambda \langle g\rangle \\
\langle f(\psi_p) \rangle &= f(\psi_p)
\end{align}


The volume average is better suited for density-like quantities
than the area average as we can see with the following identity.
Assume we have a quantity $X$ with $\partial_t X + \nabla \cdot \vec j_X = \Lambda_X$. Then we can use the volume average to write
\begin{align}
\frac{\partial}{\partial t} \langle X \rangle^{vol} + \frac{\partial}{
  \partial v} \langle \vec j_X\cdot \nabla v\rangle^{vol}  = \langle \Lambda_X\rangle^{vol}
\label{eq:fsa_balance}
\end{align}
where again $v=v(\psi_p)$ is the volume flux label.
The {\bf total flux} of a given flux density $\vec j_X$ though the
flux surface $\psi_p = \psi_{p0}$ is given by
\begin{align}
\left\langle\vec j_X\cdot\nabla v\right\rangle^{vol} &:= J_X=\oint_{\psi_p=\psi_{p0}} \vec j_X\cdot \vec{\dA} =
 \frac{\d v}{\d\psi_p} \langle \vec j_X\cdot\nabla\psi_p \rangle^{vol}\\
 &=
   2\pi f_0 \oint_0^{2\pi} \langle \vec j_X\cdot\nabla\psi_p\rangle_\varphi(\zeta,\eta) \sqrt{g}\d\eta
%2\pi\int_\Omega \vec \langle \vec j\cdot \vec\nabla\psi_p\rangle_\varphi \delta(\psi_p(R,Z)-\psi_{p0}) H(Z-Z_X)\ R \d R \d Z
\label{eq:total_flux}
\end{align}
Once we have the flux-surface averaged equation we can easily get the volume integrated version (again with the help of the co-area formula)
\begin{align}
\frac{\partial}{\partial t} \int_0^{v(\psi_p)}\langle X \rangle^{vol} \d v 
+ \langle \vec j_X\cdot \nabla v\rangle^{vol}(v(\psi_p))  = \int_0^{v(\psi_p)}\langle \Lambda_X\rangle^{vol}\d v
\label{eq:integral_balance}
\end{align}

\subsection{The safety factor}
Assume that we pick a random field line and follow it (integrate it) for exactly one
poloidal turn. The {\bf safety factor} is defined as the ratio between
the resulting toroidal angle ($\Delta\varphi$) to the poloidal angle ($2\pi$)
\begin{align}
q := \frac{\Delta\varphi}{2\pi}
\label{}
\end{align}
Since our magnetic field is symmetric in $\varphi$ and we used one
full poloidal turn this definition is independent of which
fieldline we pick on a given flux surface.

%Let us define the poloidal length $s$ as the fieldline following
%parameter i.e. $\vec B\cdot \nabla s \equiv B_p = R_0|\nabla \psi_p|/R$
%and $\d\varphi/\d s = B^\varphi(R(s), Z(s)) / B_p(R(s),Z(s))$.
%We can then express the safety factor as the line integral
%\begin{align}
%q=\frac{1}{2\pi}\oint \frac{B^\varphi}{B_p} \d s = \frac{1}{2\pi}\oint_{\psi_p=\psi_{p0}}\frac{I(\psi_p)}{R|\vec\nabla\psi_p|} \d s
%= \frac{1}{2\pi}\int \frac{I(\psi_p)}{R}\delta(\psi_p-\psi_{p0}) H(Z-Z_X) \d R\d Z
%\end{align}
%where we made use of Eq.~\eqref{eq:dirac_delta} in two dimensions in the
%last equality and thus arrive at a numerical tractable expression
%to evaluate the safety factor.
Let us define the geometric poloidal angle $\Theta$ as the fieldline following
parameter i.e. $\vec B\cdot\nabla\Theta = R_0(\psi_R (R-R_0) + \psi_Z Z)/r^2R$.
We can then directly integrate the safety factor as
\begin{align}\label{eq:safety_factor}
\frac{\d R}{\d\Theta} = \frac{B^R}{B^\Theta}\quad 
\frac{\d Z}{\d\Theta} = \frac{B^Z}{B^\Theta}\quad 
\frac{\d \varphi}{\d\Theta} = \frac{B^\varphi}{B^\Theta}\\
q\equiv\frac{1}{2\pi}\oint \frac{B^\varphi}{B^\Theta} \d\Theta
\end{align}
We integrate this equation with the help of one of our grid
construction algorithms, i.e. we use a high-order Runge-Kutta method
and refine the stepsize until machine-precision is reached.

Notice that the safety factor diverges on the last closed flux
surface whereas the Eq.~\eqref{eq:total_flux} and \eqref{eq:fsa_area}
remain finite due to the $\nabla\psi$ factor.

%%%%%%%%%%%%%%%%%%%%%%%%%%%%%%%%%%%%%%%%%%%%%%%%%%%%%%%%%%%%%%%%%%%%%%%%%%%%%%%%%
\section{The model} \label{sec:model}
\subsection{Conservative form}
We scale all spatial lengths by $\rho_s = \sqrt{T_e m_i}/(eB_0)$ and time by the ion gyro-frequency $\Omega_0 = eB_0/m_i$.
The magnetic field is scaled with $B_0$, densities with $n_0$ and the parallel velocity is scaled with $c_s = \sqrt{T_e/m_i}$.
The potential is scaled with $\hat \phi = e/T_e$ and the vector potential with
$\hat A_\parallel = \rho_s B_0$.
We introduce the dimensionless parameters
\begin{align}
  \tau_a = \frac{T_a}{z_aT_e}~,\quad \mu_a = \frac{m_a}{z_am_i}\text{ and } 
  \beta:=\frac{\mu_0 n_0 T_e}{B_0^2}
  \label{}
\end{align}
where $a\in\{e,i\}$ is the species label and $z$ is the charge number. We define with 
$\eta_\parallel := \frac{0.51 m_e \nu_{ei}}{n_e e^2}$
\begin{align}
  \eta:=\frac{en_0\eta_\parallel}{B_0} = 8.45\cdot 10^{-5}\ln \lambda \left(\frac{n_0}{10^{19}\text{m}^3}\right) \left(\frac{T_e}{\text{eV}}\right)^{-3/2} \left(\frac{B_0}{\text{T}}\right)^{-1},
    \label{eq:resistivity}
\end{align}
with $\ln \lambda \approx 10$.
 The approximate Spitzer current \(J_{\parallel,s}:= n_e \left(U_i - u_e\right)\)
 determines the parallel resistive terms to $R_\parallel:= n_e\eta J_{\parallel,s}$.
Omitting the species label we arrive at (dividing the density equation by $\Omega_0n_0$ and the velocity equation by $\Omega_0 c_s$)
\begin{align}
\frac{\partial}{\partial t} N &+ \vec\nabla\cdot\left( N \left(
    \vec v_E + \vec v_K + \vec v_{C} + U\left(\bhat + \tilde{\vec b}_\perp\right)\right)\right) = \Lambda_N + S_N \\
\mu N \frac{\partial}{\partial t} U &+ \mu N \left(
    \vec v_E + \vec v_K + \vec v_{C} + U\left(\bhat + \tilde{\vec b}_\perp\right)
    \right)\cdot \vec\nabla U  \nonumber \\
    &+ 2\mu \vec \nabla \cdot ( NU \vec v_{\nabla\times\bhat})
    -\mu NU\vec \nabla\cdot \vec v_{\nabla\times\bhat}
    + \mu NU\mathcal K_{\nabla\times\bhat}(\psi) \nonumber\\
    &= -\tau \left(\bhat + \tilde{\vec b}_\perp\right)\cdot \nabla N 
    -N \left( \left(\bhat+\tilde{\vec b}_\perp\right)\cdot \nabla \psi + \frac{\partial A_\parallel}{\partial t}\right) 
    - \eta n_e^2(U_i-u_e) + \mu N(\Lambda_U + S_U)
\label{}
\end{align}
with
\begin{align}
\vec v_E := \frac{\bhat\times\nabla\psi}{B},\quad
\vec v_{K} := \tau \left(\vec{\mathcal K_{\nabla B}} + \vec{\mathcal K_{\nabla\times\bhat}}\right)=\tau\vec{\mathcal K}  ,\nonumber\\
\vec v_C := \mu U^2\vec{\mathcal K_{\nabla\times\bhat}},\quad
\vec v_{\nabla\times\bhat} := \tau\vec{\mathcal K_{\nabla\times\bhat}},\quad
\tilde{\vec b}_\perp = \frac{\nabla\times A_\parallel \bhat}{B}.
\label{}
\end{align}

The electric potential \(\phi\) and parallel magnetic vector potential \(A_\parallel\) are
computed by the polarisation and induction equations (with $q_e=-e$ and $q_i=+e$)
\begin{align}
 -\vec{\nabla} \cdot\left(\frac{\mu_iN_i}{B^2} \vec{\nabla}_\perp \phi\right) &=  \Gamma_{1,i} N_i -n_e, \quad \Gamma_{1,i}^{-1} := 1-\frac{1}{2}\mu_i\tau_i\Delta_\perp , \\
  -\frac{1}{\beta} \Delta_\perp A_\parallel &= \left(N_i U_i-n_e u_e \right)
  \label{eq:polarisation_dimensional}
\end{align}
Given $\phi$ we define the generalised electric potential
\begin{align}
    \psi_e := \phi,\quad \psi_i&:= \Gamma_{1,i} \phi - \frac{\mu_i }{2}\left(\frac{\vec \nabla_\perp\phi}{B}\right)^2
\end{align}
In total 
we have an isothermal 3d gyro-fluid model with up to 2nd order FLR effects
on in the electric potential $\phi$ and 0th order FLR effects in the parallel magnetic
potential $A_\parallel$.
We have the continuity equation for the electron density \(n_e\) and the ion gyro-centre
density \(N_i\) and the momentum conservation equation for
the parallel electron velocity \(u_e\) and the parallel ion gyro-centre velocity \(U_i\)~\cite{WiesenbergerPhD, HeldPhD}.

\subsection{Diffusive terms}\label{sec:dissres}
The dissipative terms can be decomposed into perpendicular and parallel components
\begin{align}
 \Lambda_{n_e} &= \Lambda_{n_e,\perp}+\Lambda_{n_e,\parallel}, &
 \Lambda_{N_i} &= \Lambda_{N_i,\perp}+\Lambda_{N_i,\parallel},\\
 \Lambda_{u_e} &= \Lambda_{u_e,\perp}+\Lambda_{u_e,\parallel},&
 \Lambda_{U_i} &= \Lambda_{U_i,\perp}+\Lambda_{U_i,\parallel}.
\end{align}
For numerical stabilisation we choose:
\begin{align}
\Lambda_{n_e,\parallel} &= \nu_\parallel \Delta_\parallel n_e &
\Lambda_{N_i,\parallel} &= \nu_\parallel \Delta_\parallel N_i \\
\Lambda_{u_e,\parallel} &= \nu_\parallel \Delta_\parallel u_e &
\Lambda_{U_i,\parallel} &= \nu_\parallel \Delta_\parallel U_i 
\end{align}
Similarly, for the perpendicular dissipation we apply viscous or hyperviscous terms.
\begin{align}\label{eq:perpdiffNT}
 \Lambda_{n_e,\perp} &=  \nu_\perp \Delta_\perp n_e \text{ or } -\nu_\perp \Delta_\perp^2 n_e&
 \Lambda_{N_i,\perp} &=  \nu_\perp \Delta_\perp N_i \text{ or } -\nu_\perp \Delta_\perp^2 N_i & \\
 \Lambda_{u_e,\perp} &=  \nu_\perp \Delta_\perp u_e \text{ or } -\nu_\perp \Delta_\perp^2 u_e &
 \Lambda_{U_i,\perp} &=  \nu_\perp \Delta_\perp U_i \text{ or } -\nu_\perp \Delta_\perp^2 U_i
\end{align}
Here the mass diffusion coefficient coincides with the viscous coefficient, hence we fixed the Schmidt number \(\mathit{Sc}_\parallel:= \frac{\nu_U}{\nu_N}\) to unity.

\subsection{Boundary and initial conditions}
We define the simulation box as
$[ R_{\min}, R_{\max}]\times [Z_{\min}, Z_{\max}] \times [0,2\pi]$,
where we define
\begin{align} \label{eq:box}
    R_{\min}&=R_0-\varepsilon_{R-}a\quad
    &&R_{\max}=R_0+\varepsilon_{R+}a\nonumber\\
    Z_{\min}&=-\varepsilon_{Z-}ae\quad
    &&Z_{\max}=\varepsilon_{Z+}ae
\end{align}
where $a$ is the minor radius, $e$ is the elongation of the flux surfaces and
the $\varepsilon$ are free parameters to be specified by the user.

We choose boundary conditions separately on input for the variables
$n_e$, $u_e$ and $\phi$. The boundary condition for $N_i$, $U_i$ and
$\psi$ are equal to $n_e$, $u_e$ and $\phi$ respectively.
We choose $A_\parallel$ to have equal boundary conditions as $u_e$ and $U_i$.
This will later enable us to treat the sum of $U$ and $A_\parallel$
in the same way as $U$.
Typically,
\begin{align}
n_e = n_0, \quad u_e = \phi = 0
\text{ or } \hat n \cdot \nabla n_e = \hat n \cdot \nabla u_e = 0
\end{align}
where $\hat n$ is the normal vector to the boundary.

We initialize the parallel velocity to zero
\begin{align}
  u_e(R,Z,\varphi,0) = U_i(R,Z,\varphi,0) = 0
  \label{}
\end{align}
which in turn initializes $A_\parallel = 0$
and initialize the electron density with
\begin{align} \label{eq:initial_ne}
    n_e(R,Z,\varphi, 0)= n_{prof}(R,Z) + \tilde n(R,Z,\varphi)
\end{align}
consisting of a toroidally symmetric background profile $n_{\text{prof}}(R,Z)$ and a perturbation
$\tilde n(R,Z,\varphi)$, which breaks the toroidal symmetry.
Note that we should take care to intitialize a smooth profile with ideally well-defined $\Delta^2_\perp n_e$.
%Let us define another approximation to the Heaviside function
%\begin{align}
%  \Theta(x) := \frac{1}{2}\left( 1 + \tanh\left( \frac{x-3\alpha}{ \alpha} \right) \right) \quad \Theta(x) \approx H(x)
%  \label{eq:heaviside_profile}
%\end{align}
%where $H(x)$ is the actual Heaviside function and
%$\alpha$ is a (small) width parameter.

Let us define a flux-aligned density profile as
\begin{align} \label{eq:density_profile}
  n_{\text{prof}}(R,Z)=
      n_0 + \triangle n_{peak}\frac{\psi_p(R,Z) }{\psi_p(R_0,0)}\Theta_{\alpha}(-\psi_p(R, Z)-\alpha) H(Z-Z_X)
\end{align}
The second Heaviside is multiplied only if the equilibrium $\psi_p$ has an
X-point and avoids a profile in the private flux region. The factor $\alpha$ provides a smooth transition
zone that avoids numerical oscillations.


We have two possibilities to initialize the ion density
\begin{align} \label{eq:initphi}
  N_i = \Gamma_{1,i}^{-1} n_e \text{ or } N_i = \Gamma_{1,i}n_e\approx \left(1+\frac{1}{2}\tau_i\mu_i\Delta_\perp\right)n_e
\end{align}
In the first case the potential $\phi= 0$ while in the second case
the $E\times B$ and ion diamagnetic vorticity coincide $\Delta_\perp N_i \propto \Delta_\perp \phi$ in the long-wavelength limit.
Note that $\alpha$ must not be too small to avoid $N_i < 0$.
We can choose between several initial conditions for $\tilde n$:

\subsubsection{Blob and Straight blob}
We initialize a blob in the R-Z plane
\begin{align} \label{eq:initial_blob}
  \tilde n_{blob}(R,Z,0) = \triangle n \exp\left( -\frac{(R - R_0 - p_x a)^2 + (Z-p_ya)^2}{\sigma^2} \right)
\end{align}
Then, we use fieldline integration modulated by 
\begin{align}
  m_{blob}(s) = \exp\left( -\frac{s^2 }{\pi^2\sigma_z^2} \right)
\end{align}
to transform this blob to all other poloidal
planes.
We either follow fieldlines around the torus several times (``blob'') or only once
(``straight blob'').
\subsubsection{Turbulent bath}
We can initialize the R-Z plane with a turbulent bath with a certain amplitude $A$.
This especially has the goal to destabilize the edge region right inside the
last closed flux surface. Notice that the core region is rather stable
and quickly damps away fluctuations.
Again, we transform this to all poloidal planes along the magnetic field lines and multiply the bath with
\begin{align} \label{eq:initial_turbulent}
\tilde n_e(R,Z,\varphi) = \tilde n_{\text{bath}}(R,Z,\varphi)\Theta_{\alpha}(-\psi_p(R, Z)-\alpha) H(Z-Z_X)
\end{align}
\subsubsection{Zonal flows}
We can initialize the R-Z plane with zonal flows of amplitude $A$ and
wavelength $k_\psi$ aligned with the magnetic flux surfaces.
\begin{align} \label{eq:initial_zonal_flow}
    \tilde n_{\text{zonal}}(R,Z) &= A \sin (2\pi k_\psi \psi_p(R,Z)) \nonumber\\
\tilde n_e(R,Z,\varphi) &= \tilde n_{\text{zonal}}(R,Z)\Theta_{\alpha}(-\psi_p(R, Z)-\alpha) H(Z-Z_X)
\end{align}
\subsubsection{Turbulence on Gaussian profile}
Instead of the flux-aligned profile we can also choose a toroidally symmetric Gaussian profile
\begin{align} \label{eq:profile_blob}
  n_{prof}(R,Z) = n_0 + \triangle n_{peak} \exp\left( -\frac{(R - R_0 - p_x a)^2 + (Z-p_ya)^2}{\sigma^2} \right)
\end{align}
on top of which we can add the turbulent bath $\tilde n_{\text{bath}}$ and finally dampen it by
\begin{align}\label{eq:turbulence_on_gaussian}
n_e(R,Z,\varphi,0) = (n_{prof}(R,Z) + \tilde n_{\text{bath}})\Theta_\alpha( 1- \sqrt{(R-R_0)^2 + Z^2}/a)
\end{align}

\subsection{Sinks and sources} \label{sec:sources}
We can choose the source terms $S_N$ to either force a profile
$n_{\text{prof}}$ or provide a constant influx of particles in the
core of our domain, where our model does not apply.
We thus define a particle sink/source for electrons as
\begin{align} \label{eq:electron_source}
  S_{n_e}(R,Z,\varphi, t) &= \omega_s \begin{cases}
    (n_{prof}(R,Z) - n_e(R,Z,\varphi, t))\Theta_\alpha( \rho(R,Z) - \rho_s) H(Z-Z_X) \quad \text{ forced}\\
    S_{prof}(R,Z)\quad \text{ influx}
    \end{cases} \\
    \rho(R,Z) &:= \frac{\psi_{p,\min}- \psi_p(R,Z) }{\psi_{p,\min}}, \quad
    \psi_p(R,Z):= (1-\rho(R,Z))\psi_{p,\min},\\ \text{In general }\psi_{p,\min} &= \psi_p(R_O, Z_O) \neq\psi_{p}(R_0,0)
\end{align}
with $0 < \rho_{s}<1$
where $\omega_s$ is the source strength parameter and $R_O$, $Z_O$ are the coordinates of the O-point.
The forced source will result in exponential adaption of the core
density profile of the form $n_e \propto n_{prof}+(n_{prof}-n_{e,0})e^{-\omega_st}$.

We can choose the constant influx
\begin{align} \label{eq:electron_source_influx}
  S_{prof}(R,Z) &= \Theta_\alpha( \rho(R,Z) - \rho_s) H(Z-Z_X)
\end{align}
or a Torpex inspired source profile
\begin{align} \label{eq:electron_source_torpex}
  S_{prof}(R,Z) &= 
  \begin{cases}
    \exp\left( - \frac{(R-R_0)^2}{a^2 }- \frac{(Z-Z_0)^2}{b^2}\right) \text{ if} R > R_0 \\
    \frac{1}{2}\exp\left( - \frac{(R-R_0)^2}{a^2} -2c(R-R_0)(Z-Z_0)- \frac{(Z-Z_0)^2}{b^2} \right) \\
  +\frac{1}{2}\exp\left( - \frac{(R-R_0)^2}{a^2} +2c(R-R_0)(Z-Z_0)- \frac{(Z-Z_0)^2}{b^2} \right) \text{ else}
              \end{cases}
\end{align}
with $a=0.0335$m, $b=0.05$m, $c=565m^{-2}$, $R_0=0.98$m and $Z_0=-0.02$m.


For ions we use
\begin{align}
    S_{N_i} = \Gamma_{1,i}^{-1} S_{n_e} = \left(1-\frac{1}{2}\mu_i \tau_i \Delta_\perp\right) S_{n_e}
  \label{eq:ion_source}
\end{align}
Note that Eq.~\eqref{eq:ion_source} is explicitly chosen as to avoid vorticity generation
by the particle source (cf.~Section~\ref{sec:conservation}). $S_{n_e}$ needs to be smooth
so that $\nabla_\perp^2 S_{n_e}$ is well defined.

%The idea for the terms $S_U$ is mainly to provide more numerical stability
%in the corner regions of the domain, where the parallel derivative may lead
%to unfavourable numerical instabilities.
%For both electrons and ions we choose
%\begin{align} \label{eq:velocity_source}
%  S_{U}(R,Z,\varphi, t) := -\omega_d U \Theta( \rho(R,Z) - \rho_d)
%\end{align}
%with $\rho_d > 1$.

\subsection{Implemented form}
The form that we implement avoids derivatives on the product of
two functions for which we have no boundary conditions
\begin{subequations}
    \begin{align}
    \frac{\partial}{\partial t} N =&
        - \frac{1}{B}[\psi, N]_{\perp}%\nonumber\\
        - \bar \nabla_\parallel \left( NU\right)
        - NU\left(\vec \nabla\cdot\bhat+\vec \nabla\cdot\tilde{\vec b}_\perp\right)
        - \tau \mathcal K(N) \nonumber \\&
        - N \mathcal K(\psi)
        -\mu \mathcal K_{\nabla\times\bhat}(NU^2)
        -\mu NU^2\nabla\cdot \vec{ \mathcal K_{\nabla\times\bhat}}
        + \nu_\perp\Delta_\perp N + \nu_\parallel \Delta_\parallel N + S_N, \\
    \frac{\partial}{\partial t} W =&
        - \frac{1}{B}\left[\psi, U\right]_{\perp}%& \nonumber\\
        - \frac{1}{\mu} \bar \nabla_\parallel \psi% \nonumber\\
        - \frac{1}{2}\bar \nabla_\parallel U^2
        -\frac{\tau}{\mu} \bar \nabla_\parallel \ln N
        - U\mathcal K_{\nabla\times\bhat}(\psi)
        - \tau \mathcal K(U)
        -\tau U\nabla\cdot\vec{ \mathcal K_{\nabla\times\bhat}}\nonumber\\&
        - \left(2\tau + {\mu}U^2\right) \mathcal K_{\nabla\times\bhat} (U)
        -2\tau U\mathcal K_{\nabla\times\bhat}(\ln N)
        - \frac{\eta}{\mu} \frac{n_e}{N}n_e(U_i - u_e) \nonumber\\&
        + \nu_\perp\Delta_\perp U
        + \nu_\parallel \Delta_\parallel U + S_U,
        \label{eq:EgyrofluidU} \\
        W&:= \left( U + \frac{A_\parallel}{\mu}\right)
    \end{align}
    \label{eq:Egyrofluid}
\end{subequations}
together with
$\bar\nabla_\parallel f = \nabla_\parallel f + A_\parallel \mathcal K_{\nabla\times\bhat}(f) + \frac{1}{B}[ f, A_\parallel]_\perp$
and $\vec \nabla \cdot \tilde{ \vec b}_\perp = A_\parallel \vec \nabla\cdot\vec{ \mathcal{ K}_{\nabla\times\bhat}} - \mathcal K_{\nabla B}(A_\parallel) $
and
\begin{subequations} \label{eq:elliptic}
  \begin{align}
    -\nabla\cdot\left( \frac{N_i}{B^2}\nabla_\perp \phi \right) &= \Gamma_{1,i} N_i - n_e, \quad\quad
    \Gamma_{1,i}^{-1} = 1-\frac{1}{2}\tau_i\mu_i \Delta_\perp \\
    \psi_e = \phi, \quad \psi_i &= \Gamma_{1,i}\phi -\frac{\mu_i}{2}\frac{(\nabla_\perp\phi)^2}{B^2} \\
    \left(\frac{\beta}{\mu_i}N_i - \frac{\beta}{\mu_e}n_e-\Delta_\perp\right)
    A_\parallel &= \beta\left(N_iW_i-n_e w_e\right)
  \end{align}
\end{subequations}
Note that the negative signs make the operators in Eq.~\eqref{eq:elliptic} positive definite.
\subsection{Conservation laws} \label{sec:conservation}
\subsubsection{Mass conservation}
The density equation directly yields the particle conservation
\begin{align} \label{eq:mass_theorem}
  \frac{\partial}{\partial t} n_e
  + \nabla\cdot\vec{ j_{n_e}}
  =  \Lambda_{n_e}+S_{n_e}
\end{align}
The terms of the particle conservation thus read
\begin{align} \label{eq:mass_conservation}
  n_e= & n_e,\\
  \vec j_{n_e} =& n_e\left(
  \vec v_E + \vec v_C + \vec v_{K} +u_e\left(\bhat+\tilde{\vec b}_\perp\right)  \right) \nonumber\\
  =& n_e \left(\frac{\bhat\times \nabla\phi}{B} 
  + \tau_e \frac{\bhat\times\nabla n_e}{n_eB} 
  + \mu_e u_e^2\vec K_{\nabla\times\bhat} 
  + u_e(\bhat + \tilde{\vec b}_\perp) \right), \\
  \Lambda_{n_e} =&
  \nu_\perp\Delta_\perp n_e + \nu_\parallel\Delta_\parallel n_e
\\
  S_{n_e} =&  S_{n_e}
\end{align}
Notice that we used
\begin{align}
n_e \vec K = n_e\nabla\times\frac{\bhat}{B} = \nabla\times n_e\frac{\bhat}{B} + \frac{\bhat\times\nabla n_e}{B}
\label{}
\end{align}
such that we can define the diamagnetic flux in the particle flux since
the rotation vanishes under the divergence.

Let us here also derive the particle flux \eqref{eq:mass_conservation} through a flux surface
\begin{align} \label{eq:particle_flux}
 \vec j_{N}\cdot \vec \nabla v %=& N\left( \vec v_E + \vec v_C + \vec v_{\nabla
 %B} + U \left(\bhat + \tilde{\vec b}_\perp\right)\right) \cdot \vec
 %\nabla\psi_p \nonumber\\
 =&
  \frac{\d v}{\d \psi_p} N\left[\frac{1}{B}[\psi, \psi_p]_\perp + \left(\tau + \mu U^2\right)
   \mathcal K_{\nabla\times\bhat}(\psi_p) + \tau  \mathcal K_{\nabla B}(\psi_p) \right] \nonumber\\
 &+ NU\frac{\d v}{\d \psi_p}\left [\left( A_\parallel \mathcal
 K_{\nabla\times\bhat}(\psi_p) + \frac{1}{B}[\psi_p, A_\parallel]_\perp\right) \right] \\
\end{align}


\subsubsection{Energy theorem}
The terms of the energy theorem are
\begin{align} \label{eq:energy_theorem}
\partial_t \mathcal E +
\nabla \cdot \vec j_{\mathcal E}
= \Lambda_{\mathcal E}
+  S_{\mathcal E}
+  R_{\mathcal E}
\end{align}
with ( $z_e=-1$ and $z_i=+1$)
\begin{align} \label{eq:energy_conservation}
  \mathcal{E}= & z_e\tau_e n_e \ln{(n_e)} +z_i\tau_i N_i\ln{(N_i)}
  +\frac{1}{2\beta}\left(\vec \nabla_\perp A_\parallel\right)^2
   +  \frac{1}{2} z_i \mu_i N_i u_E^2  \nonumber\\
   & +\frac{1}{2} z_e\mu_e  n_e u_e^2
  +\frac{1}{2} z_i\mu_i  N_i U_i^2,\\
  \vec j_{\mathcal E} =& \sum_s z\left[
  \left(\tau \ln N + \frac{1}{2}\mu U^2 + \psi \right)N\left(
  \vec v_E + \vec v_C + \vec v_{K} +U\left(\bhat+\tilde{\vec b}_\perp\right)  \right) \right]
  \nonumber\\
  &+ \sum_z z\left[\mu \tau NU^2\vec K_{\nabla\times\bhat} + \tau NU \left(\bhat + \tilde{\vec b}_\perp\right)\right], \\
  \Lambda_{\mathcal E} =&  \sum_s z\left[\left( \tau\left( 1+\ln{N}\right) + \psi + \frac{1}{2} \mu U^2 \right)
  \left(\nu_\perp\Delta_\perp N + \nu_\parallel\Delta_\parallel N\right)  +  \mu NU\left(\nu_\perp\Delta_\perp U + \nu_\parallel\Delta_\parallel U\right) \right]
\nonumber \\
  S_{\mathcal E} =&  \sum_s  z\left[ \left(\tau\left( 1+\ln{N}\right) +\psi + \frac{1}{2} \mu U^2 \right)S_{N}  + \mu NU S_U\right]
\nonumber \\
  R_{\mathcal E} =&  -\eta_\parallel  \left[ n_e(U_i-u_e)\right]^2.
\end{align}
where in the energy flux $\vec j_{\mathcal E}$
we neglect terms  containing time derivatives
of the eletric and magnetic potentials and we sum over all species.
The energy density $\mathcal E$ consists of the Helmholtz free energy density for electrons and ions,
the \(\vec{E} \times \vec{B}\) energy density, the parallel energy densities for electrons and ions and the perturbed magnetic field energy density.
In \(\Lambda\) we insert the dissipative terms of Section~\ref{sec:dissres}. \\
Replace $\Delta_\perp$ with $-\Delta_\perp^2$ when hyperviscous diffusion is chosen
for the diffusion terms in the above equations.

We have the energy flux through a flux surface
\begin{align}
 \vec j_{\mathcal E}\cdot \vec \nabla v =&%\frac{\d v}{\d \psi_p} \vec j_{\mathcal E}\cdot \vec \nabla \psi_p  =
\frac{\d v}{\d \psi_p}\sum_s z\left (\tau\ln N + \frac{1}{2}\mu U^2 + \psi\right) \vec j_N\cdot\vec\nabla\psi_p
+ z \mu\tau NU^2 \mathcal K_{\nabla\times\bhat}(\psi_p) \nonumber\\
&+ z \tau NU
 \left( A_\parallel \mathcal
 K_{\nabla\times\bhat}(\psi_p) + \frac{1}{B}[\psi_p, A_\parallel]_\perp\right)
\label{eq:energy_flux}
\end{align}

\subsection{ Force balance equation}
In order to discuss poloidal flows let us first define orthogonal vectors $\{\vec{ \hat\zeta}, \vec{\hat\eta},\bhat\}$
\begin{align}
\vec{\hat\zeta} := \nabla\psi_p,\quad
\vec{\hat \eta} := \frac{\bhat \times\hat \zeta}{B}
\end{align}
where $\vec{\hat\eta}$
points in the counter-clockwise poloidal direction with our choice of the magnetic field direction.
Notice that $\vec{\hat \eta}$ in general has a (small) toroidal component in addition to the dominant poloidal component.

%We then have $\nabla\phi = \hat \zeta\partial_{\hat\zeta}\phi + \hat\eta\partial_{\hat\eta} \phi$
%with $\partial_{\hat\zeta}:= \hat\zeta\cdot\nabla$ and $\partial_{\hat\eta}:=\hat\eta\cdot\nabla$.
Furthermore, from $\vec u_E = \bhat\times \vec\nabla\phi/B$ and $\vec u_d = \tau_i \bhat\times\nabla \ln N_i$
we can derive
\begin{align}
u_E^{\hat\eta} &:= \vec u_E\cdot \vec{\hat\eta} = \frac{\partial_{\hat\zeta} \phi}{B^2}
&&u_d^{\hat\eta} := \vec u_d\cdot \vec{\hat\eta} =  \frac{\tau_i \partial_{\hat\zeta} \ln N_i}{B} \\
u_E^{\hat\zeta}&:= \vec u_E\cdot \vec{\hat\zeta}  = -\partial_{\hat\eta} \phi
&&u_d^{\hat\zeta}:= \vec u_d\cdot \vec{\hat\zeta} = -B \tau_i \partial_{\hat\eta} \ln N_i
\end{align}
With this we write the force balance equation (in the LWL)
\begin{align}
&\frac{\partial}{\partial t} \left\langle \mu_i N_i \left(
\frac{\partial_{\hat\zeta}\phi\,}{B^2} + \tau_i \partial_{\hat\zeta} \ln N_i\right) \right\rangle
\nonumber\\
&+ \frac{\partial}{\partial v} \frac{\d v}{\d\psi_p} \left\langle \mu_i N_i\partial_{\hat\eta} \phi \left(\frac{\partial_{\hat\zeta}\phi }{B^2}
+ \tau_i \partial_{\hat\zeta} \ln N_i \right) + \frac{1}{\beta} \partial_{\hat\zeta} A_\parallel \partial_{\hat\eta}A_\parallel \right\rangle
\nonumber\\
&= \left\langle (z_e \tau_e n_e + z_i\tau_i N_i)\vec K\cdot\nabla\psi_p + (z_e\mu_e n_eu_e^2 + z_i\mu_i N_iU_i^2)\vec K_{\nabla\times\bhat}\cdot\nabla\psi_p \right\rangle
\end{align}
where the right hand side represents the Lorentz force $\vec j\times\vec B$.
Notice that 
\begin{align}
\left\langle (z_e \tau_e n_e + z_i\tau_i N_i)\vec K\cdot\nabla\psi_p\right\rangle
= \left\langle \frac{ \bhat\times \nabla (z_e\tau_e n_e + z_i\tau_i N_i)}{B}\cdot\nabla\psi_p\right\rangle
= -\left\langle\partial_{\hat\eta} (z_e\tau_e n_e + z_i\tau_i N_i)\right\rangle
\end{align}
We can interpret the force balance as the flux surface average of a
continuity equation
\begin{align}
&\partial_t \Omega + \nabla\cdot \vec j_\Omega = S_\Omega \\
\Omega &:= \mu_i N_i \left(\frac{\nabla\psi_p\cdot\nabla\phi}{B^2} + \tau_i \nabla\psi_p \cdot\nabla \ln N_i\right) \\
\vec j_{\Omega} &:= -\Omega \vec u_E 
    - \frac{1}{\beta} \nabla\psi_p\cdot\nabla A_\parallel \frac{\bhat\times\nabla A_\parallel}{B} \\
    S_\Omega &:=  (z_e \tau_e n_e + z_i\tau_i N_i)\mathcal K(\psi_p) + (z_e\mu_e n_eu_e^2 + z_i\mu_i N_iU_i^2)\mathcal K_{\nabla\times\bhat}(\psi_p)
\end{align}



%%%%%%%%%%%%%%%%%%%%%%%%%%%%%%%%%%%%%%%


\subsection{Manufactured Solution}
In order to test the implementation we manufacture a solution to Eqs.~\eqref{eq:Egyrofluid} and \eqref{eq:elliptic} of the form
\begin{align*}
n_e(R,Z,\varphi, t) &:= 1 + 0.5\sin(\pi(R-R_0))\sin(\pi Z)\sin(\varphi)\sin(\pi t) \\
N_i(R,Z,\varphi, t) &:= n_e(R,Z,\varphi,t) = \gamma_{ N_i}  \\
u_e(R,Z,\varphi, t) &:= \sin(2\pi(R-R_0))\sin(2\pi Z)\sin(2\varphi)\sin(2\pi t)/(3\sqrt{-\mu_e}) \\
U_i(R,Z,\varphi, t) &:= \sqrt{-\mu_e}u_e(R,Z,\varphi,t) \\
\phi(R,Z,\varphi,t) &:= \sin(3\pi(R-R_0))\sin(3\pi Z)\sin(3\varphi)\sin(3\pi t)/5; \\
\psi(R,Z,\varphi,t) &:= \phi(R,Z,\varphi, t) = \gamma_{\phi} \\
A_\parallel( R,Z,\varphi,t) &:= \beta\sin(4\pi(R-R_0))\sin(4\pi Z)\sin(4\varphi)\sin(4\pi t)/4;
\end{align*}
We choose circular flux surfaces of the form
\begin{align*}
\psi_p(R,Z) :=0.5((R-R_0)^2 + Z^2),\quad
I_p(R,Z):=I_0
\end{align*}
with $R_0=10$ and $I_0=20$ and a simulation box $[R_0-a,R_0+a]\times[-a,a]\times[0,2\pi]$.
We then symbolically compute (with the help of Mathematica) source terms that we insert to the right hand side of
the corresponding equation in code (\texttt{manufactured.h}) and simulate from $t=0...10^{-3}$.
By comparing the numerical solution to the manufactured one we can observe the convergence of our numerical methods. Note that in order to better distinguish
the convergence of the DG discretized terms from our parallel derivative
we can selectively choose to only activate perpendicular (including $A_\parallel$ terms) or parallel terms (those that involve derivatives along $\bhat$).

Unfortunately, we were unable to find a closed solution for the energy integrals with the above fields.

\section{Numerical methods}
discontinuous Galerkin on structured grid
\rowcolors{2}{gray!25}{white} %%% Use this line in front of longtable
\begin{longtable}{p{3cm}l>{\RaggedRight}p{8cm}}
\toprule
\rowcolor{gray!50}\textbf{Term} &  \textbf{Method} & \textbf{Description}  \\ \midrule
    coordinate system & Cylindrical & equidistant discretization of $[R_{\min},R_{\max}] \times [Z_{\min},Z_{\max}] \times [0,2\pi]$ (Eq.~\eqref{eq:box}, equal number of Gaussian nodes in $R$ and $Z$, equidistant planes in $\varphi$ with one Gaussian node \\
Advection terms & direct DG & DG approximation with centered flux of derivatives \\
Elliptic terms & local DG & The local DG approximation with centered flux \\
Helmholtz and Elliptic matrix inversions & multigrid/ conjugate gradient & Use previous two solutions to extrapolate initial guess and $1/\chi$ as preconditioner \\
Parallel derivatives & regular  FCI & cf.~\cite{Held2016,Stegmeir2017} \\
time & Multistep "Karniadakis" & \\
\qquad explicit & Multistep "Karniadakis" & $3$rd order explicit\\
\qquad implicit & Multistep "Karniadakis" & $2$nd order implicit, contains perp. Diffusion and Resistive terms. In every iteration of the implicit inversion we need to solve for $A_\parallel$\\
\bottomrule
\end{longtable}

\section{Usage}

Compilation:\\
\texttt{make feltor device=\{gpu,omp\}} Compile \texttt{feltor.cu} (only shared memory)\\
\texttt{make feltor\_hpc device=\{gpu,omp\}} Compile \texttt{feltor\_hpc.cu} for shared memory system. Needs {\it serial netcdf} \\
\texttt{make feltor\_mpi device=\{gpu,omp,skl,knl\}} Compile \texttt{feltor\_hpc.cu} for distributed memory systems. Also needs {\it serial netcdf}\\
Usage:\\
\texttt{./feltor\_hpc input.json geometry.json output.nc [initial.nc]} \\
\texttt{./feltor\_mpi input.json geometry.json output.nc [initial.nc]} \\
\texttt{./feltor input.json geometry.json } \\

The programs \texttt{feltor\_hpc.cu} and \texttt{feltor.cu} expect two input
files \texttt{input.json} and \texttt{geometry.json}, described in Sections~\ref{sec:input_file} and \ref{sec:geometry_file}.
The first is for the physical and numerical parameters of the model equations
while the latter describes the Solov'ev equilibrium.
 The program \texttt{feltor.cu} plots the results directly to the screen using \texttt{glfw3}.
The program \texttt{feltor\_hpc.cu} writes results into
the output file \texttt{output.nc}.
 The output file is described in Section~\ref{sec:output_file}.
 The optional file \texttt{initial.nc} can be used to initialize a simulation from an existing file.
 This behavior is described in Section~\ref{sec:restart_file}.


\subsection{Input file structure} \label{sec:input_file}
Input file format: json

%%This is a booktabs table
\begin{longtable}{llll>{\RaggedRight}p{6cm}}
\toprule
\rowcolor{gray!50}\textbf{Name} &  \textbf{Type} & \textbf{Example} & \textbf{Default} & \textbf{Description}  \\ \midrule
n      & integer & 3 & - &Number of Gaussian nodes in R and Z \\
Nx     & integer &52& - &Number of grid points in R \\
Ny     & integer &52& - &Number of grid points in Z \\
Nz     & integer &16& - &Number of grid points in $\varphi$ (determines dt since parallel velocity dominates timestep) \\
dt     & integer &1e-2& - & time stepsize in units of $c_s/\rho_s$ \\
compression & integer[2] & [2,2] & [1,1] & Compress output file by reducing points in x and y: output contains n*Nx/c[0] points in x,
    (has to divde Nx evenly), and n*Ny/c[1] points in y,
    (has to divde Ny evenly)\\
inner\_loop & integer & 2  & 1 & Number of time steps between updates to the time integrated quantities. Note that integrating selected
quantities in time during the simulation is how we maintain the time-resolution in the file output (cf. \ref{sec:output_file}).\\
itstp       & integer & 2  & - &{ \tt inner\_loop*itstp} is the number of timesteps between file outputs (2d and 3d quantities);
Note that 1d and 0d quantities can only be computed in diagnostics since we can't compute flux-integrals in parallel in MPI. \\
maxout      & integer & 10 & - & Total Number of fields outputs excluding first (The total number of time steps is {\tt maxout$\cdot$itstp$\cdot$inner\_loop}) \\
eps\_time   & float & 1e-7  & - & Tolerance for solver for implicit part in time-stepper (if too low, you'll see oscillations in $u_e$ and/or $\phi$) \\
rtol  & float &1e-6   & - &Tolerance of adaptive time-stepper. (Ignored in Multistep) \\
eps\_pol    & float & 1e-6  & - &  Tolerance for residual of the inversion of polarisation and induction Eq. (should not be more than a factor 10 from eps\_time for $\beta\neq  0$ ) \\
jumpfactor  & float & 1 & 1 & Jumpfactor $\in \left[0.01,1\right]$ in the local DG method for the elliptic terms\\
eps\_gamma  & float & 1e-6  & - & Tolerance for $\Gamma_1$  \\
stages      & integer & 3 & 3 & number of stages in multigrid, $2^{\text{stages-1}}$
has to evenly divide both $N_x$ and $N_y$\\
refineDS     & integer[2] & [10,10] & [10,10] & refinement factor in FCI approach in R- and Z-direction\\
rk4eps     & float & 1e-5 & 1e-5 & Accuracy of fieldline integrator in FCI\\
mu         & float & -0.000272121& - & $\mu_e =-m_e/m_i$.
    One of $\left\{ -0.000544617, -0.000272121, -0.000181372 \right\}$\\
tau        & float &1      & - & $\tau = T_i/T_e$  \\
beta       & float & 5e-6  & 0 & Plasma beta $5\cdot 10^{-6}$ (TJK), $4\cdot 10^{-3}$ (Compass), If $0$, then the model is electrostatic \\
nu\_perp   & float &1e-3   & - & perpendicular viscosity $\nu_\perp$ \\
perp\_diff & string & "viscous" & "viscous" & "viscous": $\Lambda_\perp\propto \nu_\perp\Delta_\perp$ , "hyperviscous": $\Lambda_\perp \propto -\nu_\perp\Delta_\perp^2$\\
nu\_parallel & float &1e-1 & - & parallel viscosity $\nu_\parallel$ \\
resistivity & float &1e-4  & - & parallel resistivity parameter Eq.~\eqref{eq:resistivity}\\
curvmode  & string & "low beta"  & "toroidal"& curvature mode ("low beta", "true": no approximation - requires significantly more resolution in Nz, "toroidal": toroidal field approx - elliptic equation does not need communication in z)  \\
symmetric & bool & false & false & If true, initialize all quantities symmetric in $\varphi$ (effectively reducing the problem to 2d). The input $N_z$ is used to construct the parallel derivatives and then overwritten to $N_z\equiv 1$. \\
bc & dict & & & Dictionary of boundary conditions (note that $A_\parallel$ has the same bc as $U$) \ldots\\
\qquad density   & char[2] & [DIR,DIR] & -  & boundary conditions in x and y for $n_e$ and $N_i$\\
\qquad velocity  & char[2] & [DIR,DIR] & - & boundary conditions in x and y for $u_e$ and $U_i$ and $A_\parallel$\\
\qquad potential & char[2] & [DIR,DIR] & - & boundary conditions in x and y for $\phi$ and $\psi$\\
    boxscaleR  & float[2] & [1.1,1.1]     & [1.05,1.05] & $[\varepsilon_{R-}, \varepsilon_{R+}]$ scale left and right boundary in units of $a$ Eq.~\eqref{eq:box}\\
    boxscaleZ  & float[2] & [1.2,1.1]     & [1.05,1.05] & $\varepsilon_{Z-}, \varepsilon_{Z+}$ scale lower and upper boundary in units of $ae$ Eq.~\eqref{eq:box} \\
initne    & string & "turbulence"     & "blob"  & initial condition for the
perturbation $\tilde n$ in \eqref{eq:initial_ne}. "zonal" (Eq.~\eqref{eq:initial_zonal_flow}),
    "blob" = blob simulations (several rounds fieldaligned),
    "straight blob" = straight blob simulation( 1 round fieldaligned),
    "turbulence" = turbulence simulations ( 1 round fieldaligned, Eq.~\eqref{eq:initial_turbulent})
    "turbulence\_on\_gaussian" = Gaussian bg. profile with turbulence perturbation Eq.~\eqref{eq:turbulence_on_gaussian}
    See the file {\tt init.h} to add your own custom condition.
    \\
initphi   & string & "zero"  & "balance" & initial condition for $\phi$ and thus $N_i$ (Eq.~\eqref{eq:initphi}: "zero" : $\phi = 0$, vanishing
electric potential, "balance": ExB vorticity equals ion diamagnetic vorticity (For $\tau_i =0 $ both are the same)
\\
amplitude  & float &0.01   & - & amplitude $A$ of initial perturbation (blob, turbulent bath or zonal flow)  \\
sigma      & float &2      & - & Gaussian variance in units of $\rho_s$ \\
posX       & float &0.3    & - & Gaussian R-position in units of $a$\\
posY       & float &0.0    & - & Gaussian Z-position in units of $a$ \\
sigma\_z    & float &0.25   & - & toroidal variance in units of $R_0$ of the fieldline-following initialization \\
k\_psi     & float &0    & - & zonal mode wave number (only for "zonal" initial condition)  \\
nprofileamp& float &4   & - & Profile peak amplitude $N_{peak}$ in Eq.~\eqref{eq:density_profile} and Eq.~\eqref{eq:turbulence_on_gaussian}\\
alpha       & float & 0.2 & - & Width $\alpha$ of the Heaviside Eq.~\eqref{eq:approx_heaviside} in the density and source profiles (should be small but cannot be too small if $\tau_i > 0$ else $\Delta_\perp n_e$ explodes) \\
source      & float & 0    & 0 & profile source rate $\omega_s$ in Eq.~\eqref{eq:electron_source} \\
source\_type & string & "profile" & "profile" & The type of source to use: "profile" the source is multiplied by $(n_{prof} - n)$ to relax to the initial profile Eq.~\eqref{eq:electron_source};
"influx" the source has a constant source rate Eq.~\eqref{eq:electron_source_influx},
"torpex": Torpex inspired source profile Eq.~\eqref{eq:electron_source_torpex},
    See the file {\tt init.h} to add your own custom source. \\
rho\_source & float & 0.2  & 0.2 & Source region boundary $0<\rho_{s}<1$ in Eq.~\eqref{eq:electron_source} and Eq.~\eqref{eq:electron_source_influx}  \\
%damping     & float & 0    & 0   & Friction coefficient $\omega_d$ in Eq.~\eqref{eq:velocity_source} \\
alpha\_mag   & float & 0.05 & - & Width $\alpha$ of the Heaviside in the modified $\psi_p$ function \eqref{eq:modified_psip}. If zero, then we do not modify the magnetic field and rho\_damping is ignored.\\
rho\_damping& float & 0.2  & 1.2 & Modification region boundary $\psi_0 = (1-\rho_d)\psi_{p,\min}$ in Eq.~\eqref{eq:modified_psip}. \\
\bottomrule
\end{longtable}
The default value is taken if the value name is not found in the input file. If there is no default and
the value is not found,
the program exits with an error message.
\subsection{Geometry file structure} \label{sec:geometry_file}
File format: json

%%This is a booktabs table
\begin{longtable}{llll>{\RaggedRight}p{7cm}}
\toprule
\rowcolor{gray!50}\textbf{Name} &  \textbf{Type} & \textbf{Example} & \textbf{Default} & \textbf{Description}  \\ \midrule
    A      & float & 0 &  0 & Solovev parameter in Eq.~\eqref{eq:solovev} \\
    B      & float & 1 &  1 & Solovev parameter in Eq.~\eqref{eq:solovev} \\
    c      & float[12] &  - & - & Solovev coefficients in Eq.~\eqref{eq:solovev} \\
    R\_0   & float & - & -  & Major radius $R_0$ in units of $\rho_s$ in Eq.~\eqref{eq:solovev} (This is the only geometry quantity to change if $\rho_s$ changes)\\
    elongation    & float & 1 & - & Elongation $e$, used in determining the box size Eq.~\eqref{eq:box} and the initial guess for the location of the X-point $Z_X = -1.1 ea$ \\
    triangularity & float & 0 & - & Triangularity $\delta$, used in the initial guess for the location of the X-point $R_X = R_0-1.1\delta a$ \\
    inverseaspectratio & float & 0.16667 & - & minor to major radius $a/R_0$ \\
\bottomrule
\end{longtable}
The default value is taken if the value name is not found in the input file. If there is no default and
the value is not found,
the program exits with an error message.

\subsection{Output} \label{sec:output_file}
Output file format: netcdf-4/hdf5; A coordinate variable (Coord. Var.) is a Dataset with the same name as a dimension.
We follow CF Conventions CF-1.7
\url{http://cfconventions.org/Data/cf-conventions/cf-conventions-1.7/cf-conventions.html}
and write according attributes into the file.
The command \texttt{ncdump -h output.nc} gives a full list of what a file contains.
Here, we list the content without attributes
since the internal netcdf information does not display equations.
%
%Name | Type | Dimensionality | Description
%---|---|---|---|
\begin{longtable}{lll>{\RaggedRight}p{7cm}}
\toprule
\rowcolor{gray!50}\textbf{Name} &  \textbf{Type} & \textbf{Dimension} & \textbf{Description}  \\ \midrule
inputfile  &     text attribute & - & verbose input file as a string (valid JSON, no comments) \\
geomfile   &     text attribute & - & verbose geometry input file as a string (valid JSON, no comments) \\
x                & Coord. Var. & 1 (x) & $R$-coordinate (computational space, compressed size: $nN_x/c_x$)\\
y                & Coord. Var. & 1 (y) & $Z$-coordinate (computational space, compressed size: $nN_y/c_y$)\\
z                & Coord. Var. & 1 (z) & $\varphi$-coordinate (computational space, size: $N_z$) \\
time             & Coord. Var. & 1 (time)& time at which fields are written (variable size: maxout$+1$, dimension size: unlimited) \\
xc           & Dataset & 3 (z,y,x) & Cartesian x-coordinate $x=R\sin(\varphi)$ \\
yc           & Dataset & 3 (z,y,x) & Cartesian y-coordinate $y=R\cos(\varphi)$\\
zc           & Dataset & 3 (z,y,x) & Cartesian z-coordinate $z=Z$ \\
Psip             & Dataset & 3 (z,y,x) & Flux function $\psi_p(R,Z)$ \\
Nprof            & Dataset & 3 (z,y,x) & Density profile $n_\text{prof}$ used in the forcing source \\
Source           & Dataset & 3 (z,y,x) & Source  profile $\Theta_\alpha(\rho(R,Z) - \rho_s) H(Z-Z_X)$\\
BR               & Dataset & 3 (z,y,x) & Contravariant magnetic field component $B^R$ \\
BZ               & Dataset & 3 (z,y,x) & Contravariant magnetic field component $B^Z$ \\
BP               & Dataset & 3 (z,y,x) & Contravariant magnetic field component $B^\varphi$ \\
electrons        & Dataset & 4 (time, z, y, x) & electron density $n_e$ \\
ions             & Dataset & 4 (time, z, y, x) & ion density $N_i$ \\
Ue               & Dataset & 4 (time, z, y, x) & electron velocity $u_e$ \\
Ui               & Dataset & 4 (time, z, y, x) & ion velocity $U_i$ \\
potential        & Dataset & 4 (time, z, y, x) & electric potential $\phi$ \\
induction        & Dataset & 4 (time, z, y, x) & parallel vector potential $A_\parallel$ \\
X\_2d            & Dataset & 3 (time,y,x) & Selected plane $X(\varphi=0)$ \\
X\_ta2d          & Dataset & 3 (time,y,x) & Toroidal average $\langle X
    \rangle_\varphi$ Eq.~\eqref{eq:phi_average} \\
Y\_tt\_2d        & Dataset & 3 (time,y,x) & Time integrated (between two outputs) selected plane
$\int_{t_0}^{t_1}\d t Y(\varphi=0) $
where $t_1 - t_0 = ${\tt dt*inner\_loop*itstp} and {\tt itstp} is the number of discretization points\\
Y\_tt\_ta2d      & Dataset & 3 (time,y,x) & Time integrated (between two outputs) toroidal average (Eq.~\eqref{eq:phi_average})
$\int_{t_0}^{t_1}\d t \langle Y \rangle_\varphi$
where $t_1 - t_0 = ${\tt dt*inner\_loop*itstp} and {\tt itstp} is the number of discretization points\\
\bottomrule
\end{longtable}
where the time integrals are computed with the help of Simpson's rule
and X $\in$
\begin{longtable}{llll}
\toprule
\rowcolor{gray!50}\textbf{Name} &  \textbf{Equation} & \textbf{Name} &  \textbf{Equation}\\
\midrule
    electrons &$n_e$ &
    ions &$N_i$ \\
    Ue &$u_e$ &
    Ui &$U_i$ \\
    potential &$\phi$ &
    psi &$\psi$ \\
    induction &$A_\parallel$ &
    vorticity &$-\Delta_\perp\phi$ \\
    dssue & $\nabla_\parallel^2 u_e$&
    dppue & $\partial_\varphi^2 u_e$\\
    dpue2 & $(\partial_\varphi u_e)^2$&
    apar\_vorticity &$-\Delta_\perp A_\parallel$ \\
    neue &$n_e u_e$ &
    niui &$N_i U_i$ \\
    neuebphi &$n_eu_eb_\varphi$ &
    niuibphi &$N_iU_ib_\varphi$ \\
    lperpinv &$L_\perp^{-1} := |\vec\nabla_\perp n_e|/n_e$ &
    perpaligned &$(\vec\nabla_\perp n_e)^2/n_e$ \\
    lparallelinv &$L_\parallel^{-1} := |\nabla_\parallel n_e|/n_e$ &
    aligned &$ (\nabla_\parallel n_e)^2/n_e$ \\
    ne2 & $n_e^2$ &
    phi2 & $\phi^2$ \\
    nephi & $n_e\phi$ &
     & \\
    nelnne &$ z_e\tau_e n_e \ln n_e$ &
    nilnni &$ z_i\tau_i N_i \ln N_i$ \\
    aperp2 &$ (\nabla_\perp A_\parallel)^2/2/\beta$ &
    ue2   &$z_i\mu_i N_i u_E^2 /2$ \\
    neue2 &$ z_e\mu_e n_e u_e^2/2$ &
    niui2 &$ z_i\mu_i N_i U_i^2/2$ \\
    oexbe &$\mu_i n_e \nabla\psi_p\cdot\nabla\phi/B^2$ &
    oexbi &$\mu_i N_i \nabla\psi_p\cdot\nabla\phi/B^2$ \\
    odiae &$\mu_i \tau_i\nabla\psi_p\cdot\nabla n_e$ &
    odiai &$\mu_i \tau_i\nabla\psi_p\cdot\nabla N_i$ \\
\bottomrule
\end{longtable}
and the sources and currents Y $\in$
\begin{longtable}{ll}
\toprule
\rowcolor{gray!50}\textbf{Name} &  \textbf{Equation}\\
\midrule
    jsne &$ n_e (\vec v_E + \vec v_K + \vec v_C )\cdot\nabla \psi_p$ \\
    jsneA &$ n_e u_e \vec{\tilde b}_\perp  \cdot\nabla \psi_p$ \\
    lneperp &$ \Lambda_{\perp,n_e} = \nu_\perp \Delta_\perp n_e$ or $-\nu_\perp \Delta^2_\perp n_e$ \\
    lneparallel &$ \Lambda_{\parallel,n_e} = \nu_\parallel \Delta_\parallel n_e$ \\
    resistivity &-$\eta_\parallel n_e^2 (U_i-u_e)^2$ \\
    jsee &$z_e(\tau_e \ln n_e + \mu_e u_e^2/2 + \phi)n_e(\vec v_E + \vec v_C + \vec v_K)\cdot\nabla \psi_p
        + z_e \tau_e n_e u_e^2 \vec K_{\nabla\times\bhat}\cdot\nabla \psi_p$ \\
    jsei &$z_i(\tau_i \ln N_i + \mu_i U_i^2/2 + \psi_i)N_i(\vec v_E^i + \vec v_C + \vec v_K)\cdot\nabla \psi_p
        + z_i \tau_i N_i U_i^2 \vec K_{\nabla\times\bhat}\cdot\nabla \psi_p$ \\
    jseea &$z_e(\tau_e \ln n_e + \mu_e u_e^2 + \phi)n_e \vec {\tilde b}_\perp\cdot\nabla \psi_p
        + z_e \tau_e n_e u_e \vec{\tilde b}_\perp \cdot \nabla \psi_p $ \\
    jseia &$z_i(\tau_i \ln N_i + \mu_i U_i^2 + \psi_i)N_i \vec {\tilde b}_\perp\cdot\nabla \psi_p
        + z_i \tau_i N_i U_i \vec{\tilde b}_\perp \cdot \nabla \psi_p $ \\
    leeperp &$z_e(\tau_e(1+\ln n_e) + \phi + \mu_eu_e^2/2) \nu_\perp \Delta_\perp n_e + z_e\mu_e n_e u_e \nu_\perp \Delta_\perp u_e$ \\
    leiperp &$z_i(\tau_i(1+\ln N_i) + \psi_i + \mu_iU_i^2/2) \nu_\perp \Delta_\perp N_i + z_i\mu_i N_i U_i \nu_\perp \Delta_\perp U_i$ \\
    leeparallel &$z_e(\tau_e(1+\ln n_e) + \phi + \mu_eu_e^2/2) \nu_\parallel \Delta_\parallel n_e + z_e\mu_e n_e u_e \nu_\parallel \Delta_\parallel u_e$ \\
    leiparallel &$z_i(\tau_i(1+\ln N_i) + \psi_i + \mu_iU_i^2/2) \nu_\parallel \Delta_\parallel N_i + z_i\mu_i N_i U_i \nu_\parallel \Delta_\parallel U_i$ \\
    jsoexbi &$-(\mu_i N_i \nabla\psi_p\cdot\nabla\phi/B^2+\mu_i \tau_i\nabla\psi_p\cdot\nabla N_i)  \vec u_E\cdot\nabla \psi_p$ \\
    jsoexbe &$-(\mu_i n_e \nabla\psi_p\cdot\nabla\phi/B^2+\mu_i \tau_i\nabla\psi_p\cdot\nabla n_e) \vec u_E\cdot\nabla \psi_p$ \\
    jsoapar &$ \nabla\psi_p\cdot\nabla A_\parallel \bhat\times\nabla A_\parallel\cdot\nabla \psi_p/B/\beta$ \\
    socurve &$z_e\tau_e n_e \mathcal K(\psi_p)$ \\
    socurvi &$z_i\tau_i N_i \mathcal K(\psi_p)$ \\
    socurvkappae &$z_e\mu_e n_eu_e^2 \mathcal K_{\nabla\times\bhat}(\psi_p)$ \\
    socurvkappai &$z_i\mu_i N_iU_i^2 \mathcal K_{\nabla\times\bhat}(\psi_p)$ \\
\bottomrule
\end{longtable}
\subsection{Restart file} \label{sec:restart_file}
The program \texttt{feltor\_hpc.cu} has the possibility to initialize time and the fields with
the results of a previous simulation. This behaviour is enabled by giving an additional file \texttt{initial.nc}
to the command line. In this case the \texttt{initne} and \texttt{initphi} parameters of the input
file are ignored. Instead, the fields \texttt{electrons, ions, Ue, Ui, induction} at the latest timestep
are read from the given file, interpolated to the current grid and used to initialize the simulation.
Apart from that the behaviour of the program is unchanged i.e. the magnetic field, profiles, resolutions, etc.
are all taken from the regular input files. This means that the user must take care that these are consistent
with the paramters in the existing \texttt{initial.nc} file. Also note that we try to discourage
appending new results to an exisiting file directly,
because if for some reason the cluster crashes and the file is corrupted
the whole simulation is lost. It is safer to just merge files afterwards with for example\\
\texttt{ncrcat output1.nc output2.nc output.nc}
%%%%%%%%%%%%%%%%%%%%%%%%%%%%%%%%%%%%%%%%%%%%%%%%%%%%%%%%%%%%%%%%%%%%%%%%%
\section{Diagnostics}\label{sec:diagnostics}
\subsection{Program and files}
We have the program \texttt{feltor/diag/feltordiag.cu}.
This program reads a previously generated simulation file \texttt{input.nc} described in Section~\ref{sec:output_file} and writes into a second output file \texttt{output.nc} described as follows. \\
Compilation\\
\texttt{make feltordiag} \\
Usage \\
\texttt{./feltordiag input.nc output.nc} \\

Output file format: netcdf-4/hdf5; A coordinate variable (Coord. Var.) is a Dataset with the same name as a dimension.

\begin{longtable}{lll>{\RaggedRight}p{7cm}}
\toprule
\rowcolor{gray!50}\textbf{Name} &  \textbf{Type} & \textbf{Dimension} & \textbf{Description}  \\ \midrule
inputfile  &     text attribute & - & verbose input file as a string (valid JSON, no comments) \\
geomfile   &     text attribute & - & verbose geometry input file as a string (valid JSON, no comments) \\
x                & Coord. Var. & 1 (x) & $R$-coordinate (computational space, compressed size: $nN_x/c_x$)\\
y                & Coord. Var. & 1 (y) & $Z$-coordinate (computational space, compressed size: $nN_y/c_y$)\\
z                & Coord. Var. & 1 (z) & $\varphi$-coordinate (computational space, compressed size: $N_z$)\\
psi              & Coord. Var. & 1 (psi) & $\psi_p$-coordinate ( size: $3\cdot 50$) \\
time             & Coord. Var. & 1 (time)& time at which fields are written (variable size: maxout$+1$, dimension size: unlimited) \\
xc           & Dataset & 3 (z,y,x) & Cartesian x-coordinate $x=R\sin(\varphi)$ \\
yc           & Dataset & 3 (z,y,x) & Cartesian y-coordinate $y=R\cos(\varphi)$\\
zc           & Dataset & 3 (z,y,x) & Cartesian z-coordinate $z=Z$ \\
q-profile        & Dataset & 1 (psi) & The safety factor $q(\psi_p)$ \eqref{eq:safety_factor} \\
psi\_psi         & Dataset & 1 (psi) & explicit $\psi_p$ values; Same as psi \\
rho              & Dataset & 1 (psi) & Transformed flux label $\rho:= (\psi_{p,\min} - \psi_p)/\psi_{p,\min}$ \\
psi\_area        & Dataset & 1 (psi) & The area of the flux surfaces $A(\psi_p) = 2\pi \int_\Omega |\nabla\psi_p| \delta(\psi_p - \psi_{p0}) H(Z-Z_X) R\d R\d Z$ \\
psi\_vol         & Dataset & 1 (psi) & The volume enclosed by the flux surfaces $v(\psi_p) = \int_{\psi_p} \dV $ \\
dvdpsip          & Dataset & 1 (psi) & $\d v/\d\psi_p$ \\
Z\_fluc2d        & Dataset & 3 (time,y,x) & Fluctuation level on selected plane ($\varphi= 0$) $\delta Z := Z(R,Z,0) - \langle Z\rangle_{\psi_{p}}$ \\
Z\_fsa2d         & Dataset & 3 (time, y,x) & Flux surface average $\langle Z\rangle_{\psi_p}$ interpolated onto 2d plane Eq.~\eqref{eq:fsa_vol} \\
Z\_fsa           & Dataset & 2 (time, psi) & Flux surface average $\langle Z\rangle_{\psi_p}$ Eq.~\eqref{eq:fsa_vol} \\
Z\_ifs           & Dataset & 2 (time, psi) & Volume integrated flux surface average $\int\d v\langle Z\rangle_{\psi_p}$ unless Z is a current, then it is the volume derived flux-surface average $\partial_v \langle Z\rangle_{\psi_p}$ \\
Z\_ifs\_lcfs     & Dataset & 1 (time) & Volume integrated flux surface average evaluated on last closed flux surface $\int_0^{v(0)}\d v\langle Z\rangle_{\psi_p}$ unless Z is a current, then it is the fsa evaluated $\langle j_v\rangle_{\psi_p}(0)$ \\
Z\_ifs\_norm     & Dataset & 1 (time) & Volume integrated square flux surface average, unless Z is a current, then it is the square derivative of the flux surface average \\
\bottomrule
\end{longtable}
where Z $\in$ \{X, Y\_tt\}
Note that feltoridag converts all $jsX$ quantities into $jvX$
by multiplying $\d v/\d \psi_p$
in the sense that $\vec j\cdot \nabla v  = \vec j \cdot \nabla \psi_p \d v/\d\psi_p$.




%..................................................................
\bibliography{../../doc/related_pages/references}
%..................................................................


\end{document}
