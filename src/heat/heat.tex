%%%%%%%%%%%%%%%%%%%%%definitions%%%%%%%%%%%%%%%%%%%%%%%%%%%%%%%%%%%%%%%

%\documentclass[12pt]{article}
%\documentclass[12pt]{scrartcl}
\documentclass{hitec} % contained in texlive-latex-extra
\settextfraction{0.9} % indent text
\usepackage{csquotes}
\usepackage[hidelinks]{hyperref} % doi links are short and usefull?
\hypersetup{%
    colorlinks=true,
    linkcolor=blue,
    urlcolor=magenta
}
\urlstyle{rm}
\usepackage[english]{babel}
\usepackage{mathtools} % loads and extends amsmath
\usepackage{amssymb}
% packages not used
%\usepackage{graphicx}
%\usepackage{amsthm}
%\usepackage{subfig}
\usepackage{bm}
\usepackage{longtable}
\usepackage{booktabs}
\usepackage{ragged2e} % maybe use \RaggedRight for tables and literature?
\usepackage[table]{xcolor} % for alternating colors
%\rowcolors{2}{gray!25}{white} %%% Use this line in front of longtable
\renewcommand\arraystretch{1.3}
\usepackage[most]{tcolorbox}
\usepackage{doi}
\usepackage[sort,square,numbers]{natbib}
\bibliographystyle{abbrvnat}
%%% reset bibliography distances %%%
\let\oldthebibliography\thebibliography
\let\endoldthebibliography\endthebibliography
\renewenvironment{thebibliography}[1]{
  \begin{oldthebibliography}{#1}
    \RaggedRight % remove if justification is desired
    \setlength{\itemsep}{0em}
    \setlength{\parskip}{0em}
}
{
  \end{oldthebibliography}
}
%%% --- %%%

\definecolor{light-gray}{gray}{0.95}
\newcommand{\code}[1]{\colorbox{light-gray}{\texttt{#1}}}
\newcommand{\eps}{\varepsilon}
\renewcommand{\d}{\mathrm{d}}
\renewcommand{\vec}[1]{{\boldsymbol{#1}}}
\newcommand{\dx}{\,\mathrm{d}x}
%\newcommand{\dA}{\,\mathrm{d}(x,y)}
%\newcommand{\dV}{\mathrm{d}^3{x}\,}
\newcommand{\dA}{\,\mathrm{dA}}
\newcommand{\dV}{\mathrm{dV}\,}

\newcommand{\Eins}{\mathbf{1}}

\newcommand{\ExB}{$\bm{E}\times\bm{B} \,$}
\newcommand{\GKI}{\int d^6 \bm{Z} \BSP}
\newcommand{\GKIV}{\int dv_{\|} d \mu d \theta \BSP}
\newcommand{\BSP}{B_{\|}^*}
\newcommand{\Abar}{\langle A_\parallel \rangle}
%Averages
\newcommand{\RA}[1]{\left \langle #1 \right \rangle} %Reynolds (flux-surface) average
\newcommand{\RF}[1]{\widetilde{#1}} %Reynolds fluctuation
\newcommand{\FA}[1]{\left[\left[ #1 \right]\right]} %Favre average
\newcommand{\FF}[1]{\widehat{#1}} %Favre fluctuation
\newcommand{\PA}[1]{\left \langle #1 \right\rangle_\varphi} %Phi average

%Vectors
\newcommand{\ahat}{\bm{\hat{a}}}
\newcommand{\bhat}{\bm{\hat{b}}}
\newcommand{\chat}{\bm{\hat{c}}}
\newcommand{\ehat}{\bm{\hat{e}}}
\newcommand{\bbar}{\overline{\bm{b}}}
\newcommand{\xhat}{\bm{\hat{x}}}
\newcommand{\yhat}{\bm{\hat{y}}}
\newcommand{\zhat}{\bm{\hat{z}}}

\newcommand{\Xbar}{\bar{\vec{X}}}
\newcommand{\phat}{\bm{\hat{\perp}}}
\newcommand{\that}{\bm{\hat{\theta}}}

\newcommand{\eI}{\bm{\hat{e}}_1}
\newcommand{\eII}{\bm{\hat{e}}_2}
\newcommand{\ud}{\mathrm{d}}

%Derivatives etc.
\newcommand{\pfrac}[2]{\frac{\partial#1}{\partial#2}}
\newcommand{\ffrac}[2]{\frac{\delta#1}{\delta#2}}
\newcommand{\fixd}[1]{\Big{\arrowvert}_{#1}}
\newcommand{\curl}[1]{\nabla \times #1}

\newcommand{\np}{\vec{\nabla}_{\perp}}
\newcommand{\npc}{\nabla_{\perp} \cdot }
\newcommand{\nc}{\vec\nabla\cdot}
\newcommand{\cn}{\cdot\vec\nabla}
\newcommand{\vn}{\vec{\nabla}}
\newcommand{\npar}{\nabla_\parallel}

\newcommand{\GAI}{\Gamma_{1}^{\dagger}}
\newcommand{\GAII}{\Gamma_{1}^{\dagger -1}}
\newcommand{\T}{\mathrm{T}}
\newcommand{\Tp}{\mathcal T^+_{\Delta\varphi}}
\newcommand{\Tm}{\mathcal T^-_{\Delta\varphi}}
\newcommand{\Tpm}{\mathcal T^\pm_{\Delta\varphi}}
%%%%%%%%Some useful abbreviations %%%%%%%%%%%%%%%%
\def\feltor{{\textsc{Feltor }}}

\def\fixme#1{\typeout{FIXME in page \thepage :{#1}}%
 \textsc{\color{red}[{#1}]}}



%%%%%%%%%%%%%%%%%%%%%%%%%%%%%DOCUMENT%%%%%%%%%%%%%%%%%%%%%%%%%%%%%%%%%%%%%%%
\begin{document}

\title{The heat diffusion project}
\author{ M.~Wiesenberger and M.~Held}
\maketitle

\begin{abstract}
  This is a program for the 3d heat diffusion in arbitrary magnetic field
  geometry.
  \end{abstract}

\section{Equations}
The temperature $T$ follows a non-isotropic heat diffusion equation
\begin{align}
\frac{\partial T}{\partial t} = \nu_\parallel\Delta\parallel T
\label{eq:temperature}
\end{align}
where $\nabla_\parallel = \bhat \cdot\nabla$ and
$\bhat$ is the prescribed magnetic field unit vector.
$\nu_\parallel$ is the respective conduction
coefficient parallel to this field.


\subsection{Initialization}
Initialization of $T$ is a Gaussian in the $\varphi =0$ plane.
\begin{align}
    T(x,y,0) = 1 + A\exp\left( -\frac{(x-X)^2 + (y-Y)^2}{2\sigma^2}\right)
    \label{}
\end{align}
where $X = p_x l_x$ and $Y=p_yl_y$ are the initial centre of mass position coordinates, $A$ is the amplitude and $\sigma$ the
radius of the blob.
We then use the fieldline projection to initialize the blob on other planes.
\subsection{Diagnostics}
Integrating Eq.~\eqref{eq:temperature} over the volume yields
\begin{align}
 \frac{\partial}{\partial t} \int T \dV = \nu_\parallel \int \dA\cdot \bhat \nabla_\parallel T
\label{}
\end{align}
\section{Numerical methods}
discontinuous Galerkin on structured grid
\rowcolors{2}{gray!25}{white} %%% Use this line in front of longtable
\begin{longtable}{ll>{\RaggedRight}p{7cm}}
\toprule
\rowcolor{gray!50}\textbf{Term} &  \textbf{Method} & \textbf{Description}  \\ \midrule
coordinate system & Cylindrical 3D & equidistant discretization of $[R_0-a,R_0+a] \times [-a,a] \times [0,2\pi]$, equal number of Gaussian nodes in R and Z, one node in $\varphi$ \\
$\Delta_\parallel$ & FCI & forward-backward symmetric sym, See the documentation on DS\\
time &  Adaptive embedded ARK-4-2-3 & $3rd$ order accurate in both
explicit and implicit parts\\
\bottomrule
\end{longtable}

\section{Compilation and useage}
There are two programs heat.cu and heat\_hpc.cu . Compilation with
\begin{verbatim}
make device = <omp or gpu>
\end{verbatim}
Run with
\begin{verbatim}
path/to/feltor/src/heat/heat input.json geometry_params.json
path/to/feltor/src/heat/heat_hpc input.json geometry_params.json output.nc
\end{verbatim}
All programs write performance informations to std::cout.
The first opens a terminal window with life simulation results
the
other writes the results to disc. Both programs run on shared memory
systems.
For the hpc code, there is another mode:
\begin{verbatim}
path/to/feltor/src/heat/heat_hpc input.json geometry_params.json output.nc input.nc
\end{verbatim}
In this case the temperature field of \code{input.nc} is read
and taken as a reference solution.


\subsection{Input file structure}
Input file format: json

%%This is a booktabs table
\begin{longtable}{llll>{\RaggedRight}p{7cm}}
\toprule
\rowcolor{gray!50}\textbf{Name} &  \textbf{Type} & \textbf{Example} & \textbf{Default} & \textbf{Description}  \\ \midrule
  n      & integer & 3 & - &\# Gaussian nodes in x and y \\
  Nx     & integer &20& - &\# grid points in x \\
  Ny     & integer &40& - &\# grid points in y \\
  Nz     & integer &60& - &\# grid points in z \\
  dt     & integer &0.001& - &initial time step\\
  n\_out  & integer &3  & - &\# Gaussian nodes in x and y in output \\
  Nx\_out & integer &20& - &\# grid points in x in output fields \\
  Ny\_out & integer &40& - &\# grid points in y in output fields \\
  Nz\_out & integer &60& - &\# grid points in z in output fields \\
  itstp  & integer &2  & - &   steps between outputs \\
  maxout & integer &100& - &      \# outputs excluding first \\
  eps\_time  & float &1e-10   & - & accuracy of implicit time-stepper \\
  nu\_parallel    & float &100   & - & parallel viscosity $\nu_\parallel$ \\
  amplitude  & float &0.1    & - & amplitude $A$ of the blob \\
  sigma      & float &5     & - & blob radius $\sigma$ \\
  sigma\_z   & float &0.025  & - & variance in z in units of $R_0$ \\
  posX       & float &0.8    & - & blob x-position in units of $a$\\
  posY       & float &0    & - & blob y-position in units of $a$\\
  bcx   & char & "NEU"      & - & boundary condition in x (one of PER, DIR, NEU, DIR\_NEU or NEU\_DIR) \\
  bcy   & char & "NEU"      & - & boundary condition in y (one of PER, DIR, NEU, DIR\_NEU or NEU\_DIR) \\
  boxscaleRp & float &  1.05, & - & (a little larger than 1) \\
  boxscaleRm & float &  1.05, & - & (a little larger than 1) \\
  boxscaleZp & float &  1.05, & - & (a little larger than 1) \\
  boxscaleZm & float &  1.15, & - & (a little larger than 1) \\
  diff       & char & non-adjoint & non-adjoint & adjoint, non-adjoint, elliptic (discretization for $\Delta_\parallel$) \\
\bottomrule
\end{longtable}

The default value is taken if the value name is not found in the input file. If there is no default and
the value is not found,
the program exits with an error message.

\subsection{Geometry file structure}
File format: json

%%This is a booktabs table
\begin{longtable}{llll>{\RaggedRight}p{7cm}}
\toprule
\rowcolor{gray!50}\textbf{Name} &  \textbf{Type} & \textbf{Example} & \textbf{Default} & \textbf{Description}  \\ \midrule
    A      & float & 1 &  - & Solovev parameter \\
    R\_0   & float & - & -  & Major radius in $\rho_s$ \\
    C      & float[12] &  - & - & Solovev coefficients \\
    elongation & float & 1 & - & Elongation \\
    triangularity & float & 0 & - & Triangularity \\
    inverseaspectratio & float & 0.16667 & - & $a/R_0$ \\
    alpha  & float & 0.02 & - & Width of Heaviside profile in Eq.~\eqref{eq:source_profile} \\
    psip\_min & float & -6 & - & $\psi_{p,min}$ in Heaviside profile Eq.~\eqref{eq:source_profile} \\
    psip\_max & float & 0 & - & $\psi_{p,max}$ in density profile Eq.~\eqref{eq:density_profile} \\
    psip\_max\_lim & float & 1e10 & - & $\psi_p$ for limiter in DS \\
    rk4eps & float & 0.01 & - & Accuracy of fieldline integration in DS \\
\bottomrule
\end{longtable}

\subsection{Structure of output file}
Output file format: netcdf-4/hdf5
%
%Name | Type | Dimensionality | Description
%---|---|---|---|
\begin{longtable}{lll>{\RaggedRight}p{7cm}}
\toprule
\rowcolor{gray!50}\textbf{Name} &  \textbf{Type} & \textbf{Dimension} & \textbf{Description}  \\ \midrule
inputfile  &     text attribute & 1 & verbose input file as a string \\
energy\_time     & Dataset & 1 & timesteps at which 1d variables are written \\
time             & Dataset & 1 & time at which temperature is written \\
x                & Dataset & 1 & x-coordinate  \\
y                & Dataset & 1 & y-coordinate \\
z                & Dataset & 1 & z-coordinate \\
T                & Dataset & 4 (time, z, y, x) & temperature $T$ \\
energy           & Dataset & 1 (energy\_time) & total energy integral  \\
mass             & Dataset & 1 (energy\_time) & mass integral   \\
dEdt             & Dataset & 1 (energy\_time) & change of energy per time  \\
dissipation      & Dataset & 1 (energy\_time) & diffusion integrals  \\
accuracy         & Dataset & 1 (energy\_time) & accuracy in time  \\
error            & Dataset & 1 (energy\_time) & relative distance to temperature field at time 0\\
relerror         & Dataset & 1 (energy\_time) & relative difference to reference solution if one is provide, else 0 \\
\bottomrule
\end{longtable}

%..................................................................
\bibliography{../../doc/related_pages/references}
%..................................................................


\end{document}
