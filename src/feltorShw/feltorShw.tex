%\documentclass[12pt]{article}
%\documentclass[12pt]{scrartcl}
\documentclass{hitec} % contained in texlive-latex-extra 
\settextfraction{0.9} % indent text
\usepackage{csquotes}
\usepackage[hidelinks]{hyperref} % doi links are short and usefull?
\hypersetup{%
    colorlinks=true,
    linkcolor=blue,
    urlcolor=magenta
}
\urlstyle{rm}
\usepackage[english]{babel}
\usepackage{mathtools} % loads and extends amsmath
\usepackage{amssymb}
% packages not used
%\usepackage{graphicx}
%\usepackage{amsthm}
%\usepackage{subfig}
\usepackage{bm}
\usepackage{longtable}
\usepackage{booktabs}
\usepackage{ragged2e} % maybe use \RaggedRight for tables and literature?
\usepackage[table]{xcolor} % for alternating colors
\rowcolors{2}{gray!25}{white}
\renewcommand\arraystretch{1.3}

%%% reset bibliography distances %%%
\let\oldthebibliography\thebibliography
\let\endoldthebibliography\endthebibliography
\renewenvironment{thebibliography}[1]{
  \begin{oldthebibliography}{#1}
    \RaggedRight % remove if justification is desired
    \setlength{\itemsep}{0em}
    \setlength{\parskip}{0em}
}
{
  \end{oldthebibliography}
}
%%% --- %%%

%%%%%%%%%%%%%%%%%%%%%definitions%%%%%%%%%%%%%%%%%%%%%%%%%%%%%%%%%%%%%%%
\newcommand{\eps}{\varepsilon}
\renewcommand{\d}{\mathrm{d}}
\newcommand{\T}{\mathrm{T}}
\renewcommand{\vec}[1]{\boldsymbol{#1}}

\newcommand{\dx}{\,\mathrm{d}x}
%\newcommand{\dA}{\,\mathrm{d}(x,y)}
%\newcommand{\dV}{\mathrm{d}^3{x}\,}
\newcommand{\dA}{\,\mathrm{dA}}
\newcommand{\dV}{\mathrm{dV}\,}
\newcommand{\Eins}{\mathbf{1}}
\newcommand{\ExB}{$\bm{E}\times\bm{B} \,$}
\newcommand{\GKI}{\int d^6 \bm{Z} \BSP}
\newcommand{\GKIV}{\int dv_{\|} d \mu d \theta \BSP}
\newcommand{\BSP}{B_{\|}^*}
\newcommand{\GA}[1]{\langle #1   \rangle}
\newcommand{\Abar}{\langle A_\parallel \rangle}
%Vectors
\newcommand{\bhat}{\bm{\hat{b}}}
\newcommand{\bbar}{\overline{\bm{b}}}
\newcommand{\chat}{\bm{\hat{c}}}
\newcommand{\ahat}{\bm{\hat{a}}}
\newcommand{\xhat}{\bm{\hat{x}}}
\newcommand{\yhat}{\bm{\hat{y}}}
\newcommand{\zhat}{\bm{\hat{z}}}
\newcommand{\Xbar}{\bar{\vec{X}}}
\newcommand{\phat}{\bm{\hat{\perp}}}
\newcommand{\that}{\bm{\hat{\theta}}}
\newcommand{\eI}{\bm{\hat{e}}_1}
\newcommand{\eII}{\bm{\hat{e}}_2}
\newcommand{\ud}{\mathrm{d}}
%Derivatives etc.
\newcommand{\pfrac}[2]{\frac{\partial#1}{\partial#2}}
\newcommand{\ffrac}[2]{\frac{\delta#1}{\delta#2}}
\newcommand{\fixd}[1]{\Big{\arrowvert}_{#1}}
\newcommand{\curl}[1]{\nabla \times #1}
\newcommand{\np}{\nabla_{\perp}}
\newcommand{\npc}{\nabla_{\perp} \cdot }
\newcommand{\nc}{\nabla\cdot }
\newcommand{\GAI}{\Gamma_{1}^{\dagger}}
\newcommand{\GAII}{\Gamma_{1}^{\dagger -1}}

%%%%%%%%%%%%%%%%%%%%%%%%%%%%%DOCUMENT%%%%%%%%%%%%%%%%%%%%%%%%%%%%%%%%%%%%%%%
\begin{document}

\title{The feltorShw project}
\maketitle

\begin{abstract}
This is a program for 2d isothermal electrostatic \(\delta\)f and full-F simulations for resistive drift-wave turbulence.
\end{abstract}
\section{Equations}
\subsection{\(\delta\)f Hasegawa-Wakatani models}
The continuity equations for relative electron density fluctuations \(\delta n_e\) and relative ion gyro-centre density fluctuations  \(\delta N_i\) and the \(\delta\)f polarisation equation read:
\begin{align}
 \frac{\partial}{\partial t }\delta n_e + \frac{1}{B_0} \left[\phi,\delta n_e\right]_{\perp}  +\frac{1}{L_n B_0}  \frac{\partial }{\partial y} \phi  &= \Lambda_\delta + \Lambda_{\nu_\perp,\delta n_e}
,\\
 \frac{\partial}{\partial t }\delta N_i +\frac{1}{B_0} \left[\Gamma_{i,1}\phi,\delta N_i \right]_{\perp}   +\frac{1}{L_n B_0}  \frac{\partial }{\partial y}\phi  &= \Lambda_{\nu_\perp,\delta N_i},\\
 \vec{\nabla} \cdot \left(\frac{1}{\Omega_{i,0}} \frac{\vec{\nabla}_\perp \phi}{B_0}\right)= \delta n_e - \Gamma_{i,1}\delta N_i,
\end{align}
Note that polarisation effects are taken in the long wavelength limit in order to be consistent with the full-F model.
\subsection{Full-f Hasegawa-Wakatani models}
The continuity equations for electron density \(n_e\) and ion gyro-centre density \(N_i\) and the polarisation equatio read:
\begin{align}
 \frac{\partial}{\partial t }n_e  + \frac{1}{B_0} \left[\phi, n_e\right]_{\perp} &=\Lambda + \Lambda_{\nu_\perp, n_e} +  \Lambda_{S, n_e}
,\\
\frac{\partial}{\partial t } N_i+\frac{1}{B_0} \left[\psi_i, N_i\right]_{\perp} &=\Lambda_{\nu_\perp, N_i} + \Lambda_{S, N_i} , \\
 \vec{\nabla} \cdot \left(\frac{N_i}{\Omega_{i,0}} \frac{\vec{\nabla}_\perp \phi}{B_0}\right)= n_e - \Gamma_{i,1} N_i, 
 \end{align}
with generalized potential, \(\vec{E} \times \vec{B}\) drift velocity and Pade approximated gyro-average operator
\begin{align}
  \psi_i&:= \Gamma_{1,i} \phi - \frac{m u_E^2}{2 q}, \\
  \vec{u}_E &:= \frac{1}{B} \vec{\hat{b}} \times \vec{\nabla} \phi \\
  \Gamma_{1,i} f&:= \frac{1}{1-\frac{\rho_i^2}{2}\vec{\nabla}_\perp^2} f. 
\end{align}
Note that the \(\delta\)f densities are related to the full-F quantities according to \(n_e = n_G (1+\delta n_e)\), etc.
\subsubsection{Thin layer approximation}
The thin layer approximation appears in three terms in the full-F model
\begin{align}
 \Lambda &\approx n_e \Lambda_\delta \\ 
 \psi_i &\approx \Gamma_{i,1}\phi \\
 \vec{\nabla} \cdot \left(\frac{N_i}{\Omega_{i,0}} \frac{\vec{\nabla}_\perp \phi}{B_0}\right) &\approx   \frac{n_G}{\Omega_{i,0}} \frac{\vec{\nabla}_\perp^2 \phi}{B_0}
\end{align}

\subsection{\(\delta\)f and full-F Hasegawa-Wakatani closures}
The closures are given in Table~\ref{table:hwmhwtable}.
\begin{table}[!htp]
\centering
 \caption{Ordinary and modified HW closures for \(\delta\)f and full-F models.}
 \label{table:hwmhwtable}
\begin{tabular}{ l  l  l }
   & ordinary HW &  modified HW \\
%      \vspace{+1 mm}
     \hline			
  \(\Lambda_\delta/(\alpha_\delta \Omega_{i,0})\) & \( \frac{e}{t_{e0}}\phi-\delta n_e \) & \(  \frac{e}{t_{e0}}\widetilde{\phi}- \widetilde{ \delta n_e}\)\\
 \({\Lambda}/(\alpha n_{e,0} \Omega_{i,0})\) & \( \frac{e}{t_{e0}}\phi -\ln\left(n_e/\langle{  n_e }\rangle\right)  \)          & \( \frac{e}{t_{e0}} \widetilde{\phi}-\widetilde{\ln\left(n_e\right)}\)  \\
\end{tabular}
\end{table}
Here, we introduced the delta-f and full-F adiabaticity parameters according to
\begin{align}
\alpha:= \frac{ t_{e}  k_\parallel^2 }{ \eta_\parallel e^2   n_{e,0} \Omega_{i,0} } \\
 \alpha_\delta := \frac{t_e k_\parallel^2 }{0.51 m_e \nu_{e0} \Omega_{i,0}} 
\end{align}
\subsection{Dissipative terms}
\subsubsection{\(\delta\)f model}
\begin{align}\label{eq:perpdiffn}
 \Lambda_{\nu_\perp,\delta n_e} &=  -\nu_\perp \vec{\nabla}_\perp^4 \delta n_e, &
 \Lambda_{\nu_\perp,\delta N_i} &=  -\nu_\perp \vec{\nabla}_\perp^4 \delta  N_i.
\end{align}
\subsubsection{Full-F model}
\begin{align}\label{eq:perpdiffn}
 \Lambda_{\nu_\perp,n_e} &=  -\nu_\perp \vec{\nabla}_\perp^4 n_e, &
 \Lambda_{\nu_\perp,N_i} &=  -\nu_\perp \vec{\nabla}_\perp^4 N_i.
\end{align}
\subsection{Energy theorem}
\subsubsection{\(\delta\)f model}
\begin{eqnarray}
 \frac{\partial}{\partial t}  \mathcal{E}_{\delta } = \mathcal{D}_{L_n}+ \mathcal{D}_{\parallel,\delta } + \mathcal{D}_{\nu_\perp,\delta }
\end{eqnarray}
with the energy
\begin{eqnarray}
%  \qquad
 \mathcal{E}_{\delta f} =  \int dA n_{e0}\left[\frac{t_e}{2} \delta{n}_e^2  +  \frac{T_i}{2} \delta{N}_i^2    + \frac{1}{2} m_i  u_E^2\right]
%  \mathcal{E}_{\delta f} =  \int dA n_{e0}\left[\frac{t_e}{2} \tilde{n}_e^2   + \frac{1}{2} m_i  u_E^2\right]
\end{eqnarray}
and 
\begin{align}
 \mathcal{D}_{L_n}                  &= - \frac{1}{L_n}\int dA \left( t_e \delta{n}_e \frac{\partial \phi }{\partial y}  - T_i \delta{N}_i \frac{\partial \Gamma_{i,1} \phi }{\partial y} \right) \\
\mathcal{D}_{\parallel,\delta } &=  \int dA \left(t_e \delta{n}_e - e \phi \right) \Lambda_\delta \\
\mathcal{D}_{\nu_\perp,\delta } &=  \int dA \left[ \left(t_e \delta{n}_e - e \phi \right) \Lambda_{\nu_\perp,\delta n_e} +  \left(T_i \delta{N}_i + e \Gamma_{1,i} \phi \right) \Lambda_{\nu_\perp,\delta N_i}\right]
\end{align}

\subsubsection{Full-F model}
The energytheorem reads
\begin{eqnarray}
 \frac{\partial}{\partial t}  \mathcal{E} = \mathcal{D}_\parallel + \mathcal{D}_{\nu_\perp} + \mathcal{D}_{S}
\end{eqnarray}
with energy density
\begin{eqnarray}
 \mathcal{E} = \int dA \left[ n_e t_e   \ln{\left(n_e\right)} + N_i T_i   \ln{\left(N_i\right)}+ \frac{1}{2} m_i N_i u_E^2\right]
\end{eqnarray}
and source term
\begin{align}
\mathcal{D}_\parallel  &= \int dA \left(t_e \left(1+\ln{n_e}\right) -e \phi\right)  \Lambda \\
\mathcal{D}_{\nu_\perp} &= \int dA \left[t_{e} (1+\ln (n_e))  - e \phi \right]\Lambda_{\nu_\perp,n_e} +\left[T_{i}  (1+\ln (N_i))+ e \psi_i \right] \Lambda_{\nu_\perp,N_i} \\
\mathcal{D}_{S} &= \int dA \left[t_{e} (1+\ln (n_e))  - e \phi \right]\Lambda_{S,n_e} +\left[T_{i}  (1+\ln (N_i))+ e \psi_i \right] \Lambda_{S,N_i} 
\end{align}
\subsection{Sources and Sinks}
We choose particle sources, which do not contribute to the polarisation charge density. They read
\begin{align}
 \Lambda_{S,n_e} :=\omega_S z \hspace{0.5mm} \Theta \left(z\right)  \\
 \Lambda_{S,N_i} := \Gamma_{i,1}^{-1} \Lambda_{S,n_e}
\end{align}
where we defined the  Heaviside function \(\Theta(z)\) and  
 \begin{align}
  z&:=g(x) \left[n_G - \langle n_e \rangle\right] \\
  g(x) &:= \left[1-\tanh{(x-x_b)/\sigma_b}\right]/2 
 \end{align}
to maintain the initial profile in a small region \(x \in \left[0,x_b\right]\).
\subsection{Initialization}
We initialize our fields with vanishing initial electric potential. Thus in the \(\delta f\) cas we use
\begin{align}
  \delta n_e  =\Gamma_{1,i} \delta  N_i,
\end{align}
whereas in the full-F case we use
\begin{align}
  n_e  =\Gamma_{1,i}^\dagger N_i.
\end{align}
\subsubsection{Blob}
\begin{align}
N_{i}\left(\vec{x},0\right) =   n_G\left[1+A \exp{\left(-\frac{\left(\vec{x}-\vec{x}_0\right)^2}{2\sigma^2}\right)}\right]
\end{align}
\subsubsection{Wave}
\begin{align}
N_{i}\left(\vec{x},0\right) =   n_G\left[1+A \sin{(k_x x)} \cos{(k_y y)}\right]  
\end{align}
\subsubsection{Bath}
\begin{align}
N_{i}\left(\vec{x},0\right) =   n_G\left[1+\delta N_{bath}(\vec{x})\right]  
\end{align}
see further feltor dG docu
\section{Numerical methods}
discontinuous Galerkin on structured grid 
\begin{longtable}{ll>{\RaggedRight}p{7cm}}
\toprule
\rowcolor{gray!50}\textbf{Term} &  \textbf{Method} & \textbf{Description}  \\ \midrule
coordinate system & cartesian 2D & equidistant discretization of $[0,l_x] \times [0,l_y]$, equal number of Gaussian nodes in x and y \\
matrix inversions & conjugate gradient & Use previous two solutions to extrapolate initial guess and $1/\chi$ as preconditioner \\
\ExB advection & Poisson & \\
time &  Karniadakis multistep & $3rd$ order explicit, diffusion $2nd$ order implicit \\
\bottomrule
\end{longtable}
\subsection{Input file structure}
Input file format: json

%%This is a booktabs table
\begin{longtable}{llll>{\RaggedRight}p{7cm}}
\toprule
\rowcolor{gray!50}\textbf{Name} &  \textbf{Type} & \textbf{Example} & \textbf{Default} & \textbf{Description}  \\ \midrule
n      & integer & 3 & - &\# Gaussian nodes in x and y \\
Nx     & integer &64& - &\# grid points in x \\
Ny     & integer &64& - &\# grid points in y \\
dt     & integer &0.1& - &time step in units of $c_s/\rho_s$ \\
n\_out  & integer &1  & - &\# Gaussian nodes in x and y in output \\
Nx\_out & integer &64& - &\# grid points in x in output fields \\
Ny\_out & integer &64& - &\# grid points in y in output fields \\
itstp  & integer &10  & - &   steps between outputs \\
maxout & integer &1000& - &      \# outputs excluding first \\
eps\_pol   & float &1e-6   & - &  accuracy of polarisation solver \\
jumpfactor & float &0.1    & - &  jumpfactor $\in \left[0.01,1\right]$\ \\
eps\_gamma & float &1e-14  & - & accuracy of $\Gamma_1$ (only in gyrofluid model) \\
eps\_time  & float &1e-14  & - & accuracy of implicit time-stepper \\
tau        & float &1      & - & $\tau = T_i/T_e$ (only in gyrofluid models) \\
modelmode  & integer &0   & - & Full-F/Full-F-bousinesq/delta-f/Full-F with "df notation and dissipation" \\
cmode      & integer &0   & - & dyn/const collisionality (0/1) \\
hwmode     & integer &0   & - & ordinary/modified HW (0/1) \\
nu\_perp   & float &1e-3  & - & pependicular viscosity $\nu$ \\
alpha      & float &0.005  & - & adiabaticity $\alpha$ or $\alpha_\delta$ \\
initmode   & integer &0    & - & (0) blob, (1) Wave in x and y (2) bath\\
amplitude  & float &1.0    & - & amplitude $A$ of the blob \\
sigma      & float &10     & - & blob radius $\sigma$ \\
posX       & float &0.3    & - & blob x-position in units of $l_x$\\
posY       & float &0.5    & - & blob y-position in units of $l_y$ \\
prof\_amp    & float &1   & - &Profile amplitude \\
bg\_prof\_amp& float &0   & - &Background Prof amplitude  \\
invkappa   & float &64   & - &gradient length $L_n$ (in units of $\rho_s$)   \\
lx         & float &64   & - & $l_x$  \\
ly         & float &64  & - & $l_y$  \\
bc\_x   & char & "DIR\_NEU"      & - & boundary condition in x (one of DIR, NEU, DIR\_NEU or NEU\_DIR) \\
bc\_x\_phi   & char & "DIR\_NEU"      & - & boundary condition for electric potential $\phi$ in x (one of  DIR, DIR\_NEU or NEU\_DIR) \\
bc\_y   & char & "PER"      & - & boundary condition in y (typically PER) \\
prof\_source\_rate     & float &0.1  & - & profile source rate in units $c_s/\rho_s$ \\
source\_b             & float &0.2  & - & source dampingb in u of lx (<1 no Source)  \\
source\_damping\_width & float &0.5  & - & source damping width  \\
\bottomrule
\end{longtable}

The default value is taken if the value name is not found in the input file. If there is no default and
the value is not found,
the program exits with an error message.
%..................................................................
\begin{thebibliography}{1}
 \bibitem{held:phd17}
  M. Held, Full-F gyro-fluid modelling of the tokamak edge and scrape-off, Universit{\"a}t Innsbruck (2017)
\end{thebibliography}
%..................................................................


\end{document}
