%%%%%%%%%%%%%%%%%%%%%definitions%%%%%%%%%%%%%%%%%%%%%%%%%%%%%%%%%%%%%%%
%\documentclass[12pt]{article}
%\documentclass[12pt]{scrartcl}
\documentclass{hitec} % contained in texlive-latex-extra
\settextfraction{0.9} % indent text
\usepackage{csquotes}
\usepackage[hidelinks]{hyperref} % doi links are short and usefull?
\hypersetup{%
    colorlinks=true,
    linkcolor=blue,
    urlcolor=magenta
}
\urlstyle{rm}
\usepackage[english]{babel}
\usepackage{mathtools} % loads and extends amsmath
\usepackage{amssymb}
% packages not used
%\usepackage{graphicx}
%\usepackage{amsthm}
%\usepackage{subfig}
\usepackage{bm}
\usepackage{longtable}
\usepackage{booktabs}
\usepackage{ragged2e} % maybe use \RaggedRight for tables and literature?
\usepackage[table]{xcolor} % for alternating colors
%\rowcolors{2}{gray!25}{white} %%% Use this line in front of longtable
\renewcommand\arraystretch{1.3}
\usepackage[most]{tcolorbox}
\usepackage{doi}
\usepackage[sort,square,numbers]{natbib}
\bibliographystyle{abbrvnat}
%%% reset bibliography distances %%%
\let\oldthebibliography\thebibliography
\let\endoldthebibliography\endthebibliography
\renewenvironment{thebibliography}[1]{
  \begin{oldthebibliography}{#1}
    \RaggedRight % remove if justification is desired
    \setlength{\itemsep}{0em}
    \setlength{\parskip}{0em}
}
{
  \end{oldthebibliography}
}
%%% --- %%%

\definecolor{light-gray}{gray}{0.95}
\newcommand{\code}[1]{\colorbox{light-gray}{\texttt{#1}}}
\newcommand{\eps}{\varepsilon}
\renewcommand{\d}{\mathrm{d}}
\renewcommand{\vec}[1]{{\boldsymbol{#1}}}
\newcommand{\dx}{\,\mathrm{d}x}
%\newcommand{\dA}{\,\mathrm{d}(x,y)}
%\newcommand{\dV}{\mathrm{d}^3{x}\,}
\newcommand{\dA}{\,\mathrm{dA}}
\newcommand{\dV}{\mathrm{dV}\,}

\newcommand{\Eins}{\mathbf{1}}

\newcommand{\ExB}{$\bm{E}\times\bm{B} \,$}
\newcommand{\GKI}{\int d^6 \bm{Z} \BSP}
\newcommand{\GKIV}{\int dv_{\|} d \mu d \theta \BSP}
\newcommand{\BSP}{B_{\|}^*}
\newcommand{\Abar}{\langle A_\parallel \rangle}
%Averages
\newcommand{\RA}[1]{\left \langle #1 \right \rangle} %Reynolds (flux-surface) average
\newcommand{\RF}[1]{\widetilde{#1}} %Reynolds fluctuation
\newcommand{\FA}[1]{\left[\left[ #1 \right]\right]} %Favre average
\newcommand{\FF}[1]{\widehat{#1}} %Favre fluctuation
\newcommand{\PA}[1]{\left \langle #1 \right\rangle_\varphi} %Phi average

%Vectors
\newcommand{\ahat}{\bm{\hat{a}}}
\newcommand{\bhat}{\bm{\hat{b}}}
\newcommand{\chat}{\bm{\hat{c}}}
\newcommand{\ehat}{\bm{\hat{e}}}
\newcommand{\bbar}{\overline{\bm{b}}}
\newcommand{\xhat}{\bm{\hat{x}}}
\newcommand{\yhat}{\bm{\hat{y}}}
\newcommand{\zhat}{\bm{\hat{z}}}

\newcommand{\Xbar}{\bar{\vec{X}}}
\newcommand{\phat}{\bm{\hat{\perp}}}
\newcommand{\that}{\bm{\hat{\theta}}}

\newcommand{\eI}{\bm{\hat{e}}_1}
\newcommand{\eII}{\bm{\hat{e}}_2}
\newcommand{\ud}{\mathrm{d}}

%Derivatives etc.
\newcommand{\pfrac}[2]{\frac{\partial#1}{\partial#2}}
\newcommand{\ffrac}[2]{\frac{\delta#1}{\delta#2}}
\newcommand{\fixd}[1]{\Big{\arrowvert}_{#1}}
\newcommand{\curl}[1]{\nabla \times #1}

\newcommand{\np}{\vec{\nabla}_{\perp}}
\newcommand{\npc}{\nabla_{\perp} \cdot }
\newcommand{\nc}{\vec\nabla\cdot}
\newcommand{\cn}{\cdot\vec\nabla}
\newcommand{\vn}{\vec{\nabla}}
\newcommand{\npar}{\nabla_\parallel}

\newcommand{\GAI}{\Gamma_{1}^{\dagger}}
\newcommand{\GAII}{\Gamma_{1}^{\dagger -1}}
\newcommand{\T}{\mathrm{T}}
\newcommand{\Tp}{\mathcal T^+_{\Delta\varphi}}
\newcommand{\Tm}{\mathcal T^-_{\Delta\varphi}}
\newcommand{\Tpm}{\mathcal T^\pm_{\Delta\varphi}}
%%%%%%%%Some useful abbreviations %%%%%%%%%%%%%%%%
\def\feltor{{\textsc{Feltor }}}

\def\fixme#1{\typeout{FIXME in page \thepage :{#1}}%
 \textsc{\color{red}[{#1}]}}


\usepackage{minted}

%%%%%%%%%%%%%%%%%%%%%%%%%%%%%DOCUMENT%%%%%%%%%%%%%%%%%%%%%%%%%%%%%%%%%%%%%%%
\begin{document}

\title{The poet project}
\author{M.~Held}
\maketitle

\begin{abstract}
  This is a program for 2d isothermal delta-f and full-f gyro-fluid simulations with different treatments of the polarization charge
\end{abstract}

\section{Equations}
Currently we implemented $5$ slightly different sets of equations. $n$ is the electron density, $N$ is the ion gyrocentre density. $\phi$ is the electric potential. We
use Cartesian coordinates $x$, $y$.
\subsection{Models}
\subsubsection{Delta-F gyro-fluid models}
The two delta-f models ("df-lwl" \& "df-O2") share the same evolution equations\footnote{The densities are for delta-f models are understood as fluctuating quantities with respect to a background} 
\begin{subequations}
\begin{align}
 \frac{\partial n}{\partial t}     &= 
    \{ n, \phi\} 
  + \kappa \frac{\partial \phi}{\partial y} 
  -\kappa \frac{\partial n}{\partial y}
  + \nu \Delta_{\perp} n  \\
  \frac{\partial N}{\partial t} &=
  \{ N, \psi\} 
  + \kappa \frac{\partial \psi}{\partial y} 
  + \tau \kappa\frac{\partial N}{\partial y} +\nu\Delta_{\perp}N \\
   \vec{\nabla}\cdot \vec{P}_2 &=  \Gamma_1 N -n, 
\end{align}
\end{subequations}
and differ only in the polarization equation given in Section~\ref{sec:polarization}:
%%%%%%%%%
\subsubsection{Full-F gyro-fluid models}
Likewise the full-f models ("ff-lwl" \& "ff-O2" \& "ff-O4") share the same evolution equations
\begin{subequations}
\begin{align}
 \frac{\partial n}{\partial t}     &= 
    \frac{1}{B}\{ n, \phi\} 
  + \kappa n\frac{\partial \phi}{\partial y} 
  -\kappa \frac{\partial n}{\partial y}
  + \nu \nabla_\perp^2 n  \\
  \frac{\partial N}{\partial t} &=
  \frac{1}{B}\{ N, \psi\} 
  + \kappa N\frac{\partial \psi}{\partial y} 
  + \tau \kappa\frac{\partial N}{\partial y} +\nu\nabla_\perp^2N \\
   \vec{\nabla}\cdot \vec{P}_2 &= \Gamma_1 N-n
\end{align}
\end{subequations}
and are based upon a ``radially'' varying magnetic field magnitude
\begin{align}
 B(x)^{-1} = \kappa x +1-\kappa X 
\end{align}
The full-f models differ only in the treatment the  polarization terms \(\psi_2\) and \(\vec{P}_2\), given in Sec.~\ref{sec:polarization}
%%%%%
\subsubsection{Polarization and FLR terms}
\label{sec:polarization}
The FLR and polarization operators enter the gyro-fluid potential (\(\psi)\) as well as the polarization density.
The operators and gyro-fluid potential are
\begin{align}
 \Gamma_1 &= ( 1- \tau/2 \Delta_{\perp})^{-1} &
 \Gamma_0 &= ( 1- \tau\Delta_{\perp})^{-1} &
   \psi = \Gamma_1 \phi + \psi_2 
\end{align}
The  (negative) polarization charge is 
\begin{align}
  \vec{\nabla}\cdot \vec{P}_2 &=    
\begin{cases}
-\Delta_{\perp} \phi , 
        &\ \text{if "df-lwl"} \\
-\Gamma_0\Delta_{\perp} \phi, 
      &\ \text{if "df-O2"} \\
-\nabla\cdot \left(\frac{N}{B^2} \nabla_\perp \phi\right)
       &\ \text{if "ff-lwl"} \\
-\sqrt{\Gamma_0}\nabla\cdot \left(\frac{N}{B^2} \nabla_\perp\sqrt{\Gamma_0} \phi\right) 
         &\ \text{if "ff-O2"} \\
-\Gamma_1 \left[\nabla\cdot \left(\frac{N}{B^2} \nabla_\perp\right)-2   \nabla\cdot\nabla\cdot \left(\frac{\tau N}{B^4} \nabla_\perp^2\right)+  \Delta_{\perp} \left(\frac{\tau N}{B^4} \Delta_{\perp}\right)\right]\Gamma_1 \phi 
        &\ \text{if "ff-O4"} \\
\end{cases}
\end{align}
The polarization part of the gyro-fluid potential is
\begin{align}
\psi_2&=    
\begin{cases}
0 , 
        &\ \text{if "df-lwl"} \\
0, 
      &\ \text{if "df-O2"} \\
- \frac{1}{2} \frac{(\nabla\phi)^2}{B^2}
       &\ \text{if "ff-lwl"} \\
- \frac{1}{2} \frac{(\nabla\sqrt{\Gamma_0}\phi)^2}{B^2}
         &\ \text{if "ff-O2"} \\
- \frac{1}{2 B^2} \left[|\nabla_\perp \Gamma_1\phi|^2 + \frac{\tau}{2 B^2} \left[|\nabla_\perp^2 \Gamma_1\phi |^2 - (\Delta_{\perp} \Gamma_1\phi)^2 /2 \right]\right]
        &\ \text{if "ff-O4"} \\
\end{cases}
\end{align}
\subsection{Initialization}
We follow the strategy to enforce the initial fields of the physical variables, the electron density \(n\) and the electric potential \(\phi\), in order to compute the initial ion gyro-center density.
\subsubsection{Non-rotating Gaussian ("blob")}
Initialization of $n$ is a Gaussian 
\begin{align}
    n(x,y,0) &= 1 + A\exp\left( -\frac{(x-X)^2 + (y-Y)^2}{2\sigma^2}\right) \\
    \phi(x,y)&=const.
\end{align}
where $X = p_x l_x$ and $Y=p_yl_y$ are the initial centre of mass position coordinates, $A$ is the amplitude and $\sigma$ the
radius of the blob.
We initialize 
\begin{align}
    N &= \Gamma_1^{-1} n 
\end{align}
\subsubsection{Gaussian with zero polarization charge density("blob")}
Initialization of $n$ is a Gaussian 
\begin{align}
    n(x,y,0) &= 1 + A\exp\left( -\frac{(x-X)^2 + (y-Y)^2}{2\sigma^2}\right) \\
\end{align}
where $X = p_x l_x$ and $Y=p_yl_y$ are the initial centre of mass position coordinates, $A$ is the amplitude and $\sigma$ the radius of the blob. We initialize then
\begin{align}
    N &= n 
\end{align}
so that the total polarization charge vanishes  \(\vec{\nabla}\cdot \vec{P}=0\).

\subsubsection{Double Rotating Gaussian }
We initialize two blobs
\begin{align}
    n(x,y) &= 1 + A\exp\left( -\frac{(x-X)^2 + (y-Y)^2}{2\sigma^2}\right) \\
    \Omega_E:= \vec{\nabla} \cdot \left(\frac{1}{B} \vec{\nabla}_\perp \phi\right) &= \delta \hspace{1mm} (n-1)
\end{align}
where \(\delta = 10\) is a parameter for the rotation.
Thus we have
\begin{align}
  \phi(x,y) &= \Delta_\perp^{-1} \Omega_E 
\end{align}

the ion gyro-center density follows from the respective polarization equation.

\subsubsection{Shear flow (double shear layer)}
\begin{align}
 n(x,y) &= n_{e0}\\
    \Omega_E:= \vec{\nabla} \cdot \left(\frac{1}{B} \vec{\nabla}_\perp \phi\right) &= A
    \begin{cases}
        \delta \cos(2 \pi y/l_y) - \frac{1}{\rho} \text{sech}^2 \left(\frac{2 \pi x/l_x-\pi/2}{\rho}\right),\ x \leq l_x/2 \\
        \delta \cos(2 \pi y/l_y) + \frac{1}{\rho} \text{sech}^2 \left(\frac{3 \pi /2-2 \pi x/l_x}{\rho}\right),\ x > l_x/2 \\
    \end{cases}
%  \phi(x,y) &= \dots \\
\end{align}
where \(\rho=\pi/15\) is the width of the shear layer and \(\delta=0.05\). The A parameter serves to control the magnitude.
\begin{align}
  \phi(x,y) &= \Delta_\perp^{-1} \Omega_E
\end{align}
the ion gyro-center density follows from the respective polarization equation, which requires a non-linear solve \(f(N,\phi) = n_e\). An initial guess for the ion gyro-cente rdensity \(N\) is obtained via the lwl approximated polarization equation.

\subsection{Diagnostics}
Diagnostics are the mass \(M\), the center of mass position \(\vec{X}_{com}\) and the (non-normalized) compactness \(I_c\)
\begin{align}
    M(t) &:= \int dA (n-1)  \\
    M \vec{X}_{com} &:= \int dA \vec{x} (n-1) \\
    I_c(t)  &:= \int dA (n(\vec{x},t)-1) h(\vec{x},t) \\
    h(\vec{x},t) &:= \begin{cases}
          1, 
        &\ \text{if} \hspace{2mm}||\vec{x} - \vec{X}_{max}||^2 < \sigma^2 \\
0,  &\ \text{else} 
           \end{cases}
\end{align}
\section{Numerical methods}
discontinuous Galerkin on structured grid
\rowcolors{2}{gray!25}{white} %%% Use this line in front of longtable
\begin{longtable}{ll>{\RaggedRight}p{7cm}}
\toprule
\rowcolor{gray!50}\textbf{Term} &  \textbf{Method} & \textbf{Description}  \\ \midrule
coordinate system & Cartesian 2D & equidistant discretization of $[0,l_x] \times [0,l_y]$, equal number of Gaussian nodes in x and y \\
matrix inversions & multigrid conjugate gradient &  \\
matrix functions & cauchy integral  method & \\
\ExB advection & centered upwind-scheme\\
curvature terms & centered difference & \\
time &  Karniadakis multistep & $3rd$ order explicit, diffusion $2nd$ order implicit \\
\bottomrule
\end{longtable}

\section{Compilation and useage}
There are two programs poet.cu and poet\_hpc.cu . Compilation with
\begin{verbatim}
make <poet poet_hpc poet_mpi> device = <omp gpu>
\end{verbatim}
Run with
\begin{verbatim}
path/to/feltor/src/poet/poet input.json
path/to/feltor/src/poet/poet_hpc input.json output.nc
echo np_x np_y | mpirun -n np_x*np_y path/to/feltor/src/poet/poet_mpi\
    input.json output.nc
\end{verbatim}
All programs write performance informations to std::cout.
The first is for shared memory systems (OpenMP/GPU) and opens a terminal window with life simulation results.
 The
second can be compiled for both shared and distributed memory systems and uses serial netcdf in both cases
to write results to a file.
For distributed
memory systems (MPI+OpenMP/GPU) the program expects the distribution of processes in the
x and y directions as command line input parameters.

\subsection{Input file structure}
Input file format: json
\begin{minted}[texcomments]{js}
"grid":
{
    "n" :  5,    //Legendre polynomial order in x and y 
    "Nx" : 32,   //grid points in x
    "Ny" : 32,   //grid points in y
    "lx"  : 64,  //Box size in x in units of $\rho_s$
    "ly"  : 64   //Box size in y in units of $\rho_s$
},
"timestepper":
{
    "type": "adaptive", //"adaptive" (explicit adaptive RK) or "multistep" (explicit multistep)
    "tableau": "Bogacki-Shampine-4-2-3", // recommended "Bogacki-Shampine-4-2-3" (default adaptive) or "TVB-3-3" (multistep)
    "rtol": 1e-10, //relative tolerance of adaptive time stepper
    "atol": 1e-12, //absolute tolerance of adaptive time stepper
    "dt" : 0.1   //time step in units of $c_s/\rho_s$ for multistep. Also determines together with itstep output step $dt_{out} = dt itstp$
},
"output":
{
    "type": "glfw",  // output format "glfw" & "netcdf",
    "itstp"  : 1,    //time steps between outputs
    "maxout" : 2,    //\# of netcdf outputs
    "n" : 5,         //Legendre polynomial order in x and y for netcdf output
    "Nx" : 32,       //grid points in x for netcdf output
    "Ny" : 32        //grid points in y for netcdf output
},    
"elliptic":
{
    "stages"     : 3, //\# of stages of the multigrid solver for the (tensor) elliptic operator
    "eps_pol"    : [1e-6,1.0,1.0], //accuracy at each stage of the multigrid solver
    "jumpfactor": 1 //jump factor of the (tensor) elliptic operator
},
"helmholtz":
{
    "eps_gamma1" :   1e-8, //accuracy of the $\Gamma_1$ operator
    "eps_gamma0" :   1e-6, //accuracy of the $\Gamma_0$ or $\sqrt{\Gamma_0}$ operator
    "maxiter_sqrt":  200,  //max iterations of the $\sqrt{\Gamma_0}$ computation
    "maxiter_cauchy": 35,  //max iterations of the Cauchy terms in $\sqrt{\Gamma_0}$ computation
    "eps_cauchy" :  1e-12  //accuracy of the Cauchy integral in the $\sqrt{\Gamma_0}$ computation
},
"physical":
{
    "curvature"  : 0.00015, //$\kappa$
    "tau"  : 4.0,           //$\tau_i = T_{i0}/T_{e0}$
    "equations": "ff-O2"    //model $\in$ ("df-lwl" ,"df-O2" ,"ff-lwl" ,"ff-O2" ,"ff-O4" )
},
"init":
{
    "type"       : "blob",  // Gaussian blob initialization
    "amplitude"  :1.0,      //$A$ of the Gaussian blob
    "sigma"  : 5,           //$\sigma$ of Gaussian blob units of $\rho_s$
    "posX"  : 0.5,          //x initial position $\in (0,1)$
    "posY"  : 0.5           //y initial position $\in (0,1)$
},
"nu_perp"  : 0e-5,          //hyper-diffusion parameter
"bc_x"  : "DIR",            //Boundary condtion in x
"bc_y"  : "PER"             //Boundary condtion in y
\end{minted}
\subsection{Structure of output file}
Output file format: netcdf-4/hdf5
%
%Name | Type | Dimensionality | Description
%---|---|---|---|
\begin{longtable}{lll>{\RaggedRight}p{7cm}}
\toprule
\rowcolor{gray!50}\textbf{Name} &  \textbf{Type} & \textbf{Dimension} & \textbf{Description}  \\ \midrule
inputfile  &             text attribute & 1 & verbose input file as a string \\
energy\_time             & Dataset & 1 & timesteps at which 1d variables are written \\
time                     & Dataset & 1 & time at which fields are written \\
x                        & Dataset & 1 & x-coordinate  \\
y                        & Dataset & 1 & y-coordinate \\
electrons                & Dataset & 3 (time, y, x) & electon density $n$ \\
ions                     & Dataset & 3 (time, y, x) & ion density $N$ \\
potential                & Dataset & 3 (time, y, x) & electric potential $\phi$  \\
vorticity                & Dataset & 3 (time, y, x) & z-component of ExB vorticity  $\Omega_E = \vec{\nabla}\cdot (B^{-1} \vec{\nabla}_{\perp}\phi)$  \\
% dEdt                     & Dataset & 1 (energy\_time) & change of energy per time  \\
% dissipation              & Dataset & 1 (energy\_time) & diffusion integrals  \\
% energy                   & Dataset & 1 (energy\_time) & total energy integral  \\
% mass                     & Dataset & 1 (energy\_time) & mass integral   \\
\bottomrule
\end{longtable}
%..................................................................
\bibliography{../../doc/related_pages/references}
%..................................................................


\end{document}
