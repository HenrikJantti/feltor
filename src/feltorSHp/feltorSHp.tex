%\documentclass[12pt]{article}
%\documentclass[12pt]{scrartcl}
\documentclass{hitec} % contained in texlive-latex-extra 
\settextfraction{0.9} % indent text
\usepackage{csquotes}
\usepackage[hidelinks]{hyperref} % doi links are short and usefull?
\hypersetup{%
    colorlinks=true,
    linkcolor=blue,
    urlcolor=magenta
}
\urlstyle{rm}
\usepackage[english]{babel}
\usepackage{mathtools} % loads and extends amsmath
\usepackage{amssymb}
% packages not used
%\usepackage{graphicx}
%\usepackage{amsthm}
%\usepackage{subfig}
\usepackage{bm}
\usepackage{longtable}
\usepackage{booktabs}
\usepackage{ragged2e} % maybe use \RaggedRight for tables and literature?
\usepackage[table]{xcolor} % for alternating colors
\rowcolors{2}{gray!25}{white}
\renewcommand\arraystretch{1.3}

%%% reset bibliography distances %%%
\let\oldthebibliography\thebibliography
\let\endoldthebibliography\endthebibliography
\renewenvironment{thebibliography}[1]{
  \begin{oldthebibliography}{#1}
    \RaggedRight % remove if justification is desired
    \setlength{\itemsep}{0em}
    \setlength{\parskip}{0em}
}
{
  \end{oldthebibliography}
}
%%% --- %%%

%%%%%%%%%%%%%%%%%%%%%definitions%%%%%%%%%%%%%%%%%%%%%%%%%%%%%%%%%%%%%%%
\newcommand{\eps}{\varepsilon}
\renewcommand{\d}{\mathrm{d}}
\newcommand{\T}{\mathrm{T}}
\renewcommand{\vec}[1]{\boldsymbol{#1}}

\newcommand{\dx}{\,\mathrm{d}x}
%\newcommand{\dA}{\,\mathrm{d}(x,y)}
%\newcommand{\dV}{\mathrm{d}^3{x}\,}
\newcommand{\dA}{\,\mathrm{dA}}
\newcommand{\dV}{\mathrm{dV}\,}
\newcommand{\Eins}{\mathbf{1}}
\newcommand{\ExB}{$\bm{E}\times\bm{B} \,$}
\newcommand{\GKI}{\int d^6 \bm{Z} \BSP}
\newcommand{\GKIV}{\int dv_{\|} d \mu d \theta \BSP}
\newcommand{\BSP}{B_{\|}^*}
\newcommand{\GA}[1]{\langle #1   \rangle}
\newcommand{\Abar}{\langle A_\parallel \rangle}
%Vectors
\newcommand{\bhat}{\bm{\hat{b}}}
\newcommand{\bbar}{\overline{\bm{b}}}
\newcommand{\chat}{\bm{\hat{c}}}
\newcommand{\ahat}{\bm{\hat{a}}}
\newcommand{\xhat}{\bm{\hat{x}}}
\newcommand{\yhat}{\bm{\hat{y}}}
\newcommand{\zhat}{\bm{\hat{z}}}
\newcommand{\Xbar}{\bar{\vec{X}}}
\newcommand{\phat}{\bm{\hat{\perp}}}
\newcommand{\that}{\bm{\hat{\theta}}}
\newcommand{\eI}{\bm{\hat{e}}_1}
\newcommand{\eII}{\bm{\hat{e}}_2}
\newcommand{\ud}{\mathrm{d}}
%Derivatives etc.
\newcommand{\pfrac}[2]{\frac{\partial#1}{\partial#2}}
\newcommand{\ffrac}[2]{\frac{\delta#1}{\delta#2}}
\newcommand{\fixd}[1]{\Big{\arrowvert}_{#1}}
\newcommand{\curl}[1]{\nabla \times #1}
\newcommand{\np}{\nabla_{\perp}}
\newcommand{\npc}{\nabla_{\perp} \cdot }
\newcommand{\nc}{\nabla\cdot }
\newcommand{\GAI}{\Gamma_{1}^{\dagger}}
\newcommand{\GAII}{\Gamma_{1}^{\dagger -1}}

%%%%%%%%%%%%%%%%%%%%%%%%%%%%%DOCUMENT%%%%%%%%%%%%%%%%%%%%%%%%%%%%%%%%%%%%%%%
\begin{document}

\title{The feltorSHp project}
\maketitle

\begin{abstract}
This is a program for 2d thermal electrostatic full-F gyro-fluid simulations for blob studies.
\end{abstract}

\section{Equations}
The four-field model evolves electron density 
\(n_e\), ion gyro-center density \(N_i\), (perpendicular) electron pressure \(p_e\) and (perpendicular) ion gyro-center pressure \(P_i\)
\begin{align}
\label{eq:dtne}
\frac{\partial}{\partial t} n_e =&-
\frac{1}{B} \left[\phi,n_e \right]_{x y} - n_e \mathcal{K}\left(\phi \right) + \frac{1}{e} \mathcal{K}\left(t_{e} n_e \right) +  \Lambda_{n_e}, 
 \\
\label{eq:dtNi}
\frac{\partial}{\partial t} N_i =&
-\frac{1}{B} \left[\psi_i,N_i \right]_{x y} 
+\frac{1}{B} \left[\ln T_{i},N_i \chi_i \right]_{x y} - N_i \mathcal{K}\left(\psi_i + \chi_i\right) 
+ N_i \chi_i \mathcal{K}\left(\ln T_{i} -  \ln N_i \right) \nonumber \\ 
&
- \frac{1}{e} \mathcal{K}\left(T_{i} N_i \right) 
+  \Lambda_{N_i} , \\
\label{eq:dtpe}
     \frac{\partial }{\partial t}   p_e  =&
          -  \frac{1}{B} \left[\phi ,p_e\right]_{xy}  
          - 2 p_e \mathcal{K}(\phi ) 
          + \frac{2 }{e}\mathcal{K}(p_e t_e )\
          + \Lambda_{p_e}  ,
   \\
   \label{eq:dtPi}
     \frac{\partial }{\partial t}   P_i  =&
          -  \frac{1}{B} \left[\psi_i + \chi_i,P_i\right]_{xy} 
          + \frac{2}{B} \left[\ln T_i, P_i \chi_i \right]_{xy}  - 2 P_i \mathcal{K}(\psi_i + 2\chi_i) 
          - \frac{2 }{e}\mathcal{K}(P_i T_i )\nonumber \\ 
         &+2 P_i \chi_i \mathcal{K}(2\ln T_i - \ln P_i )   
          + \Lambda_{P_i}  .
\end{align}
The latter equations are coupled by the thermal version of the nonlinear polarisation equation, which reads:
\begin{align}
  n_e -\Gamma_{1,i}^\dagger N_i &= \vec{\nabla} \cdot\left(\frac{N_i}{\Omega_i B} \vec{\nabla}_\perp \phi\right).
\end{align}
The generlized electric potential, its FLR due to a dynamic gyro-radius and the ExB drift velocity are defined by
\begin{align}
 \psi_i  = \Gamma_{1} \phi - \frac{m u_E^2 }{2 q}, \\ 
 \chi_i := \Gamma_{2,i} \phi \\
 \vec{u}_E = \frac{1}{B} \vec{\hat{b}} \times \vec{\nabla} \phi .
\end{align}
The gyro-averaging operators read:
\begin{align}\label{eq:gamma1def} 
 \Gamma_{1,i} f&:= \frac{1}{1-\frac{\rho_i^2}{2}\vec{\nabla}_\perp^2} f. & 
  \Gamma_{1,i}^\dagger f&:= \frac{1}{1-\vec{\nabla}_\perp^2\frac{\rho_i^2}{2}} f.
\end{align}
\begin{align}\label{eq:gamma2def}
 \Gamma_{2,i} f&: = \frac{\frac{\rho_i^2}{2}\vec{\nabla}_\perp^2}{\left(1-\frac{\rho_i^2}{2}\vec{\nabla}_\perp^2\right)^2} f.&
 \Gamma_{2,i}^\dagger f&: = \frac{\vec{\nabla}_\perp^2\frac{\rho_i^2}{2}}{\left(1-\vec{\nabla}_\perp^2\frac{\rho_i^2}{2}\right)^2} f. 
\end{align}
\subsection{Perpendicular dissipation}
The perpendicular diffusive terms are given by
\begin{align}\label{eq:perpdiffNT}
 \Lambda_{n_e} &=  -\nu_\perp \vec{\nabla}_\perp^4 n_e, &
 \Lambda_{N_i} &=  -\nu_\perp \vec{\nabla}_\perp^4 N_i, &
 \Lambda_{p_e} &=  -\nu_\perp \vec{\nabla}_\perp^4 p_e, &
 \Lambda_{P_i} &=  -\nu_\perp \vec{\nabla}_\perp^4 P_i.
\end{align}
\subsection{Energy theorem}
The energy of our system is \(\frac{d}{d t}\mathcal{E}=\Lambda\)
\begin{align}
 \mathcal{E} &= \int d^2 \vec{x} \left( p_e + P_i + \frac{m_i N_i u_E^2}{2} \right)\\
 \Lambda &= \int d^2 \vec{x} \left[- e \phi \Lambda_{n_e} + e\left( \psi -\chi_i\right) \Lambda_{N_i} + \Lambda_{p_e} + \left(1+ \frac{e}{T_i}\chi_i\right)\Lambda_{P_i} \right].
\end{align}
\subsection{Slab magnetic field}\label{sec:slabapprox}
The here presented slab approximation employs Cartesian coordinates \((x,y,z)\) with a slab magnetic field unit vector \(\vec{\hat{b}} = \vec{\hat{e}}_z\). The inverse magnetic field magnitude is 
radially varying 
\begin{align}
 \frac{1}{B}&= \frac{1}{B_0} \left( 1+x/R_0\right) .
\end{align}
with \(R_0\) the distance to the outboard mid-plane and \(B_0\) the reference magnetic field magnitude. 
For a slab magnetic field the curvature \(\vec{\kappa} = 0\) vanishes and the curvature operator \(\mathcal{K} (f)\) reduces to
\begin{align}
  \mathcal{K} (f) &=\mathcal{K}_{\vec{\nabla}  B} (f) =-\frac{1}{B_0 R_0 } \frac{\partial }{\partial y }f.
\end{align}
We note here that no factor of two arises, as is the case in the low beta approximation. In slab approximation the following relations hold
\begin{align}
   \vec{\nabla} \cdot  \vec{\mathcal{K}}_{\kappa} &=   \vec{\nabla} \cdot \vec{\mathcal{K}}_{\vec{\nabla}  B} =  \vec{\nabla} \cdot \vec{ \mathcal{K}} =   \vec{\nabla} \cdot \vec{\hat{b}} =  0,     
    &  \vec{\kappa} \cdot \vec{\mathcal{K}}_{\vec{\nabla}  B} &= 0.
\end{align}
and energetic consistency is assured in the curvature parts.
\section{Initialization}
\subsection{\(\phi=0\) initialization}
\begin{align}
  n_e  =\Gamma_{1,i}^\dagger N_i, \\ 
 p_{i} = \left(\Gamma_{1,i}^{\dagger} + \Gamma_{2,i}^{\dagger} \right)P_{i} .
\end{align}
For the ion gyro-centre density and perpendicular temperature we mimic an initial blob by a Gaussian of the form
\begin{eqnarray}
%  \qquad 
 N_{i}\left(\vec{x},0\right) =   n_{e0}\left[1+A \exp{\left(-\frac{\left(\vec{x}-\vec{x}_0\right)^2}{2\sigma^2}\right)}\right],  \\
%  \qquad 
 T_{i}\left(\vec{x},0\right) =   t_{i0}\left[1+A \exp{\left(-\frac{\left(\vec{x}-\vec{x}_0\right)^2}{2\sigma^2}\right)} \right],
\end{eqnarray}
\section{Numerical methods}
discontinuous Galerkin on structured grid 
\begin{longtable}{ll>{\RaggedRight}p{7cm}}
\toprule
\rowcolor{gray!50}\textbf{Term} &  \textbf{Method} & \textbf{Description}  \\ \midrule
coordinate system & cartesian 2D & equidistant discretization of $[0,l_x] \times [0,l_y]$, equal number of Gaussian nodes in x and y \\
matrix inversions & conjugate gradient & Use previous two solutions to extrapolate initial guess and $1/\chi$ as preconditioner \\
\ExB advection & Poisson & \\
curvature terms & direct & cf. slab approximations \\
time &  Karniadakis multistep & $3rd$ order explicit, diffusion $2nd$ order implicit \\
\bottomrule
\end{longtable}
\subsection{Input file structure}
Input file format: json

%%This is a booktabs table
\begin{longtable}{llll>{\RaggedRight}p{7cm}}
\toprule
\rowcolor{gray!50}\textbf{Name} &  \textbf{Type} & \textbf{Example} & \textbf{Default} & \textbf{Description}  \\ \midrule
n      & integer & 3 & - &\# Gaussian nodes in x and y \\
Nx     & integer &100& - &\# grid points in x \\
Ny     & integer &100& - &\# grid points in y \\
dt     & integer &1.0& - &time step in units of $c_{s0}/\rho_{s0}$ \\
n\_out  & integer &3  & - &\# Gaussian nodes in x and y in output \\
Nx\_out & integer &100& - &\# grid points in x in output fields \\
Ny\_out & integer &100& - &\# grid points in y in output fields \\
itstp  & integer &2  & - &   steps between outputs \\
maxout & integer &100& - &      \# outputs excluding first \\
eps\_pol   & float &1e-6    & - &  accuracy of polarisation solver \\
eps\_gamma & float &1e-7    & - & accuracy of $\Gamma_1$ and $\Gamma_2$\\
eps\_time  & float &1e-10   & - & accuracy of implicit time-stepper \\
curvature  & float &0.00015& - & magnetic curvature $\kappa:=\rho_{s0}/R_0$ \\
tau        & float &1      & - & $\tau = T_i/T_e$  \\
nu\_perp    & float &5e-3   & - & pependicular viscosity $\nu$ \\
amplitude  & float &1.0    & - & amplitude $A$ of the blob \\
sigma      & float &10     & - & blob radius $\sigma$ \\
posX       & float &0.3    & - & blob x-position in units of $l_x$, i.e. $X = p_x l_x$\\
posY       & float &0.5    & - & blob y-position in units of $l_y$, i.e. $Y = p_y l_y$ \\
lx         & float &200    & - & $l_x$  \\
ly         & float &200    & - & $l_y$  \\
bc\_x   & char & "DIR"      & - & boundary condition in x (one of PER, DIR, NEU, DIR\_NEU or NEU\_DIR) \\
bc\_y   & char & "PER"      & - & boundary condition in y (one of PER, DIR, NEU, DIR\_NEU or NEU\_DIR) \\
initmode  & integer & 0 & - & \(n_e = \Gamma_1^\dagger N_i\)(0), \(n_e = N_i\) (1) (cf. initialization)\\
tempmode & integer & 0    & - &thermal (0), isothermal (1)\\
flrmode  & integer & 1    & - &const FLR (0), dyn. FLR (1)\\
\bottomrule
\end{longtable}

The default value is taken if the value name is not found in the input file. If there is no default and
the value is not found,
the program exits with an error message.
%..................................................................
\begin{thebibliography}{1}
 \bibitem{held16b}
  M. Held, M. Wiesenberger, J. Madsen, A. Kendl,  Nuclear Fusion 56, 126005 (2016)
\end{thebibliography}
%..................................................................


\end{document}
