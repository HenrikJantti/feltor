%\documentclass[12pt]{article}
%\documentclass[12pt]{scrartcl}
\documentclass{hitec} % contained in texlive-latex-extra 
\settextfraction{0.9} % indent text
\usepackage{csquotes}
\usepackage[hidelinks]{hyperref} % doi links are short and usefull?
\hypersetup{%
    colorlinks=true,
    linkcolor=blue,
    urlcolor=magenta
}
\urlstyle{rm}
\usepackage[english]{babel}
\usepackage{mathtools} % loads and extends amsmath
\usepackage{amssymb}
% packages not used
%\usepackage{graphicx}
%\usepackage{amsthm}
%\usepackage{subfig}
\usepackage{bm}
\usepackage{longtable}
\usepackage{booktabs}
\usepackage{ragged2e} % maybe use \RaggedRight for tables and literature?
\usepackage[table]{xcolor} % for alternating colors
\rowcolors{2}{gray!25}{white}
\renewcommand\arraystretch{1.3}

%%% reset bibliography distances %%%
\let\oldthebibliography\thebibliography
\let\endoldthebibliography\endthebibliography
\renewenvironment{thebibliography}[1]{
  \begin{oldthebibliography}{#1}
    \RaggedRight % remove if justification is desired
    \setlength{\itemsep}{0em}
    \setlength{\parskip}{0em}
}
{
  \end{oldthebibliography}
}
%%% --- %%%

%%%%%%%%%%%%%%%%%%%%%definitions%%%%%%%%%%%%%%%%%%%%%%%%%%%%%%%%%%%%%%%
\newcommand{\eps}{\varepsilon}
\renewcommand{\d}{\mathrm{d}}
\newcommand{\T}{\mathrm{T}}
\renewcommand{\vec}[1]{\boldsymbol{#1}}

\newcommand{\dx}{\,\mathrm{d}x}
%\newcommand{\dA}{\,\mathrm{d}(x,y)}
%\newcommand{\dV}{\mathrm{d}^3{x}\,}
\newcommand{\dA}{\,\mathrm{dA}}
\newcommand{\dV}{\mathrm{dV}\,}
\newcommand{\Eins}{\mathbf{1}}
\newcommand{\ExB}{$\bm{E}\times\bm{B} \,$}
\newcommand{\GKI}{\int d^6 \bm{Z} \BSP}
\newcommand{\GKIV}{\int dv_{\|} d \mu d \theta \BSP}
\newcommand{\BSP}{B_{\|}^*}
\newcommand{\GA}[1]{\langle #1   \rangle}
\newcommand{\Abar}{\langle A_\parallel \rangle}
%Vectors
\newcommand{\bhat}{\bm{\hat{b}}}
\newcommand{\bbar}{\overline{\bm{b}}}
\newcommand{\chat}{\bm{\hat{c}}}
\newcommand{\ahat}{\bm{\hat{a}}}
\newcommand{\xhat}{\bm{\hat{x}}}
\newcommand{\yhat}{\bm{\hat{y}}}
\newcommand{\zhat}{\bm{\hat{z}}}
\newcommand{\Xbar}{\bar{\vec{X}}}
\newcommand{\phat}{\bm{\hat{\perp}}}
\newcommand{\that}{\bm{\hat{\theta}}}
\newcommand{\eI}{\bm{\hat{e}}_1}
\newcommand{\eII}{\bm{\hat{e}}_2}
\newcommand{\ud}{\mathrm{d}}
%Derivatives etc.
\newcommand{\pfrac}[2]{\frac{\partial#1}{\partial#2}}
\newcommand{\ffrac}[2]{\frac{\delta#1}{\delta#2}}
\newcommand{\fixd}[1]{\Big{\arrowvert}_{#1}}
\newcommand{\curl}[1]{\nabla \times #1}
\newcommand{\np}{\nabla_{\perp}}
\newcommand{\npc}{\nabla_{\perp} \cdot }
\newcommand{\nc}{\nabla\cdot }
\newcommand{\GAI}{\Gamma_{1}^{\dagger}}
\newcommand{\GAII}{\Gamma_{1}^{\dagger -1}}

%%%%%%%%%%%%%%%%%%%%%%%%%%%%%DOCUMENT%%%%%%%%%%%%%%%%%%%%%%%%%%%%%%%%%%%%%%%
\begin{document}

\title{The feltorSesol project}
\maketitle

\begin{abstract}
This is a program for 2d isothermal electrostatic full-F gyro-fluid simulations for drift-wave turbulence and blob studies around the separatrix.
\end{abstract}

\section{Equations}
The two-field model evolves electron density 
\(n_e\), ion gyro-center density \(N_i\)
% In the following we derive an isothermal coupled edge-scrape-off layer model in slab geometry (cf.~\ref{sec:slabapprox}). The magnetic field varies now in radial direction \(\vec{B} = B(x) \vec{\hat{e}}_z\), which introduces
% the interchange instability. Since the model is derived in slab geometry, parallel closure terms are applied in order to derive a 2D model. 
% In the edge we assume that a single \(k_\parallel \neq 0 \) mode drives the drift wave (universal) instability, 
% whereby zonal components of the fluctuations do not contribute to the parallel current. 
% In the SOL only a single  \(k_\parallel = 0 \) mode is kept and the Bohm-sheath boundary condition enters due to
% the parallel closure term. The boundary between the edge and SOL is defined by \(x_B = \alpha L_x\), with \(\alpha\) the parameter \(\alpha \in \left[0,1\right]\) and \(L_x\) the radial length of the simulation box.
The considered species are again electrons and ions. The gyro-fluid moment equations read 
\begin{align}\label{eq:2Desolne}
\frac{\partial}{\partial t} n_e =&-
\frac{1}{B} \left[\phi,n_e \right]_{\perp} - n_e \mathcal{K}\left(\phi \right) + \frac{t_{e\perp}}{e} \mathcal{K}\left( n_e \right) 
-\nabla_\parallel \left(n_e u_e\right)
+  \Lambda_{n_e}  ,
% + \Lambda_{n_e,\parallel} 
\\
\label{eq:2DesolNi}
\frac{\partial}{\partial t} N_i =&
-\frac{1}{B} \left[\psi_i,N_i \right]_{\perp}- N_i \mathcal{K}\left(\psi_i \right) 
- \frac{T_{i\perp}}{e} \mathcal{K}\left( N_i \right) 
-\nabla_\parallel \left(N_i U_i\right)
+  \Lambda_{N_i}  ,
% + \Lambda_{N_i,\parallel} 
\end{align}
which are coupled via the nonlinear polarisation equation:
\begin{align}\label{eq:hwpol}
  n_e -\Gamma_{1,i} N_i &= \vec{\nabla} \cdot\left(\frac{N_i}{\Omega_i B} \vec{\nabla}_\perp \phi\right).
\end{align}.
The perpendicular diffusive terms are given by Equations~\eqref{eq:perpdiffn}.
\subsection{Edge closure}\label{sec:edgeclosure}
In the edge we use the modified Hasegawa-Wakatani closure 
\begin{align}\label{eq:modhw}
\nabla_\parallel \left(n_e u_e\right) &= n_{e,0} \Omega_{i,0} \alpha  h_{e}(x)
\left[ \ln (n_e) - \langle  \ln (n_e)\rangle_y  - \frac{e}{t_{e,\parallel}} \left(\phi - \langle \phi\rangle_y\right)
\right].
\end{align}
for the electron continuity Equation~\eqref{eq:2Desolne} and \(\nabla_\parallel \left(N_i U_i\right) = 0\) for the ion continuity Equation~\eqref{eq:2DesolNi}.
The adiabaticity parameter is given by 
\begin{align}
 \alpha:= \frac{ t_{e,\parallel}  k_\parallel^2 }{ \eta_\parallel e^2   n_{e,0} \Omega_{i,0} }
\end{align}
The damping function is 
\begin{align}
 h_{e}(x) &=\frac{1}{2}\left[1- \tanh{\left(\frac{(x-x_e)}{\sigma_e}\right)}\right], & 
\end{align}
\subsection{SOL closure}\label{sec:solclosure}
We write down the Bohm sheath boundary condition, which are 
\begin{align}
 n_e \Big|_{se} &= n_e, &
 u_e \Big|_{se} &= \pm c_{s0} \exp{\left(\Lambda-\frac{e \phi}{t_e}\right)},&
 u_i \Big|_{se} &= \pm c_{s}.
\end{align}
Here, we introduced the cold ion acoustic speed \(c_{s0} \equiv \sqrt{\frac{t_{e0} }{ m_i}}\), the ion acoustic speed \(c_{s} = \sqrt{\frac{t_{e0} + t_{i0} }{ m_i}}\) and the constant \(\Lambda = \ln{\sqrt{ \frac{m_i}{2 \pi m_e} }}\). 
The subscript \(se\) denotes the position at the presheath entrance.
In the  SOL we average the Equations~\eqref{eq:2Desolne} and~\eqref{eq:2DesolNi} over the parallel coordinate 
according to \(\langle f\rangle_\parallel =  \frac{1}{L_\parallel} \int_{0}^{L_\parallel} f d z \) and retain only modes with \(k_\parallel = 0\). 
For the parallel closure terms we apply the fundamental theorem of calculus and insert the Bohm-sheath boundary conditions
\begin{align}\label{eq:sheathneclosure}
 \langle \nabla_\parallel (u_e n_e)  \rangle_\parallel 
 &=
		       h_{s}(x)  n_e c_{s0} \frac{2}{L_\parallel}\exp{\left(\Lambda-\frac{e \phi}{t_e}\right)}, \\
		        \label{eq:sheathNiclosure}
   \langle \nabla_\parallel (U_i N_i)   \rangle_\parallel
		      &= h_{s}(x) c_{s} \frac{2}{L_\parallel}  \Gamma_1^{-1}  n_e .
\end{align}
The damping function is 
\begin{align}
 h_{s}(x) &=\frac{1}{2}\left[1+ \tanh{\left(\frac{(x-x_s)}{\sigma_s}\right)}\right] ,
\end{align}
With the help of the Bohm sheath closures of Equations~\eqref{eq:sheathneclosure} and~\eqref{eq:sheathNiclosure} the parallel derivative of the total parallel current 
\(J_\parallel = -e  \left[u_e n_e - \Gamma_1 (U_i N_i)\right] \) is derived to
\begin{align}
 \langle \nabla_\parallel  J_\parallel \rangle_\parallel &= - e h_{s}(x)  n_e  \frac{2}{L_\parallel}c_{s}  \left[1-\frac{c_{s0}}{ c_{s} } \exp{\left(\Lambda-\frac{e \phi}{t_e}\right)} \right].
\end{align}
This term appears explicitly in the LWL gyro-fluid vorticity equation. We note here that in the isothermal case the constant \(\Lambda\) can be dropped by renormalisation of the electric potential \(\phi\).
\subsection{Perpendicular dissipation}
Dissipation in the perpendicular plane is approached via hyperviscosity \(\nu_\perp(-1)^{m+1}\vec{\nabla}_\perp^{2m}\) of order \(m\), which replaces the usual viscous term 
\(\nu_\perp \vec{\nabla}_\perp^2\) of direct numerical simulations. In contrast to the common dissipation, the hyperviscous treatment causes the dissipation range to be narrower, 
requiring fewer modes to resolve.
In our simulations the dissipative term is replaced by a second order hyperviscous term \(\nu_\perp \vec{\nabla}_\perp^2 \rightarrow -\nu_\perp \vec{\nabla}_\perp^4 \). The diffusive terms are
\begin{align}\label{eq:perpdiffn}
 \Lambda_{n_e} &=  -\nu_\perp \vec{\nabla}_\perp^4 n_e, &
 \Lambda_{N_i} &=  -\nu_\perp \vec{\nabla}_\perp^4 N_i.
\end{align}
\subsection{Sources and Sinks}
\subsection{Energy theorem}
The energy theorem is in general stated by
\begin{align}\label{eq:energytheorem}
 \frac{\partial \mathcal{E} }{\partial t} &= \Lambda,
\end{align}
The energy is 
The explicit expressions are
\begin{align}\label{eq:energyhw}
  \mathcal{E} =& \int d\vec{x} \left(n_e t_{e} \ln (n_e) +N_i T_{i} \ln (N_i)  + \frac{m_i N_i u_E^2}{2} \right) 
  \end{align}
  In the edge the right hand side is
  \begin{align}
\label{eq:dissipationhw}
\Lambda = &
 \int d\vec{x} \bigg\{\left[t_{e} (1+\ln (n_e))  - e \phi \right] \left(\Lambda_{n_e}-\nabla_\parallel \left(n_e u_e\right)\right)  +\left(T_{i}  (1+\ln (N_i))+ e \psi_i \right) \Lambda_{N_i}  \bigg\}.
\end{align}
whereas in the SOL we use
\begin{align}
\Lambda = &
 \int d\vec{x} \bigg\{\left[t_{e} (1+\ln (n_e))  - e \phi \right] \left(\Lambda_{n_e}- \langle \nabla_\parallel (u_e n_e)  \rangle_\parallel 
\right)  
\nonumber \\ &
+\left(T_{i}  (1+\ln (N_i))+ e \psi_i \right) \left(\Lambda_{N_i}-   \langle \nabla_\parallel (U_i N_i)   \rangle_\parallel
  \right)\bigg\}.
\end{align}
In the edge we can still employ the expression given by Equation~\eqref{eq:dissipationhw}. \\
Numerical solutions of the isothermal edge-SOL model are discussed in Section~\ref{sec:esolsim}. The discussion covers the impact of finite ion background temperature on the mean \(\vec{E}\times\vec{B}\) flow and
the blob and structure formation in the edge and SOL.\section{Numerical methods}
discontinuous Galerkin on structured grid 
\begin{longtable}{ll>{\RaggedRight}p{7cm}}
\toprule
\rowcolor{gray!50}\textbf{Term} &  \textbf{Method} & \textbf{Description}  \\ \midrule
coordinate system & cartesian 2D & equidistant discretization of $[0,l_x] \times [0,l_y]$, equal number of Gaussian nodes in x and y \\
matrix inversions & conjugate gradient & Use previous two solutions to extrapolate initial guess and $1/\chi$ as preconditioner \\
\ExB advection & Poisson & \\
curvature terms & direct & cf. slab approximations \\
time &  Karniadakis multistep & $3rd$ order explicit, diffusion $2nd$ order implicit \\
\bottomrule
\end{longtable}
%..................................................................
\begin{thebibliography}{1}
 \bibitem{held:phd17}
  M. Held, Full-F gyro-fluid modelling of the tokamak edge and scrape-off, Universit{\"a}t Innsbruck (2017)
\end{thebibliography}
%..................................................................


\end{document}
