%%%%%%%%%%%%%%%%%%%%%definitions%%%%%%%%%%%%%%%%%%%%%%%%%%%%%%%%%%%%%%%

%\documentclass[12pt]{article}
%\documentclass[12pt]{scrartcl}
\documentclass{hitec} % contained in texlive-latex-extra
\settextfraction{0.9} % indent text
\usepackage{csquotes}
\usepackage[hidelinks]{hyperref} % doi links are short and usefull?
\hypersetup{%
    colorlinks=true,
    linkcolor=blue,
    urlcolor=magenta
}
\urlstyle{rm}
\usepackage[english]{babel}
\usepackage{mathtools} % loads and extends amsmath
\usepackage{amssymb}
% packages not used
%\usepackage{graphicx}
%\usepackage{amsthm}
%\usepackage{subfig}
\usepackage{bm}
\usepackage{longtable}
\usepackage{booktabs}
\usepackage{ragged2e} % maybe use \RaggedRight for tables and literature?
\usepackage[table]{xcolor} % for alternating colors
%\rowcolors{2}{gray!25}{white} %%% Use this line in front of longtable
\renewcommand\arraystretch{1.3}
\usepackage[most]{tcolorbox}
\usepackage{doi}
\usepackage[sort,square,numbers]{natbib}
\bibliographystyle{abbrvnat}
%%% reset bibliography distances %%%
\let\oldthebibliography\thebibliography
\let\endoldthebibliography\endthebibliography
\renewenvironment{thebibliography}[1]{
  \begin{oldthebibliography}{#1}
    \RaggedRight % remove if justification is desired
    \setlength{\itemsep}{0em}
    \setlength{\parskip}{0em}
}
{
  \end{oldthebibliography}
}
%%% --- %%%

\definecolor{light-gray}{gray}{0.95}
\newcommand{\code}[1]{\colorbox{light-gray}{\texttt{#1}}}
\newcommand{\eps}{\varepsilon}
\renewcommand{\d}{\mathrm{d}}
\renewcommand{\vec}[1]{{\boldsymbol{#1}}}
\newcommand{\dx}{\,\mathrm{d}x}
%\newcommand{\dA}{\,\mathrm{d}(x,y)}
%\newcommand{\dV}{\mathrm{d}^3{x}\,}
\newcommand{\dA}{\,\mathrm{dA}}
\newcommand{\dV}{\mathrm{dV}\,}

\newcommand{\Eins}{\mathbf{1}}

\newcommand{\ExB}{$\bm{E}\times\bm{B} \,$}
\newcommand{\GKI}{\int d^6 \bm{Z} \BSP}
\newcommand{\GKIV}{\int dv_{\|} d \mu d \theta \BSP}
\newcommand{\BSP}{B_{\|}^*}
\newcommand{\Abar}{\langle A_\parallel \rangle}
%Averages
\newcommand{\RA}[1]{\left \langle #1 \right \rangle} %Reynolds (flux-surface) average
\newcommand{\RF}[1]{\widetilde{#1}} %Reynolds fluctuation
\newcommand{\FA}[1]{\left[\left[ #1 \right]\right]} %Favre average
\newcommand{\FF}[1]{\widehat{#1}} %Favre fluctuation
\newcommand{\PA}[1]{\left \langle #1 \right\rangle_\varphi} %Phi average

%Vectors
\newcommand{\ahat}{\bm{\hat{a}}}
\newcommand{\bhat}{\bm{\hat{b}}}
\newcommand{\chat}{\bm{\hat{c}}}
\newcommand{\ehat}{\bm{\hat{e}}}
\newcommand{\bbar}{\overline{\bm{b}}}
\newcommand{\xhat}{\bm{\hat{x}}}
\newcommand{\yhat}{\bm{\hat{y}}}
\newcommand{\zhat}{\bm{\hat{z}}}

\newcommand{\Xbar}{\bar{\vec{X}}}
\newcommand{\phat}{\bm{\hat{\perp}}}
\newcommand{\that}{\bm{\hat{\theta}}}

\newcommand{\eI}{\bm{\hat{e}}_1}
\newcommand{\eII}{\bm{\hat{e}}_2}
\newcommand{\ud}{\mathrm{d}}

%Derivatives etc.
\newcommand{\pfrac}[2]{\frac{\partial#1}{\partial#2}}
\newcommand{\ffrac}[2]{\frac{\delta#1}{\delta#2}}
\newcommand{\fixd}[1]{\Big{\arrowvert}_{#1}}
\newcommand{\curl}[1]{\nabla \times #1}

\newcommand{\np}{\vec{\nabla}_{\perp}}
\newcommand{\npc}{\nabla_{\perp} \cdot }
\newcommand{\nc}{\vec\nabla\cdot}
\newcommand{\cn}{\cdot\vec\nabla}
\newcommand{\vn}{\vec{\nabla}}
\newcommand{\npar}{\nabla_\parallel}

\newcommand{\GAI}{\Gamma_{1}^{\dagger}}
\newcommand{\GAII}{\Gamma_{1}^{\dagger -1}}
\newcommand{\T}{\mathrm{T}}
\newcommand{\Tp}{\mathcal T^+_{\Delta\varphi}}
\newcommand{\Tm}{\mathcal T^-_{\Delta\varphi}}
\newcommand{\Tpm}{\mathcal T^\pm_{\Delta\varphi}}
%%%%%%%%Some useful abbreviations %%%%%%%%%%%%%%%%
\def\feltor{{\textsc{Feltor }}}

\def\fixme#1{\typeout{FIXME in page \thepage :{#1}}%
 \textsc{\color{red}[{#1}]}}



\renewcommand{\ne}{\ensuremath{n }}
\newcommand{\Ni}{\ensuremath{N}}
\newcommand{\ue}{\ensuremath{u_\parallel}}
\newcommand{\Ui}{\ensuremath{U_\parallel}}
\newcommand{\Apar}{\ensuremath{A_\parallel}}
\newcommand{\neref}{\ensuremath{n_0}}
\newcommand{\Teref}{\ensuremath{T_{e0}}} %\rhoN

%%%%%%%%%%%%%%%%%%%%%%%%%%%%%DOCUMENT%%%%%%%%%%%%%%%%%%%%%%%%%%%%%%%%%%%%%%%
\begin{document}

\title{Collisionless Reconnection}
\author{ M.~Held and M.~Wiesenberger}
\maketitle

\begin{abstract}
\end{abstract}

\section{Equations}
%tearing instability <-> reconnection
Collisionless reconnection was studied in~\cite{stanier15}, gyro-fluid and gyro-kinetic studies~\cite{comisso13,zacharias14}. drift fluid-kinetic model~\cite{loureiro13}.
Our reduced model is based on a recent formulation of a full-F gyro-fluid model~\cite{Madsen2013}.
It consists of the first two moment equations for electrons and ions
\begin{align}
%%%
\frac{\partial}{\partial t} \ne =&
 - \left[\phi, \ne\right]
+ \left[\Apar,\ne \ue  \right]
\\
%%%
\label{firstgyromom}
\frac{\partial}{\partial t} \Ni =&
 - \left[\psi, \Ni\right]
+ \left[\Gamma_1 \Apar ,\Ni \Ui  \right]
 \\
 %%%
\frac{\partial}{\partial t} \left( \ue+ \frac{1}{\mu_e} \Apar \right) =&
      -  \left[ \psi, \ue+ \frac{1}{\mu_e} \Apar  \right]% \nonumber \\
    +   \left[\Apar,\ue^2/2   \right]% \nonumber \\
      - \frac{1}{\mu_e}  \left[\Apar,\ln{\ne}   \right]
      \\
      \frac{\partial}{\partial t} \left( \Ui+ \Gamma_1 \Apar  \right) =&
      -  \left[ \psi, \Ui+ \Gamma_1 \Apar  \right]% \nonumber \\
     +   \left[\Gamma_1 \Apar,\Ui^2/2   \right]% \nonumber \\
      + \tau_i  \left[\Gamma_1 \Apar,\ln{\Ni}   \right]
\end{align}
which are coupled by polarisation and induction
\begin{align}
 \ne - \Gamma_1 \Ni &= \vec{\nabla}\cdot\left(\frac{\Ni }{  \Ni} \vec{\nabla}_\perp \phi\right) \\
 \frac{1}{\beta} \vec{\nabla}_\perp^2 \Apar &=  \ne \ue - \Gamma_1 (\Ni \Ui)
\end{align}
with gyro-averaged parallel electromagnetic vector potential, generalized electric potential and gyro-averaging operator
\begin{align}
 \psi &:= \Gamma_1 \phi - u_E^2 /2  \\
  \Gamma_1 &:= (1-\frac{\tau_i}{2} \vec{\nabla}_\perp^2 )^{-1} 
\end{align}
and ion gyro-radius, vacuum permeability and ion gyrofrequency
\begin{align}
  \rho_{i}     &:= \frac{\sqrt{T_{i} m_i}}{e B} \\
  \mu_0 &:= 1/(\epsilon_0 c^2) \\
  \Omega_i &:= e B / m_i
\end{align}
After exploiting the gyro-Bohm normalisation, our model is controlled by the dimensionless parameters. These
are the
\begin{align}
 \mu   &:=  \frac{m}{Z m_i} \\ 
 \tau  &:=  \frac{T}{ Z T_i}\\
 \beta_{e0} &:=  \frac{\mu_0 \neref \Teref }{ B_0^2 }
\end{align}
These parameters are related to the conventional MHD normalisation (e.g.~\cite{comisso13,zacharias14}) according to
\begin{align}
 \hat{d}_i &= \frac{\hat{\rho}_{s}}{\sqrt{\beta_{e0}}} \\
 \hat{d}_e &= \hat{d}_i \sqrt{|\mu_e|} \\
 \hat{\rho}_{i} &:= \frac{ \hat{\rho}_{s}}{\sqrt{\tau}}\\
 \hat{\rho}_{s} &:= \frac{\rho_{s}}{L}
\end{align}
Here, \(\hat{d}_i:=d_i/L=c/(\omega_{p,i} L)\) is the normalised ion skin depth, \(\hat{d}_e:=d_e/L=c/(\omega_{p,e} L)\) is the normalised electron skin depth, 
\(\hat{\rho}_{i}:=\rho_i/L\) is the normalised ion gyro-radius and \(\hat{\rho}_{s}\) is the normalised length scale.
\\

\section{Reconnection rate}
 A consistent definition of the reconnection rate is not trivial~\cite{comisso16}. 
 We adopt a similar definiton as in Ref.~\cite{comisso13}
\begin{align}
 Q_X&:=|  \Apar(\vec{x}_X, 0)-\Apar(\vec{x}_X,t) | \\
 \gamma&:= d \ln{(Q_X)}/dt
\end{align}
\section{Energy}
The inherent energy of our system is:
\begin{align}
 \mathcal{E} := \int dV \big[&
                    \ne \ln{(\ne)}
                  + \tau_i \Ni \ln{( \Ni)}
                  \nonumber \\
                 &- \frac{1}{2} \mu_e \ne \ue^2
                  + \frac{1}{2} \Ni \Ui^2
                  \nonumber \\
                 &+\frac{1}{2} \Ni u_E^2
                  + \frac{|\nabla_\perp \Apar|^2}{2 \beta}
                  \big]
\end{align}

\section{Initial condition:}
The initial parallel magnetic vector potential consists of a equilbrium part and 
a fluctuating part in the periodic y-direction
\begin{align}
 \Apar = A_{\parallel,eq} + \tilde{A}_\parallel
\end{align}
which are given by
\begin{align}
 A_{\parallel,eq}   &:=  A_0 / \cosh{( 4  \pi x / L_x )}^2\\
 \tilde{A}_\parallel&:=A_1\cos{(2 \pi y/L_y)} 
\end{align}
The remaining fields are initially 
\begin{align}
 \Ni&=\ne=\neref \\
 \phi &= 0 \\
  \Ui&=0 \\
  \ue &= \frac{1}{\ne \beta }\vec{\nabla}^2_{\perp} \Apar
\end{align}

\subsection{Initialization}
Initialization of $n$ is a Gaussian 
\begin{align}
    n(x,y) = 1 + A\exp\left( -\frac{(x-X)^2 + (y-Y)^2}{2\sigma^2}\right)
    \label{}
\end{align}
where $X = p_x l_x$ and $Y=p_yl_y$ are the initial centre of mass position coordinates, $A$ is the amplitude and $\sigma$ the
radius of the blob.
We initialize 
\begin{align}
    N = \Gamma_1^{-1} n \quad \phi = 0 \\
    \rho = \phi = 0
    \label{}
\end{align}
\subsection{Diagnostics}
\begin{align}
    M(t) = \int n-1 \\
    \Lambda_n = \nu \int \Delta n  \\
    ...
    \label{}
\end{align}
\section{Numerical methods}
discontinuous Galerkin on structured grid
\rowcolors{2}{gray!25}{white} %%% Use this line in front of longtable
\begin{longtable}{ll>{\RaggedRight}p{7cm}}
\toprule
\rowcolor{gray!50}\textbf{Term} &  \textbf{Method} & \textbf{Description}  \\ \midrule
coordinate system & Cartesian 2D & equidistant discretization of $[0,l_x] \times [0,l_y]$, equal number of Gaussian nodes in x and y \\
matrix inversions & conjugate gradient & Use previous two solutions to extrapolate initial guess and $1/\chi$ as preconditioner \\
\ExB advection & Arakawa & s.a. \cite{Einkemmer2014} \\
curvature terms & direct & flux conserving \\
time &  Karniadakis multistep & $3rd$ order explicit, diffusion $2nd$ order implicit \\
\bottomrule
\end{longtable}

%..................................................................
\bibliography{../../doc/related_pages/references}
\end{document}
