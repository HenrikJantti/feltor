%%%%%%%%%%%%%%%%%%%%%definitions%%%%%%%%%%%%%%%%%%%%%%%%%%%%%%%%%%%%%%%

%\documentclass[12pt]{article}
%\documentclass[12pt]{scrartcl}
\documentclass{hitec} % contained in texlive-latex-extra
\settextfraction{0.9} % indent text
\usepackage{csquotes}
\usepackage[hidelinks]{hyperref} % doi links are short and usefull?
\hypersetup{%
    colorlinks=true,
    linkcolor=blue,
    urlcolor=magenta
}
\urlstyle{rm}
\usepackage[english]{babel}
\usepackage{mathtools} % loads and extends amsmath
\usepackage{amssymb}
% packages not used
%\usepackage{graphicx}
%\usepackage{amsthm}
%\usepackage{subfig}
\usepackage{bm}
\usepackage{longtable}
\usepackage{booktabs}
\usepackage{ragged2e} % maybe use \RaggedRight for tables and literature?
\usepackage[table]{xcolor} % for alternating colors
%\rowcolors{2}{gray!25}{white} %%% Use this line in front of longtable
\renewcommand\arraystretch{1.3}
\usepackage[most]{tcolorbox}
\usepackage{doi}
\usepackage[sort,square,numbers]{natbib}
\bibliographystyle{abbrvnat}
%%% reset bibliography distances %%%
\let\oldthebibliography\thebibliography
\let\endoldthebibliography\endthebibliography
\renewenvironment{thebibliography}[1]{
  \begin{oldthebibliography}{#1}
    \RaggedRight % remove if justification is desired
    \setlength{\itemsep}{0em}
    \setlength{\parskip}{0em}
}
{
  \end{oldthebibliography}
}
%%% --- %%%

\definecolor{light-gray}{gray}{0.95}
\newcommand{\code}[1]{\colorbox{light-gray}{\texttt{#1}}}
\newcommand{\eps}{\varepsilon}
\renewcommand{\d}{\mathrm{d}}
\renewcommand{\vec}[1]{{\boldsymbol{#1}}}
\newcommand{\dx}{\,\mathrm{d}x}
%\newcommand{\dA}{\,\mathrm{d}(x,y)}
%\newcommand{\dV}{\mathrm{d}^3{x}\,}
\newcommand{\dA}{\,\mathrm{dA}}
\newcommand{\dV}{\mathrm{dV}\,}

\newcommand{\Eins}{\mathbf{1}}

\newcommand{\ExB}{$\bm{E}\times\bm{B} \,$}
\newcommand{\GKI}{\int d^6 \bm{Z} \BSP}
\newcommand{\GKIV}{\int dv_{\|} d \mu d \theta \BSP}
\newcommand{\BSP}{B_{\|}^*}
\newcommand{\Abar}{\langle A_\parallel \rangle}
%Averages
\newcommand{\RA}[1]{\left \langle #1 \right \rangle} %Reynolds (flux-surface) average
\newcommand{\RF}[1]{\widetilde{#1}} %Reynolds fluctuation
\newcommand{\FA}[1]{\left[\left[ #1 \right]\right]} %Favre average
\newcommand{\FF}[1]{\widehat{#1}} %Favre fluctuation
\newcommand{\PA}[1]{\left \langle #1 \right\rangle_\varphi} %Phi average

%Vectors
\newcommand{\ahat}{\bm{\hat{a}}}
\newcommand{\bhat}{\bm{\hat{b}}}
\newcommand{\chat}{\bm{\hat{c}}}
\newcommand{\ehat}{\bm{\hat{e}}}
\newcommand{\bbar}{\overline{\bm{b}}}
\newcommand{\xhat}{\bm{\hat{x}}}
\newcommand{\yhat}{\bm{\hat{y}}}
\newcommand{\zhat}{\bm{\hat{z}}}

\newcommand{\Xbar}{\bar{\vec{X}}}
\newcommand{\phat}{\bm{\hat{\perp}}}
\newcommand{\that}{\bm{\hat{\theta}}}

\newcommand{\eI}{\bm{\hat{e}}_1}
\newcommand{\eII}{\bm{\hat{e}}_2}
\newcommand{\ud}{\mathrm{d}}

%Derivatives etc.
\newcommand{\pfrac}[2]{\frac{\partial#1}{\partial#2}}
\newcommand{\ffrac}[2]{\frac{\delta#1}{\delta#2}}
\newcommand{\fixd}[1]{\Big{\arrowvert}_{#1}}
\newcommand{\curl}[1]{\nabla \times #1}

\newcommand{\np}{\vec{\nabla}_{\perp}}
\newcommand{\npc}{\nabla_{\perp} \cdot }
\newcommand{\nc}{\vec\nabla\cdot}
\newcommand{\cn}{\cdot\vec\nabla}
\newcommand{\vn}{\vec{\nabla}}
\newcommand{\npar}{\nabla_\parallel}

\newcommand{\GAI}{\Gamma_{1}^{\dagger}}
\newcommand{\GAII}{\Gamma_{1}^{\dagger -1}}
\newcommand{\T}{\mathrm{T}}
\newcommand{\Tp}{\mathcal T^+_{\Delta\varphi}}
\newcommand{\Tm}{\mathcal T^-_{\Delta\varphi}}
\newcommand{\Tpm}{\mathcal T^\pm_{\Delta\varphi}}
%%%%%%%%Some useful abbreviations %%%%%%%%%%%%%%%%
\def\feltor{{\textsc{Feltor }}}

\def\fixme#1{\typeout{FIXME in page \thepage :{#1}}%
 \textsc{\color{red}[{#1}]}}


\usepackage{minted}

\renewcommand{\ne}{\ensuremath{n }}
\newcommand{\Ni}{\ensuremath{N}}
\newcommand{\ue}{\ensuremath{u_\parallel}}
\newcommand{\Ui}{\ensuremath{U_\parallel}}
\newcommand{\Apar}{\ensuremath{A_\parallel}}
\newcommand{\bperp}{\ensuremath{ \vec b_\perp}}
\newcommand{\neref}{\ensuremath{n_0}}
\newcommand{\Teref}{\ensuremath{T_{e0}}} %\rhoN

%%%%%%%%%%%%%%%%%%%%%%%%%%%%%DOCUMENT%%%%%%%%%%%%%%%%%%%%%%%%%%%%%%%%%%%%%%%
\begin{document}

\title{Collisionless Reconnection}
\author{ M.~Held and M.~Wiesenberger}
\maketitle

\begin{abstract}
\end{abstract}

\section{Compilation and Usage}
Input file format: json\\
Output file format: NetCDF-4\\
File path: \texttt{feltor/src/reco2D/}
\begin{verbatim}
make reconnection device = <omp gpu>
make reconnection_hpc device = <omp gpu>
make reconnection_mpi device = <omp gpu>
\end{verbatim}
\texttt{reconnection} depends on both GLFW3 and NETCDF, while
\texttt{reconnection\_hpc} and \texttt{reconnection\_mpi} avoids the GLFW3 dependency and only operate with ``netcdf'' output.
Run with
\begin{verbatim}
path/to/feltor/src/reco2D/reconnection(_hpc) input.json <output.nc>
\end{verbatim}
The output file is only needed if you chose ``netcdf'' in the output field
of the input file.
\begin{verbatim}
echo 2 2 | mpirun -n 4 path/to/feltor/src/reco2D/reconnection_mpi input.json output.nc
\end{verbatim}
For the mpi program you need to provide the distribution of processes along each
axis on the command line.

\section{Equations}
%tearing instability <-> reconnection
Collisionless reconnection was studied in~\cite{Stanier2015}, gyro-fluid and gyro-kinetic studies~\cite{Comisso2013,Zacharias2014}.
Our reduced model is based on a recent formulation of a full-F gyro-fluid model~\cite{Madsen2013}.
It consists of the first two moment equations for electrons and ions
\begin{align}
%%%
\frac{\partial}{\partial t} \ne =&
 - \left[\phi, \ne\right]
+ \left[\Apar,\ne \ue  \right]
\\
%%%
\label{firstgyromom}
\frac{\partial}{\partial t} \Ni =&
 - \left[\psi, \Ni\right]
+ \left[\Gamma_1 \Apar ,\Ni \Ui  \right]
 \\
 %%%
\frac{\partial}{\partial t} \left( \ue+ \frac{1}{\mu_e} \Apar \right) =&
      -  \left[ \psi, \ue+ \frac{1}{\mu_e} \Apar  \right]% \nonumber \\
    +   \left[\Apar,\ue^2/2   \right]% \nonumber \\
      - \frac{1}{\mu_e}  \left[\Apar,\ln{\ne}   \right]
      \\
      \frac{\partial}{\partial t} \left( \Ui+ \Gamma_1 \Apar  \right) =&
      -  \left[ \psi, \Ui+ \Gamma_1 \Apar  \right]% \nonumber \\
     +   \left[\Gamma_1 \Apar,\Ui^2/2   \right]% \nonumber \\
      + \tau_i  \left[\Gamma_1 \Apar,\ln{\Ni}   \right]
\end{align}
which are coupled by polarisation and induction
\begin{align}
 -\ne + \Gamma_1 \Ni &= -\vec{\nabla}\cdot\left(\Ni \vec{\nabla}_\perp \phi\right) \\
 -\frac{1}{\beta} \vec{\nabla}_\perp^2 \Apar &=  -\ne \ue + \Gamma_1 (\Ni \Ui)
\end{align}
where the signs are such that the elliptic operators are positive definite and
with gyro-averaged parallel electromagnetic vector potential, generalized electric potential and gyro-averaging operator
\begin{align}
 \psi &:= \Gamma_1 \phi - u_E^2 /2  \\
  \Gamma_1 &:= (1-\frac{\tau_i}{2} \vec{\nabla}_\perp^2 )^{-1}
\end{align}
and ion gyro-radius, vacuum permeability and ion gyrofrequency
\begin{align}
  \rho_{i}   := \frac{\sqrt{T_{i} m_i}}{e B} \quad
  \mu_0 := 1/(\epsilon_0 c^2) \quad
  \Omega_i := e B / m_i
\end{align}
After exploiting the gyro-Bohm normalisation, our model is controlled by the dimensionless parameters. These
are the
\begin{align}
 \mu   :=  \frac{m}{Z m_i} \quad
 \tau  :=  \frac{T}{ Z T_i}\quad
 \beta_{e0} :=  \frac{\mu_0 \neref \Teref }{ B_0^2 }
\end{align}

Input file format: \href{https://en.wikipedia.org/wiki/JSON}{json} \\
\begin{minted}[texcomments]{js}
"physical" : // physical parameters
{
    "mu"   : -0.000544617, // the electron to ion mass ratio $\mu_e$
    "tau"  :  0.0,    // electron to ion temperature $\tau_i$
    "beta" :  1e-3    // plasma beta $\beta$
}
\end{minted}

\section{Initial and boundary conditions}
The boundary conditions for all variables are Dirichlet in $x$.
The $y$-direction is periodic.

The initial conditions are set in the file \texttt{feltor/src/reco2D/init.h}.

We initialize
\begin{align}
 \Ni&=\ne=1 \\
 \phi &= 0 \\
  \Ui&=0 \\
  \ue &= \frac{1}{\ne \beta }\vec{\nabla}^2_{\perp} \Apar
\end{align}
where we can choose $\Apar$.
\subsection{Harris sheet}
The initial parallel magnetic vector potential is
\begin{align}
    \Apar =\beta\left( A_0 / \cosh{( 4  \pi x / L_x )}^2 +A_1\cos{(2 m_y\pi y/L_y)}\right) \cos( \pi x /L_x)
\end{align}
The harris sheet initial condition can be chosen with the following parameters in the input file
\begin{minted}[texcomments]{js}
"init" : // Parameters for initialization
{
    "type" : "harris", // This choice necessitates the following parameters
    "amplitude0"  : 0.1, // harris amplitude
    "amplitude1"  : 1e-3, // perturbation amplitude (making this negative will
    // shift the perturbation by an angle $\pi$)
    "my"  :  1    // perturbation wave number in y
}
\end{minted}
We analytically compute the Laplacian of $\Apar$ with the help of Mathematica.
\subsection{Island}
The initial parallel magnetic vector potential is~\cite{Stanier2015}
\begin{align}
    \Apar =\beta \left[ A_0\frac{L_x}{4\pi} \ln \left\{ \cosh( 4\pi x/L_x) + \varepsilon \cos( 4\pi y/L_x)\right\} + A_1\cos{(2m_y \pi y/L_y)}\right] \cos( \pi x /L_x)
\end{align}
with $\varepsilon = 0.2$.
The island initial condition can be chosen with the following parameters in the input file
\begin{minted}[texcomments]{js}
"init" : // Parameters for initialization
{
    "type" : "island", // This choice necessitates the following parameters
    "amplitude0"  : 0.1, // island amplitude
    "amplitude1"  : 1e-3, // perturbation amplitude (making this negative will
    // shift the perturbation by an angle $\pi$)
    "my"  :  1    // perturbation wave number in y
}
\end{minted}
We analytically compute the Laplacian of $\Apar$ with the help of Mathematica.
(The island simulation takes approximately a factor 5 times smaller timestep than
the harris simulation).

\section{Invariants}
The mass density and diffusions are
\begin{align}
    \mathcal M &= \ne \\
    \vec j_n &= \ne \vec u_E + \ne\ue\bperp \\
     \Lambda_n &= -\nu_\perp \Delta_\perp^2 \ne
\end{align}
\begin{tcolorbox}[title=Note]
    We already incorporate the artificial diffusion terms defined in
    Section~\ref{sec:regularization}
\end{tcolorbox}

The inherent energy density of our system is:
\begin{align}
 \mathcal{E} := &
                    \ne \ln{(\ne)}
                  + \tau_i \Ni \ln{( \Ni)}
                  \nonumber \\
                 &- \frac{1}{2} \mu_e \ne \ue^2
                  + \frac{1}{2} \Ni \Ui^2
                  \nonumber \\
                 &+\frac{1}{2} \Ni u_E^2
                  + \frac{|\nabla_\perp \Apar|^2}{2 \beta}
\end{align}
The energy current density and diffusion are
\begin{align}
  \vec j_{\mathcal E} =& \sum_s z\left[
  \left(\tau \ln N + \frac{1}{2}\mu U_\parallel^2 + \psi \right)N\left(
  \vec u_E +U_\parallel\bperp  \right) + \tau NU_\parallel \bperp\right]
  , \\
    \Lambda_\mathcal{E} :=  &\left( (1+\ln \ne) + \frac{1}{2} z_e \mu_e \ue^2 - \phi \right) (-\nu_\perp \Delta_\perp^2 \ne)
    \nonumber\\
    &+ \left( \tau_i ( 1+\ln \Ni) + \frac{1}{2} \Ui^2 + \psi_i\right)(-\nu_\perp \Delta_\perp^2 \Ni)
    \nonumber\\
    &+ z_e\mu_e \ue\ne (-\nu_\perp \Delta_\perp^2 \ue)
    \nonumber\\
    &+ \Ui\Ni (-\nu_\perp \Delta_\perp^2 \Ui)
    \label{eq:energy_diffusion}
\end{align}
where in the energy flux $\vec j_{\mathcal E}$
we neglect terms  containing time derivatives
of the eletric and magnetic potentials and we sum over all species.

With our choice of boundary conditions both the mass as well as the energy flux
vanishes on the boundary. In the absence of artificial viscosity the volume integrated
mass and energy density are thus exact invariants of the system.
\begin{tcolorbox}[title=Note]
    For the canonical-viscosity scheme the viscosity terms in the last two
    terms of Eq.~\eqref{eq:energy_diffusion} change to the canonical velocity.
\end{tcolorbox}


\section{Reconnection rate}
 A consistent definition of the reconnection rate is not trivial~\cite{Comisso2016}.
 We adopt a similar definiton as in Ref.~\cite{Comisso2013}
\begin{align}
    Q_X&:= \Apar(\vec{x}_X,t) - \Apar( \vec{x}_X,0) \\
 \gamma&:= \frac{1}{Q_X}\frac{\d Q_X}{\d t} = \frac{\d \ln{|Q_X|}}{\d t}
\end{align}
where $\vec{x}_X = [0,0]$.
This can be easily evaluated in post-processing using a python script.

\section{Numerical methods}

\subsection{Spatial grid}
The spatial grid is a two-dimensional Cartesian product-grid $[-L_x/2, L_x/2]\times [-L_y/2, L_y/2]$ adaptable with the following parameters
\begin{minted}[texcomments]{js}
"grid" :
{
    "n"  :  3, // The number of polynomial coefficients
    "Nx"  : 48, // Number of cells in x
    "Ny"  : 48, // Number of cells in y
    "lxhalf"  : 80.0, // Half box length in x
    "lyhalf"  : 80.0, // Half box length in y
}
\end{minted}
\subsection{Timestepper}
The time stepper can be an explicit multistep method where you can chose the
tableau to use
\begin{minted}[texcomments]{js}
"timestepper" :
{
    "type"    : "multistep", //Choose an explicit multistep method
    "tableau" : "TVB-3-3", //  Any explicit multistep tableau *
    "dt"      : 20.0, // Fixed timestep
}
\end{minted}
The second option is an adaptive explicit embedded Runge-Kutta scheme
\begin{minted}[texcomments]{js}
"timestepper":
{
    "type"    : "adaptive", //Choose an explicit adaptive RK scheme
    "tableau" : "Tsitouras09-7-4-5", // Any explicit embedded RK tableau *
    "rtol"    : 1e-7, // The relative tolerance in the timestep
    "atol"    : 1e-10 // The absolute tolerance in the timestep
}
\end{minted}
*See the dg documentation for what tableaus are available.

\subsection{Advection scheme}
We implemented the advection in terms of Arakawa brackets
(note that this should only be used in connection with artificial diffusion)
or in terms of an upwind scheme.
This can be selected in the input file via
\begin{minted}[texcomments]{js}
"advection":
{
    "type" : "arakawa" // The Arakawa bracket scheme
    "type" : "upwind"  // The upwind scheme
}
\end{minted}
For the upwind scheme we first rewrite the equations in terms of
$\vec u_E := \zhat \times \nabla\phi = [-\partial_y \phi,\ \partial_x \phi]$ and
$\bperp := \nabla \Apar \times \zhat = [\partial_y \Apar,\ -\partial_x \Apar]$. We use
that $\nc \vec u_E = \nc \bperp =0$
\begin{align}
%%%
\frac{\partial}{\partial t} \ne =&
- ( \vec u_E + \ue \bperp)\cn \ne - \ne \bperp \cn  \ue
\\
%%%
\frac{\partial}{\partial t} \Ni =&
- ( \vec u_E^i + \Ui \bperp^i)\cn \Ni - \Ni \bperp^i\cn  \Ui
 \\
 %%%
\frac{\partial}{\partial t} \left( \ue+ \frac{1}{\mu_e} \Apar \right) =&
-  (\vec u_E + \ue \bperp ) \cn \ue
+ \frac{1}{\mu_e} \bperp \cn \ln \ne
- \frac{1}{\mu_e} \bperp \cn \phi
      \\
      \frac{\partial}{\partial t} \left( \Ui+ \Gamma_1 \Apar  \right) =&
    - (\vec u_E^i + \Ui \bperp^i ) \cn \Ui
    - \tau_i \bperp^i \cn \ln \Ni
    - \bperp^i \cn \psi
\end{align}
where the $i$ index signifies that $\vec u_E$ respectively $\bperp$ are to
be evaluated using the ion potentials $\psi$ and $\Apar$.
\begin{tcolorbox}[title=Note]
    We also tried to use the conservative upwind scheme, where the density equation
    is discretized in divergence form together with an interpolated multiplication
    scheme. This however was completely unstable for the harris sheet.
\end{tcolorbox}
\subsection{Regularization} \label{sec:regularization}
In order to prevent shocks and regularize the advection scheme
we implement an artificial viscosity of order 2.
\begin{align}
%%%
    \frac{\partial}{\partial t} \ne =& \ldots -\nu_\perp \Delta_\perp^2 \ne
\\
%%%
\frac{\partial}{\partial t} \Ni =& \ldots -\nu_\perp \Delta_\perp^2 \Ni
 \\
 %%%
\frac{\partial}{\partial t} \left( \ue+ \frac{1}{\mu_e} \Apar \right) =&
\ldots -\nu_\perp \Delta_\perp^2 \ue
      \\
      \frac{\partial}{\partial t} \left( \Ui+ \Gamma_1 \Apar  \right) =& \ldots -\nu_\perp \Delta_\perp^2 \Ui
\end{align}
Regularly we apply the artificial viscosity to the parallel velocities. However,
we can also apply it to the canonical velocity:
\begin{align}
 %%%
\frac{\partial}{\partial t} \left( \ue+ \frac{1}{\mu_e} \Apar \right) =&
\ldots -\nu_\perp \Delta_\perp^2 \left( \ue+ \frac{1}{\mu_e} \Apar \right)
      \\
      \frac{\partial}{\partial t} \left( \Ui+ \Gamma_1 \Apar  \right) =& \ldots -\nu_\perp \Delta_\perp^2 \left( \Ui+ \Gamma_1 \Apar  \right)
\end{align}
\begin{longtable}{llll}
\toprule
\rowcolor{gray!50}\textbf{Name} &  \textbf{Type} & \textbf{Value}  & \textbf{Description}  \\ \midrule
regularization & dict & & \\
\qquad type & string & velocity-viscosity & Apply viscosity on density and parallel velocity \\
\qquad type & string & canonical-viscosity & Apply viscosity on density and parallel canonical velocity \\
\qquad direction & string & centered & The direction of the Laplacian \\
\qquad nu\_perp & float & 1e-6  &  The strength of the diffusion \\
\bottomrule
\end{longtable}
Simply set the diffusion to 0 if you do not want any regularization.
\begin{tcolorbox}[title=Note]
    If you set nu\_perp too large the timestep in the explicit timestepper will
    have to be very small due to the restrictive CFL condition. At the same
    time in our experience having to make the timestep smaller due to viscosity
    is a clear warning that nu\_perp is too large for the chosen resolution.
    Chose nu\_perp as large as you can without decreasing the timestep.
\end{tcolorbox}

\subsection{Elliptic solvers}
In order to solve the elliptic equations we chose a multigrid scheme (nested
iterations). The accuaracies for the polarization equation can be chosen for
each stage separately, while for the Helmholtz type equations (The gamma
operators and the Ampere equation) only one accuracy can be set:
\begin{minted}[texcomments]{js}
"elliptic",
{
    "stages"    : 3,  // Number of stages (3 is best in virtually all cases)
    "eps_pol"   : [1e-6,10,10],
    // Accuracy requirement on each stage of the
    // multigrid scheme. $\eps_0 = \eps_{pol,0}$, $\eps_i = \eps_{pol,i} \eps_{pol,0}$  for $i>1$. \\
    "eps_gamma" : 1e-10, // Accuracy requirement of Gamma operator on density
    "eps_maxwell": 1e-7, //Accuracy requirement of Ampere equation
    "direction" : "forward", // Direction of the Laplacian: forward or centered
    "jumpfactor" : 1.0
    // Jump factor of Laplacian in polarization, Gamma and Ampere equations
}
\end{minted}

%..................................................................
\section{Output}
Our program can either write results directly to screen using the glfw library
or write results to disc using netcdf.
This can be controlled via
\begin{minted}[texcomments]{js}
"output":
{
    // Use glfw to display results in a window while computing (requires to
    // compile with the glfw3 library)
    "type"  : "glfw"
    "itstp"  : 4, // The number of steps between outputs of 2d fields \\
    // Use netcdf to write results into a file (filename given on command line)
    // (see next section for information about what is written in there)
    "type"  : "netcdf"
    "itstp"  : 4,    // The number of steps between outputs of 2d fields \\
    "maxout"  : 500, // The total number of field outputs. The endtime is
    // T=itstp*maxout*dt
    "n"  : 3 , // The number of polynomial coefficients in the output file\\
    "Nx" : 48, // Number of cells in x in the output file \\
    "Ny" : 48  // Number of cells in y in the output file
}
\end{minted}
The number of points in the output file can be lower (or higher) than the number of
grid points used for the calculation. The points will be interpolated from the
computational grid.
\subsection{Netcdf file}
Output file format: netcdf-4/hdf5

\begin{longtable}{lll>{\RaggedRight}p{7cm}}
\toprule
\rowcolor{gray!50}\textbf{Name} &  \textbf{Type} & \textbf{Dimension} & \textbf{Description}  \\ \midrule
inputfile        & text attribute & 1 & verbose input file as a string \\
time             & Coord. Var. & 1 (time) & time at which fields are written \\
x                & Coord. Var. & 1 (x) & x-coordinate  \\
y                & Coord. Var. & 1 (y) & y-coordinate \\
xc               & Dataset & 2 (y,x) & Cartesian x-coordinate  \\
yc               & Dataset & 2 (y,x) & Cartesian y-coordinate \\
X                & Dataset & 3 (time, y, x) & 2d outputs \\
X\_1d            & Dataset & 1 (time) & 1d volume integrals $\int \dV X$ \\
time\_per\_step  & Dataset & 1 (time) & Average computation time for one step \\
\bottomrule
\end{longtable}
The output fields X are determined in the file \texttt{feltor/src/reco2D/diag.h}.

%..................................................................
\bibliography{../../doc/related_pages/references, references}
%..................................................................
\end{document}
