%%%%%%%%%%%%%%%%%%%%%definitions%%%%%%%%%%%%%%%%%%%%%%%%%%%%%%%%%%%%%%%

%\documentclass[12pt]{article}
%\documentclass[12pt]{scrartcl}
\documentclass{hitec} % contained in texlive-latex-extra
\settextfraction{0.9} % indent text
\usepackage{csquotes}
\usepackage[hidelinks]{hyperref} % doi links are short and usefull?
\hypersetup{%
    colorlinks=true,
    linkcolor=blue,
    urlcolor=magenta
}
\urlstyle{rm}
\usepackage[english]{babel}
\usepackage{mathtools} % loads and extends amsmath
\usepackage{amssymb}
% packages not used
%\usepackage{graphicx}
%\usepackage{amsthm}
%\usepackage{subfig}
\usepackage{bm}
\usepackage{longtable}
\usepackage{booktabs}
\usepackage{ragged2e} % maybe use \RaggedRight for tables and literature?
\usepackage[table]{xcolor} % for alternating colors
%\rowcolors{2}{gray!25}{white} %%% Use this line in front of longtable
\renewcommand\arraystretch{1.3}
\usepackage[most]{tcolorbox}
\usepackage{doi}
\usepackage[sort,square,numbers]{natbib}
\bibliographystyle{abbrvnat}
%%% reset bibliography distances %%%
\let\oldthebibliography\thebibliography
\let\endoldthebibliography\endthebibliography
\renewenvironment{thebibliography}[1]{
  \begin{oldthebibliography}{#1}
    \RaggedRight % remove if justification is desired
    \setlength{\itemsep}{0em}
    \setlength{\parskip}{0em}
}
{
  \end{oldthebibliography}
}
%%% --- %%%

\definecolor{light-gray}{gray}{0.95}
\newcommand{\code}[1]{\colorbox{light-gray}{\texttt{#1}}}
\newcommand{\eps}{\varepsilon}
\renewcommand{\d}{\mathrm{d}}
\renewcommand{\vec}[1]{{\boldsymbol{#1}}}
\newcommand{\dx}{\,\mathrm{d}x}
%\newcommand{\dA}{\,\mathrm{d}(x,y)}
%\newcommand{\dV}{\mathrm{d}^3{x}\,}
\newcommand{\dA}{\,\mathrm{dA}}
\newcommand{\dV}{\mathrm{dV}\,}

\newcommand{\Eins}{\mathbf{1}}

\newcommand{\ExB}{$\bm{E}\times\bm{B} \,$}
\newcommand{\GKI}{\int d^6 \bm{Z} \BSP}
\newcommand{\GKIV}{\int dv_{\|} d \mu d \theta \BSP}
\newcommand{\BSP}{B_{\|}^*}
\newcommand{\Abar}{\langle A_\parallel \rangle}
%Averages
\newcommand{\RA}[1]{\left \langle #1 \right \rangle} %Reynolds (flux-surface) average
\newcommand{\RF}[1]{\widetilde{#1}} %Reynolds fluctuation
\newcommand{\FA}[1]{\left[\left[ #1 \right]\right]} %Favre average
\newcommand{\FF}[1]{\widehat{#1}} %Favre fluctuation
\newcommand{\PA}[1]{\left \langle #1 \right\rangle_\varphi} %Phi average

%Vectors
\newcommand{\ahat}{\bm{\hat{a}}}
\newcommand{\bhat}{\bm{\hat{b}}}
\newcommand{\chat}{\bm{\hat{c}}}
\newcommand{\ehat}{\bm{\hat{e}}}
\newcommand{\bbar}{\overline{\bm{b}}}
\newcommand{\xhat}{\bm{\hat{x}}}
\newcommand{\yhat}{\bm{\hat{y}}}
\newcommand{\zhat}{\bm{\hat{z}}}

\newcommand{\Xbar}{\bar{\vec{X}}}
\newcommand{\phat}{\bm{\hat{\perp}}}
\newcommand{\that}{\bm{\hat{\theta}}}

\newcommand{\eI}{\bm{\hat{e}}_1}
\newcommand{\eII}{\bm{\hat{e}}_2}
\newcommand{\ud}{\mathrm{d}}

%Derivatives etc.
\newcommand{\pfrac}[2]{\frac{\partial#1}{\partial#2}}
\newcommand{\ffrac}[2]{\frac{\delta#1}{\delta#2}}
\newcommand{\fixd}[1]{\Big{\arrowvert}_{#1}}
\newcommand{\curl}[1]{\nabla \times #1}

\newcommand{\np}{\vec{\nabla}_{\perp}}
\newcommand{\npc}{\nabla_{\perp} \cdot }
\newcommand{\nc}{\vec\nabla\cdot}
\newcommand{\cn}{\cdot\vec\nabla}
\newcommand{\vn}{\vec{\nabla}}
\newcommand{\npar}{\nabla_\parallel}

\newcommand{\GAI}{\Gamma_{1}^{\dagger}}
\newcommand{\GAII}{\Gamma_{1}^{\dagger -1}}
\newcommand{\T}{\mathrm{T}}
\newcommand{\Tp}{\mathcal T^+_{\Delta\varphi}}
\newcommand{\Tm}{\mathcal T^-_{\Delta\varphi}}
\newcommand{\Tpm}{\mathcal T^\pm_{\Delta\varphi}}
%%%%%%%%Some useful abbreviations %%%%%%%%%%%%%%%%
\def\feltor{{\textsc{Feltor }}}

\def\fixme#1{\typeout{FIXME in page \thepage :{#1}}%
 \textsc{\color{red}[{#1}]}}



%%%%%%%%%%%%%%%%%%%%%%%%%%%%%DOCUMENT%%%%%%%%%%%%%%%%%%%%%%%%%%%%%%%%%%%%%%%
\begin{document}

\title{Testing Advection Schemes}
\author{ M.~Wiesenberger}
\maketitle

\begin{abstract}
  This is a program to test various advection schemes on the 2d incompressible Euler
  equation used in Reference~\cite{Einkemmer2014}.
\end{abstract}

\section{Equations}
We implement the 2d incompressible Euler equation
\begin{subequations}
\begin{align}
 \frac{\partial \omega}{\partial t} + \{ \phi, \omega\} = 0 \\
 -\Delta \phi = \omega \label{eq:euler_poisson_elliptic}
\end{align}
\label{eq:euler_poisson}
\end{subequations}
with vorticity $\omega$ and stream-function $\phi$.
The Poisson bracket is given by $\{ \phi, \omega\} := \phi_x \omega_y - \phi_y \omega_x$.
Eq.~\eqref{eq:euler_poisson} is a reformulation of the standard conservative form
\begin{subequations}
\begin{align}
    \frac{\partial \omega}{\partial t} + \nabla\cdot({\vec v \omega}) = 0 \\
\nabla\cdot\vec v = 0 \quad \omega = -(\nabla\times \vec v)\cdot \zhat
\end{align}
\label{eq:euler_conservative}
\end{subequations}
with $v_x = - \phi_y$ and $v_y = \phi_x$.

Eqs.~\eqref{eq:euler_poisson} have an infinite amount of conserved quantities
among them the total vorticity $V$, the kinetic energy $E$ and the enstrophy $\Omega$
 \begin{align}
     V := \int_D \omega \dA\quad
     E :=\frac{1}{2} \int_D \left( \nabla \phi\right)^2 \dA \quad
     \Omega:= \frac{1}{2} \int_D \omega^2 \dA
 \end{align}


\section{Initialization}
We will consider several different initial conditions in order to test
our numerical methods
\subsection{Lamb Dipole}
The Lamb dipole is a stationary solution to the Euler equations~\cite{Nielsen1997} with infinite
boundary conditions
\begin{align}
    \omega(x,y,0) = \begin{cases}
        \frac{2\lambda U}{J_0(\lambda R)} J_1(\lambda R) \cos \theta,\ r < R,\\
        0, \text{ else}
    \end{cases}
\end{align}
Unfortunately, for a finite box this is not an exact solution any more.
on the domain $[0,1]\times [0,1]$.
The Lamb dipole is chosen with the following parameters in the input file
%%This is a booktabs table
\begin{longtable}{llll}
\toprule
\rowcolor{gray!50}\textbf{Name} &  \textbf{Type} & \textbf{Value}  & \textbf{Description}  \\ \midrule
grid & dict &  & Grid parameters \\
\qquad x & float[2]& [0,1] & Choose x box boundaries appropriately \\
\qquad y & float[2]& [0,1] & Choose y box boundaries appropriately \\
\qquad bc & string[2] & [PER, PER] & Choose boundary conditions [x,y] appropriately \\
init &  dict &   & Parameters for initialization \\
\qquad type      & string & lamb & This choice necessitates the following parameters \\
\qquad velocity  & float &  1    &  blob speed \\
\qquad sigma     & float &  0.1  & dipole radius in units of lx \\
\qquad posX      & float &  0.5  & in units of lx \\
\qquad posY      & float &  0.8  & in units of ly \\
\bottomrule
\end{longtable}
\subsection{Manufactured Solution}
We manufacture a solution via
\begin{align}
    \phi(x,y,t) &=
    x \exp\left( - \frac{ x^2 + (y+vt)^2}{\sigma^2}\right) \\
    \omega(x,y,t) &= -\Delta \phi = -\sigma^{-4} \left[ 4\phi(x,y,t) ( x^2-2\sigma^2  + (y+tv)^2)\right]
\end{align}
which is solution to the modified equations
\begin{subequations}
\begin{align}
    \frac{\partial \omega}{\partial t} + \{ \phi, \omega\} = S(x,y,t) \\
    -\Delta \phi = \omega
\end{align}
\label{eq:euler_poisson_modified}
\end{subequations}
with the source
\begin{align}
    S(x,y,t) =& 8 x \sigma^{-6}(y+vt)\exp\left( - 2\frac{ x^2 + (y+vt)^2}{\sigma^2} \right) \nonumber\\
    &\left(-\sigma^2  + \exp\left( \frac{ x^2 + (y+vt)^2}{\sigma^2} \right) v( -3\sigma^2 + x^2 + (y+vt)^2) \right)
\end{align}
on the domain $[-1,1]\times [-1,1]$.

The manufactured solution is chosen with the following parameters in the input file
\begin{longtable}{llll}
\toprule
\rowcolor{gray!50}\textbf{Name} &  \textbf{Type} & \textbf{Value}  & \textbf{Description}  \\ \midrule
grid & dict &  & Grid parameters \\
\qquad x & float[2]& [-1,1] & Choose x box boundaries appropriately \\
\qquad y & float[2]& [-1,1] & Choose y box boundaries appropriately \\
\qquad bc & string[2] & [DIR, PER] & Choose boundary conditions appropriately \\
init &  dict &   & Parameters for initialization \\
\qquad type      & string & mms & This choice necessitates the following parameters \\
\qquad sigma      & float & 0.2 & The width $\sigma$ \\
\qquad velocity   & float & 1 & The velocity $v$ \\
\bottomrule
\end{longtable}
\subsection{Simple sine function}
A simple sine function is given by
\begin{align}
    \omega(x,y,0) = 2 \sin(x)\sin(y)
\end{align}
which has an analytical solution
\begin{align}
\omega(x,y,t) = 2 \sin(x)\sin(y)\exp( -(2\nu)^s t)
\end{align}
if there is artificial viscosity and is invariant else.
on the domain $[0,2\pi]\times [0,2\pi]$.

This solution is chosen with the following parameters in the input file
\begin{longtable}{llll}
\toprule
\rowcolor{gray!50}\textbf{Name} &  \textbf{Type} & \textbf{Value}  & \textbf{Description}  \\ \midrule
grid & dict &  & Grid parameters \\
\qquad x & float[2]& [0, 6.283185307179586] & Choose x box boundaries appropriately \\
\qquad y & float[2]& [0, 6.283185307179586] & Choose y box boundaries appropriately \\
\qquad bc & string[2] & [DIR, PER] & Choose boundary conditions appropriately \\
init &  dict &   & Parameters for initialization \\
\qquad type      & string & sine & No other parameters are necessary \\
\bottomrule
\end{longtable}

\subsection{ Double Shear layer}
Here, we follow~\cite{Liu2000} and test the scheme on a double shear layer problem.
\begin{align}
    \omega(x,y,0) = \begin{cases}
        \delta \cos(x) - \frac{1}{\rho} \text{sech}^2 \left(\frac{y-\pi/2}{\rho}\right),\ y \leq \pi \\
        \delta \cos(x) + \frac{1}{\rho} \text{sech}^2 \left(\frac{3\pi/2-y}{\rho}\right),\ y > \pi \\
    \end{cases}
\end{align}
where $\rho = \pi/15$ and $\delta =0.05$ on the domain $[0,2\pi]\times [0,2\pi]$.
This solution will quickly roll-up and generate smaller and smaller scales.
A thin shear layer corresponds to $\rho = \pi/50$ or smaller.
The double shear layer initialization is chosen with the following parameters in the input file
%%This is a booktabs table
\begin{longtable}{llll}
\toprule
\rowcolor{gray!50}\textbf{Name} &  \textbf{Type} & \textbf{Value}  & \textbf{Description}  \\ \midrule
grid & dict &  & Grid parameters \\
\qquad x & float[2]& [0, 6.283185307179586] & Choose x box boundaries appropriately \\
\qquad y & float[2]& [0, 6.283185307179586] & Choose y box boundaries appropriately \\
\qquad bc & string[2] & [PER, PER] & Choose boundary conditions appropriately \\
init &  dict &   & Parameters for initialization \\
\qquad type      & string & shear & This choice necessitates the following parameters \\
\qquad rho    & float & 0.20943951023931953 & The width $\rho = \pi/15$ \\
\qquad delta  & float & 0.05 & The velocity $v$ \\
\bottomrule
\end{longtable}

\section{Numerical methods}
Our goal is to try out various time integration and advection discretization techniques.
We know from Godunov's theorem
that any linear advection scheme of order 2 or higher is prone to oscillations.
\subsection{Spatial grid}
The spatial grid is a two-dimensional Cartesian product-grid adaptable with the following parameters
\begin{longtable}{llll}
\toprule
\rowcolor{gray!50}\textbf{Name} &  \textbf{Type} & \textbf{Value}  & \textbf{Description}  \\ \midrule
grid & dict & & \\
\qquad n  & integer & 3  & The number of polynomial coefficients \\
\qquad Nx & integer & 48 & Number of cells in x \\
\qquad Ny & integer & 48 & Number of cells in y \\
\qquad  x & float[2]& [0,1] & Boundaries in x \\
\qquad  y & float[2]& [0,1] & Boundaries in y \\
\qquad bc & string[2] & [DIR, PER] & Boundary conditions in [x,y] \\
\bottomrule
\end{longtable}
\subsection{Time steppers}
Possible time-steppers are the explicit and semi-implicit multistep schemes
as well as the Shu-Osher scheme that originally incorporated limiters in the dG scheme.
\begin{longtable}{lllp{6cm}}
\toprule
\rowcolor{gray!50}\textbf{Name} &  \textbf{Type} & \textbf{Value}  & \textbf{Description}  \\ \midrule
timestepper & dict & & \\
\qquad type     & string& ImExMultistep & The semi-implicit multistep scheme (only in combination with viscosity regularization) \\
\qquad tableau  & string & Any ImEx tableau & for example ImEx-BDF-3-3* \\
\qquad dt       & float & 2e-3 & Fixed timestep \\
\qquad eps\_time & float & 1e-9 & Accuracy requirement for implicit solver \\
timestepper & dict & & \\
\qquad type & string & Shu-Osher & An explicit Runge Kutta method with filter (viscosity is treated explicitly) \\
\qquad tableau   & string & Any Shu-Osher tableau & for example SSPRK-3-3* \\
\qquad dt      & float & 1e-3 & Fixed Time-step \\
timestepper & dict & & \\
\qquad type & string & FilteredExplicitMultistep & an explicit multistep class with the option to use a filter (viscosity is treated explicitly)\\
\qquad tableau   & string & Any explicit multistep tableau & for example eBDF-3-3* \\
\qquad dt      & float & 2e-3 & Fixed timestep \\
\bottomrule
\end{longtable}
*See the dg documentation for what tableaus are available.
\subsection{Regularization technique}
Choose either no regularization or artificial viscosity or modal filtering by the following
parameters in the input file.

For no regularization choose
\begin{longtable}{lllp{7.5cm}}
\toprule
\rowcolor{gray!50}\textbf{Name} &  \textbf{Type} & \textbf{Value}  & \textbf{Description}  \\ \midrule
regularization & dict & & \\
\qquad type  & string& none & No regularization\\
\bottomrule
\end{longtable}

For artificial viscosity Eqs.~\eqref{eq:euler_poisson} are modified to
\begin{subequations}
\begin{align}
    \frac{\partial \omega}{\partial t} + \{ \phi, \omega\} = -(-\nu \Delta)^s \omega\\
 -\Delta \phi = \omega
\end{align}
\label{eq:euler_poisson_viscous}
\end{subequations}
where $\nu$ is the viscosity coefficient and $s=1,2,3,\cdots$ is the order
\begin{longtable}{lllp{7.5cm}}
\toprule
\rowcolor{gray!50}\textbf{Name} &  \textbf{Type} & \textbf{Value}  & \textbf{Description}  \\ \midrule
regularization & dict & & \\
\qquad type  & string& viscosity & the artificial viscosity \\
\qquad order    & integer & 2 & Order: 1 is normal diffusion, 2 is hyperdiffusion, can be arbitrarily high, but higher orders might take longer to solve or restrict the CFL condition \\
\qquad nu    & float & 1e-3 & Viscosity coefficient \\
\qquad directoin & string & centered & Direction of the Laplacian: forward or centered
\bottomrule
\end{longtable}
The other regularization method is the modal filter that applies an exponential filter
\begin{align}
    \begin{cases}
    1 \text{ if } \eta < \eta_c \\
    \exp\left( -\alpha  \left(\frac{\eta-\eta_c}{1-\eta_c} \right)^{2s}\right) \text { if } \eta \geq \eta_c \\
    0 \text{ else} \\
    \eta := \frac{i}{n-1}
    \end{cases}
\end{align}
and is choosable with the following parameters
\begin{longtable}{lllp{7.5cm}}
\toprule
\rowcolor{gray!50}\textbf{Name} &  \textbf{Type} & \textbf{Value}  & \textbf{Description}  \\ \midrule
regularization & dict & & \\
\qquad type  & string& modal & Not choosable for \textbf{ImExMultistep} timestepper\\
\qquad order & integer & 8 & Order: normally 8 or 16 \\
\qquad eta\_c & float & 0.5 & cutoff wavelength below which no damping is applied \\
\qquad alpha & float & 36 & damping coefficient determining damping for highest wavenumber \\
\bottomrule
\end{longtable}
%%%%%%%%%%%%%%%%%%%%%%%%%%%%%
\subsection{Elliptic solver}
Currently, the multigrid solver is the only one choosable
to solve Eq.~\eqref{eq:euler_poisson_elliptic}
\begin{longtable}{lllp{7.5cm}}
\toprule
\rowcolor{gray!50}\textbf{Name} &  \textbf{Type} & \textbf{Value}  & \textbf{Description}  \\ \midrule
elliptic & dict & & \\
\qquad type  & string& multigrid & Actually a nested iterations class \\
\qquad stages    & integer & 3 & Number of stages (3 is best in virtually all cases) \\
\qquad eps\_pol    & float[stages] & [1e-6,10,10] & Accuracy requirement on each stage of the multigrid scheme. $\eps_0 = \eps_{pol,0}$, $\eps_i = \eps_{pol,i} \eps_{pol,0}$  for $i>1$. \\
\qquad directoin & string & centered & Direction of the Laplacian: forward or centered
\bottomrule
\end{longtable}
\subsection{Advection schemes}
The following parameters control the advection scheme in the code
\begin{longtable}{lllp{7.5cm}}
\toprule
\rowcolor{gray!50}\textbf{Name} &  \textbf{Type} & \textbf{Value}  & \textbf{Description}  \\ \midrule
advection & dict & & \\
\qquad type  & string& arakawa & Discretize Eqs.~\eqref{eq:euler_poisson} using Arakawa's scheme \cite{Einkemmer2014} \\
\qquad type  & string& centered & Discretize Eqs.~\eqref{eq:euler_conservative} using the centered flux \\
\qquad type  & string& upwind & Discretize Eqs.~\eqref{eq:euler_conservative} using the upwind flux \\
\qquad multiplication    & string & pointwise & The multiplications in the scheme
are done pointwise in nodal space\\
\qquad multiplication    & string & projection & The multiplications in the scheme
are done by first interpolating to a higher polynomial grid with $n_{fine} = 2n -1$, and projecting the result back to the coarse grid\\
\bottomrule
\end{longtable}
%\subsection{Forward time and centered space}
%It is well known that the forward in time, centered in space method for solving
%hyperbolic systems is unconditionally unstable~\cite{LeVeque}.
%\subsection{Arakawa scheme and centered flux}
%Reference~\cite{Liu2000} reports that the centered flux does not have any numerical
%diffusion while the upwind flux does.
%They also prove that upwind and centered fluxes do not dissipate energy.
%From finite differences we know that centered differences for the advection term
%are unstable (or at least produce a lot of oscillations).

\section{Compilation and useage}
The program shu\_b.cu compiles with
\begin{verbatim}
make <shu_b> device = <omp gpu>
make <shu_hpc> device = <omp gpu>
\end{verbatim}
and depends on both GLFW3 and NETCDF. If GLFW3 is not available then compile shu\_hpc which avoids this dependency.
Run with
\begin{verbatim}
path/to/feltor/src/shu/shu_b input.json
\end{verbatim}

\subsection{Output structure}
Input file format: json

We can either display the results in real-time to screen using the glfw3 library or
write the results to a file in netcdf-4 format.
This is regulated by the output paramters in the input file
\begin{longtable}{lllp{7cm}}
\toprule
\rowcolor{gray!50}\textbf{Name} &  \textbf{Type} & \textbf{Value}  & \textbf{Description}  \\ \midrule
output & dict & & \\
\qquad type  & string& glfw & Use glfw to display results in a window while computing (requires to compile with the glfw3 library) \\
\qquad type  & string& netcdf & Use netcdf to write results into a file (see next section for information about what is written in there) \\
\qquad itstp  & integer& 4 & The number of steps between outputs of 2d fields \\
\qquad maxout  & integer& 500 & The total number of field outputs. The endtime is T=itstp*maxout*dt \\
\bottomrule
\end{longtable}
\subsection{Structure of output file}
Output file format: netcdf-4/hdf5
%
%Name | Type | Dimensionality | Description
%---|---|---|---|
\begin{longtable}{lll>{\RaggedRight}p{7cm}}
\toprule
\rowcolor{gray!50}\textbf{Name} &  \textbf{Type} & \textbf{Dimension} & \textbf{Description}  \\ \midrule
inputfile  &             text attribute & 1 & verbose input file as a string \\
time                     & Coord. Var. & 1 (time) & time at which fields are written \\
x                        & Coord. Var. & 1 (x) & x-coordinate  \\
y                        & Coord. Var. & 1 (y) & y-coordinate \\
vorticity                & Dataset & 3 (time, y, x) & electon density $n$ \\
potential                & Dataset & 3 (time, y, x) & electric potential $\phi$  \\
vorticity\_1d            & Dataset & 1 (time) & Vorticity integral $V$  \\
enstrophy\_1d            & Dataset & 1 (time) & Enstropy integral $\Omega$  \\
energy\_1d               & Dataset & 1 (time) & Total energy integral computed using $E = \int_D \phi\omega \dA$ \\
time\_per\_step          & Dataset & 1 (time) & Average time for one output \\
error                    & Dataset & 1 (time) & Relative error to analytical solution if available, 0 else \\
\bottomrule
\end{longtable}
The output fields are determined in the file \texttt{feltor/src/lamb\_dipole/diag.h}.

%..................................................................
\bibliography{../../doc/related_pages/references}
%..................................................................


\end{document}
